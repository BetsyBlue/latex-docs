% vim:spelllang=uk
\documentclass[a4paper,10pt,notitlepage,pdftex,headsepline]{scrartcl}

\usepackage{cmap} % чтобы работал поиск по PDF
\usepackage[utf8]{inputenc}
\usepackage[russian]{babel}
\usepackage[T2A]{fontenc}
\usepackage{concrete}
\usepackage{fullpage}

\pdfcompresslevel=9 % сжимать PDF
\usepackage{pdflscape} % для возможности альбомного размещения некоторых страниц
\usepackage[pdftex]{hyperref}
% настройка ссылок в оглавлении для pdf формата
\hypersetup{unicode=true,
            pdftitle={Гражданская оборона},
            pdfauthor={Михаил Погода},
            pdfcreator={pdflatex},
            pdfsubject={},
            pdfborder    = {0 0 0},
            bookmarksopen,
            bookmarksnumbered,
            bookmarksopenlevel = 2,
            pdfkeywords={},
            colorlinks=true, % установка цвета ссылок в оглавлении
            citecolor=black,
            filecolor=black,
            linkcolor=black,
            urlcolor=blue}

\author{Михаил Погода}
\title{Гражданская оборона}
\date{\today}

\begin{document}
  \maketitle

  Радіаційно"=небезпечними є такі, що використовують або зберігають
  радіоактивні речовини.
  Всі воно характеризуються здатністю проникати в середовище та іонізувати
  молекули.

  Аварії на атомних станціях:
  \begin{itemize}
    \item[0] --- Не мають значення.
    \item[1--2] --- Не носять загрози для людей.
    \item[3] -- Події, пов’язані з втратою одного з елементів захисту або
      незначний витік речовини в середовище (без перевищення встановлених
      меж).
    \item[4--7] --- Аварії, пов’язані з значним викидом радіоактивних речовин в
      навколишнє середовище або пошкодження активної зони атомного ректору.
  \end{itemize}

  Вражаючий фактор --- є радіоактивне випромінювання (РАВ).
  Основним параметром є рівень радіації (Р)

  $\alpha = 0.3$ --- РБМК, $\alpha = 0.4$ --- ВВЕР, $\alpha = 1.2$ --- ядерний
  вибух.

  Вплив на людей здійснюється шляхом зовнішнього опромінення або коли
  радіоактивні речовини потрапляють всередину разом з повітрям, їжею, водою.
  Для людини найбільш небезпечне є зовнішнє гамма-, та внутрішнє альфа- і
  бета-.

  Ступені ураження людей будуть залежати від отриманої дози та від часу
  впродовж якого ця доза отримана.
  Час реакції людського організму на опромінення складає 96 годин.

  Доза може бути експозиційна, поглинутою (кількість радіоактивних
  випромінювань одиницею маси тіла), еквівалентною (1 бер = 1 рентгену
  гамма"=опромінювання).

  Людину опромінюють малими дозами тривалий час --- зростає кількість лейкемій,
  онкологічних захворювань, зростає ймовірність генетичних або спадкових
  хвороб.

  Людину опромінюють великими дозами короткий проміжок часу (до чотирьох днів)
  --- може розвиватись гостра променева хвороба.

  Якщо людина отримує дозу до 50 рентген не втрачає працездатності.

  100--200 рентген --- перший (легкий) ступінь. Одужують усі.

  200--300 рентген --- другий (середній) ступінь. Температура 38 градусів,
  раптові кровотечі. Вмирають 20\% людей.

  300--600 рентген --- третій (важкий) ступінь. Починаються повний розлад
  кишкового тракту. Температура 40 градусів. Вмирає 50\% людей.

  Понад 600 рентген --- четвертий (надважкий) ступінь. Вмирають усі на 3--5
  день.

  Загальні наслідки опромінення: ураження ЦНС, система кровообігу та
  кровотворення, шлунково"=кишкового тракту, загальна інтоксикація організму.
\section{Наслідки надзвичайних ситуацій природного характеру}
  До надзвичайних ситуацій природного характеру відносять прояви явищ сил
  природи (стихійні лиха), які не підвладні людині, виникають, як правило,
  несподівано та можуть викликати порушення життєдіяльності людини, їх
  загибелі, або значні матеріальні втрати.

  За походженням поділяються на:
  \begin{itemize}
    \item Геологічні (землетруси, тощо);
    \item Метеорологічні (бурі тощо);
    \item Гідрологічні (повені, паводки, підвищення рівня ґрунтових вод);
    \item Природні пожежі;
    \item Масові захворювання людей, рослин, тварин.
  \end{itemize}

  В залежності від сфери прояву:
  \begin{itemize}
    \item Літосферні;
    \item Гідросферні;
    \item Атмосферні;
    \item Космічні.
  \end{itemize}

  \subsection{Геологічні}
    Землетрус --- стихійні підземні поштовхи та коливання земної поверхні,
    які виникають, як правило, несподівано під час раптових зсувів або
    розривів у земній корі або в верхній частині мантії.
    Такі коливання передаються на значні відстані.

    Вражаючим фактором землетрусу будуть пружні коливанню земної поверхні.

    За руйнуючої дією землетруси порівняні з атомним вибухом.
    Основним параметром є сила землетрусу (магнітуда за Ріхтером --- оцінює
    загальну енергію).

    Поштовхи 2 бали та нище --- не відчутні для людей. 3 бали ---
    відчутні для деяких людей всередині споруд. 4 бали --- відчуваються
    більшістю людей, пошкодження відступні. 5 балів --- невеликі пошкодження.
    6 балів --- помірні пошкодження. 7 балів --- великі пошкодження. 8--9 балів
    --- загальне (практично повне) руйнування.

    Існує інша школа --- РСК-64. Вимірює загальну інтенсивність енергії, що
    розповсюджується по поверхні землі.
    5- балів --- слабкі, 6--7 балів --- сильні, 8--9 --- руйнуючі, 9--12 ---
    катастрофічні.
    %TODO: Методика розрахування осередку під час землетрусу

    Для території України землетруси катастрофічні для ЗУ та для Криму.

    Дії населення під час землетрусу. При відчутті перших поштовхів необхідно
    покинути споруду, при цьому не можна користуватися ліфтом.
    Після цього відійти від будівлі на безпечну відстань.
    Якщо покинути споруду своєчасно не вдається, необхідно стати в прорізі
    вікна або дверей в несучій стіні.
    Якщо опинилися під завалом не можна нервуватися, користуватися
    запальничками, рухатися.

    Попередні заходи: сейсмостійке будівництво, підготовка рятувальних служб з
    ліквідації наслідків землетрусу, навчання населення правилам поведінки під
    час землетрусу.
  \subsection{Наслідки повеней і затоплень}
    Повень --- тимчасове затоплення суші водою в результаті інтенсивного
    танення снігу, зливних дощів, льодових заторів, руйнування дамб.
    Повені характерні особливо для прибережних територій та Закарпаття.

    Вражаючим фактором повені є руйнуюча дія мас води.
    Основним параметром затоплень є швидкісний натиск, яке буде враховувати
    густину води, швидкість води.
    До інших основних параметрів відносяться час затоплення, висота затоплень.

    Дії населення під час повеней: розрізняються два випадки:
      1. Повідомлення про повень отримано заздалегідь --- перекрить газ, воду,
      електрику, прикрити вікна, евакуатись.
      2. Раптово --- надавати допомогу дітям, літнім людям та жінкам.
      Необхідно піднятись на верхні поверхи й там чекати на допомогу.
      Якщо вода буде підніматись дуже швидко, треба плити проти течі.
      %TODO Виписати формули, за якими розраховується екстремальні
  \subsection{Бурі}
    Буря --- пересування мас повітря зі значною швидкістю. Швидкість ветру
    сягає 30 м/с.

    По руйнуючій дії урагани порівнюються з землетрусом.

    Вражаючим фактором є руйнуюча сила мас повітря.
    Основним параметром є швидкісний натиск.

    Характерні для всієї України.

    Дії населення:
      Щільно зачинити вікна та двері, прибрати речі що могуть травмувати
      людей. Загасити вогонь в пічках.
  \subsection{Лісові та торфяні пожежі}
    Лісові та торфяні пожежі поділяють на низові (6 км/г), верхові (25 км/г),
    підземні ( декілька метрів за добу).


\section{Захист населення та території у надзвичайних ситуаціях}
  \subsection{Принципи та способи захисту населення}
    1. Організація та проведення засобів захисту відбувається на території
    всієї країни на всіх підприємствах.
    Приоритетність віддається захисту життя.
    2. Планування та підготовка засобів захисту відбувається заздалегіть з
    урахуванням планів, економічного та соціального розвитку регіону або
    об’єкту.

    3. Заходи захисту організуються та здійснюються диференційовано з
    урахуванням політичного, економічного та оборонного значення.
    Вся відповідальність лежить на керівнику об’єкту.
    Населення та підлеглі мають право на інформацію.

    Основні способи захисту.
      1. Проведення евакуаційних заходів.
      2. Укриття в захисних спорудах.
      3. Використання індивідуальних засобів захисту.

  Евакуація --- переселення людей з небезпечного району в безпечний.
  Варіанти евакуації:
    1. З однієї частини держави в іншу. (Під час бойових дій)
    2. Із великих міст та промислових районів в сільську місцевість.
    3. З території де виникла надзвичайна ситуація до безпечного району. (Під
    час великих стихійних лих або аварій).

  При цьому під час евакуації розрізняють розосередження робітників та власне
  евакуація населення.
  Основні способи евакуації:
    Вивезення всіма видами транспорту.
    Виведення пішими колонами.
    Комбінований.

  Розосереджуваних робітників вивозять в заміську зону (зона, де надмірний
  тиск не буде перевищувати 10 атм).
  Їх розміщують таким чином, щоб час, який вони витрачатимуть на дорогу на
  роботу не перевищував 4-5 годин.

  Евакуація транспортом передумовлена для Розосереджуваних робітників,
  населення яке не може рухатись пішки.

  Формування цівільного захисту.

  Піші колони формують по 500--1000 чоловік.
  Розподіляють на групи по 50--100 чоловік.
  Висувають другорядними маршрутами.
  Середня швидкість руху --- 3--4 км / год.
  Величина добового переходу --- 30--40 км.
  Кожну годину робиться малий привал на 10 хв.
  Раз на добу --- великий на 2 години.

  Евакуаційни заходи включать:
    Оповіщення та збір населення.
    Реєстрацію людей та формування колон.
    Рух в район розміщення.
    Прийом та розміщення людей на новому місці.
    Всебічне забезпечення евакуації.


  У великих місцях створюють евакокомісії.
  В містах створюють збірні евакопункти.
  У сільскій місцевості створюють приймальні евакокомісії.
  В місцях переходу створюють проміжні евакопункти.

  Рішення про проведення евакуації приймає:
    у мирний час --- комісія з надзвичайних ситуацій місцевого, районного,
    обласного або державного рівня.
    При загрозі війни --- лише уряд країни.

  Засоби колективного захисту.
    Сховища, протирадіаційні укриття, найпростіші укриття.

    Сховищами є споруди, що призначені для захисту людей від усіх вражаючих
    факторів.
    Характеризуються надмірним тиском та коефіцієнтом ослаблення рівня
    радіації.
    В залежности від цього їх поділяють на 4 класи:
      >500 кПА
      >300 кПа
      >200 кПа
      >100 кПа

    Вимоги до сховищ:
      Надійно захищати людей >2 доб,
      Розсташовані не далі, ніж 500 м.
      Не вміщувати менше 150 людей.

    Класифікація сховищ:
      Вбудовані або Розсташовані окремо.

    Приміщення, що повинні бути в сховищах:
      Основні (приміщення для людей, медпункту, пункту управління ЦЗ )
      Допоміжні (фільтро-вентиляційне обладнення, дізель, електрощітова,
        санузли, продукти харчування).

    Кожне сховище повинно мати не менш ніж два входи та один аварійний вихід
    (90х150 горизонтальна галерея).

    Норми приміщень для укриваємих людей.
      Висота стелі: 2.15м--2.9м, 2.9м--3.(1)5м
      Площа: 0.5 м2 / люд.
      Об’єм повітря: 1.5 м3 / люд.
      Системи життєзабезпечення: повітрє-постачання, водо-постачання,
        електро-постачання, опалення, каналізація.

      1. Режим чистої вентиляції. Вмикається після заповнення сховища людьми
      та гермитизації. При цьому зовнишнє повітря очищується на
      противо-пильних фільтрах. Норма подачі на одну людину буде залежати від
      кліматичної зони: (перша --- 8 м^3 / год. х люд., друга --- 10 м3 / год.
      х люд., третя --- 11 м3 / год х люд, четверта --- 13 м3/год. х люд).
      2. Режим фільтро-вентиляції. Вмикається при загрозі хімічного зараження.
      При цьому зовнишнє повітря після противопильних фільтрах очіщується на
      фільтрах-поглиначаї отруйних речовин. Норма: 2 м3/год х люд.
      3. Режим повної ізоляції з регенирації внутрішнього повітря.
      Вмикається якщо отруйні речовини
      \ldots\ldots\ldots\ldots\ldots\ldots\ldots
      Внутрішнє повітря очищується від CO2, збагачується киснєм.
      Норми: поглинається CO2 --- 25л/год х люд, подається O2 --- 20 л\ldots.


      Норми\ldots норма\ldots допустиме
      Кисень --- 20--21\%; min -- 15\%
      CO2        1\%        max 3\%
      CO          0\%       max 0.01\%
      T           23        max 31
      Вологість  <70        max 80
      Тиск    +5--7мм ніж назовні,   допускається такий самий.

      Водопостачання.
      3л на людину на добу.

      Електропостачання.
      Живиться ззовні + аварійне джерело.
      < 600 осіб --- аккамуляторні батареї.
      > 600 осіб --- дизельна електростанція.

      Система опалення.

      Каналізація.


  ПРУ. Характеризуються коефіцієнтом ослаблення.
    1. > 200 раз
    2. > 100 раз
    3. > 50 раз
  Від сховищ відрізняються тим, що висота стелі > 190 см.


  Найпростіші укриття. --- усі необладнані підвали та погреби, щілі.


  Індивідуальні засоби захисту --- засоби захисту органів дихання, шкіри та
  медичні.
    Дихання --- протигази, респіратори, протипильні тканинні маски.
      Протигази --- фільтруючі та на ізолюючі.



\end{document}
