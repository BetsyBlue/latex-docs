\documentclass[a5paper, 10pt, notitlepage, pdftex,headsepline]{scrartcl}

\usepackage{cmap}
\usepackage[utf8]{inputenc}
\usepackage[russian]{babel}
\usepackage[T2A]{fontenc}
\author{Михаил Погода}
\title{КР по ОС \textnumero~1}
\date{\today}


\begin{document}
\section{Режимы работы ЭВМ. Эволюция ОС}
  \subsection{Режим непосредственного доступа}
    Самым простым и естественным является однопрограммный режим непосредственного доступа.
    Этот режим работы первых ЭВМ.
    Он предусматривает отсутствие операционных систем и доступ
    пользователя к средствам управления и индикации.
    В первых ЭВМ состав программного (математического) обеспечения часто
    входил только пакет прикладных программ (программ решения основных
    научно технических задач).

    При работе на ЭВМ пользователь должен написать и набрать на носитель
    (для первых ЭВМ это, часто, перфокарты) проблемную программу или
    выбрать ее из пакета прикладных программ.
    Но с любой выбранной программой с использование ЭВМ можно сделать
    множество операций: распечатать, скопировать на тот или иной
    носитель, модифицировать и т.\,д.

    Опыт эксплуатации ЭВМ первого и частично второго поколения выявил
    недостатки этого режима.
    Это медленная реакция пользователя на различные ситуации в ЭВМ,
    требующие реагирования со стороны пользователя, включая медленный
    ручной набор данных при их вводе.
  \subsection{Режим однопрограммной пакетной обработки}
    На смену этого режима пришел режим однопрограммной пакетной обработки.
    Этот режим предусматривал отсутствие пользователя, как причины замедления работы ЭВМ.
    В качестве компенсации присутствия пользователя используется
    резидентная программа --- диспетчер, для которой пользователи пишут
    программу (последовательность заданий на языке команд диспетчера)
    работы с этой программой (вызвать транслятор, компилировать
    программу и т.\,д.) для всех возможных вариантах развития событий.

    Эти программы на языке диспетчера и составляют пакет.

    Таким образом, при пакетном режиме программист должен писать
    программы на языках программирования и, на языке команд диспетчера ---
    программу заданий по использованию программы ,,диспетчером''.

    Для реализации этого режима ЭВМ должна быть снабжена система
    прерывания и операционная система, по крайней мере, в виде
    диспетчера и набора программ обработки прерываний.

    Однопрограммный пакетный режим работы значительно сократил простои
    ЭВМ, но оставался еще резерв.
  \subsection{Режим мультипрограммной пакетной обработки}
    При обработке программ в однопрограммном пакете работают,
    поочередно, две системы: процессор и система ввода вывода.
    Организация их параллельной работы является резервом повышения
    производительности ЭВМ. Но для этого требуется многопрограммная
    пакетная обработка.

    Это режим мультипрограммной пакетной обработки или ,,режим
    классического мультипрограммирования''.

    Цель режима --- минимизация простоев процессора при обработке пакета
    программ.

    Стратегия режима заключается в следующем:
    \begin{enumerate}
      \item Часть первых программ в пакете переводится в состоянии
        ,,задача'' но, возможно, на разных стадиях,
      \item Часть из них может находиться в состоянии ожидания.
        Это задачи, для которых производится загрузка данных.
      \item Другая часть программ может находиться в состоянии счет.
        Это задачи, для которых загружены данные и они готовы к
        обработке на процессоре.
        Одна из них находится в стадии выполнения команд процессором.
        Остальные --- ожидают своей очереди на обработку.
    \end{enumerate}

    Таким образом, производится совмещение процедур обработки данных с
    их загрузкой и сохранением.
    Возможности мультипрограммной обработки определяются архитектурными
    особенностями ЭВМ и используемой операционной системой.

    Режим классического мультипрограммирования уменьшает время простоев
    процессора, но не предусматривает повышения качества обслуживания
    пользователей.

    Дальнейшая модификация режима заключалась введением приоритетов
    программ в пакете и циклического переключения программ.

    Введение приоритетов (по параметрам программ с учетом соотношений
    времен операций ввода/вывода и процессорной обработки) предназначено
    для упрощения работы программы планировщика пакетов, а циклическое
    переключение программ --- для вывода процессора из зацикливаний.

    Переключение задач на обслуживание процессором производится:
    \begin{itemize}
      \item по команде окончания в программе,
      \item по команде обращения к устройствам ввода/вывода,
      \item по переходу в состояние задачи программы с более высоким
        уровнем приоритета,
      \item по окончанию тайм-аута обработки данной задачи.
    \end{itemize}

    Введение приоритетов программ позволило перейти к новому режиму
    работы ЭВМ --- режиму разделения времени (режим коллективного
    доступа).
  \subsection{Режим коллективного доступа}
    Это режим совмещение двух режимов --- разделения времени и
    непосредственного доступа.

    Это возврат к непосредственному доступу, но с сохранением пакетной
    обработки.
    В систему добавляются терминалы (устройство ввода, например,
    клавиатура, и вывода, например, дисплей) и системная резидентная
    программа обслуживания терминалов, как задача с наивысшим уровнем
    приоритета.

    Режим коллективного доступа строится с использованием терминалов на
    основе режима классического программирования.

    Программы пакета обрабатываются процессоров, в виде фонового пакета.
    Кроме программ фонового пакета на обработку в процессор могут
    поступать задачи с терминалов от пользователей, работающих в режиме
    непосредственного доступа за терминалами.

    Доступ с терминалов предназначался для работ по отладки программ в
    прямом доступе, обращения к разным справочникам, многоабонентном
    обслуживании, например по продажи и оформление билетов и т.\,д.

    При нажатии любой клавиши на терминале в процессор ЭВМ поступает
    сигнал прерывания.
    Программа обработки прерывания переключает процессор на обработку
    задач с терминалов.
    Так, как производительность ЭВМ намного выше человека, ЭВМ во многих
    случаях могла удовлетворять запросы пользователей без взаимных помех.
    Кроме этого, для разрешения возможных конфликтов между запросами,
    каждый запросы фиксируется на регистре первой очереди и с него, по
    очереди, поступают на обработку.

    На обработку каждого запроса отводился определенный квант времени.
    Если он достаточен для решения, ответ передается на терминал источника.
    Если квант времени был недостаточным для формирования ответа,
    дальнейшая обработка запроса откладывается с фиксацией его на
    регистре второй очереди.
    Запросы с регистра второй очереди поступают на обработку, только
    после обработки всех запросов регистра первой очереди.
    На обработку запросов с регистра второй очереди отводится квант
    времени значительно большей длительности.
    Если квант времени снова недостаточен для обработки, запрос
    переводится на регистр третий очереди, а затем в пакет фоновой обработки.
    В любом случае результат обработки запроса поступает на терминал с
    большей или меньшей задержкой.

    Эта стратегия реализует стратегию: на простой запрос --- ответ
    мгновенный, на сложный вопрос --- с задержкой, величина которой
    возрастает в разы с ростом сложности вопроса.
    Здесь простота вопроса определяется временем обработки запроса.

    Но в чистом виде этот режим работы продержался недолго.
    Причина --- появление персональных компьютеров.
    Оказалось, что PC по стоимости сравним со стоимостью терминала (с
    учетом стоимости средств удаленного доступа, так как терминалы
    располагались вне машинного зала, например, в кабинетах и даже в
    других зданиях и т.\,д.).
  \subsection{Режим клиент-сервер}
    Персональные компьютеры изначально позиционировались, как дешевое
    средство обработки данных ,,одно для одного'', для широкого круга
    пользователей, не специалистов в области программирования.
    Режим коллективного доступа не мог конкурировать по стоимости услуг
    и комфортности работы на РС.

    Но, несмотря на то, что РС изначально предназначались для работы
    ,,один для одного'' их стали объединять в локальные сети.
    Для этого было достаточно много причин.
    Это, в первую очередь, повышение производительности парка ЭВМ в
    случаях их коллективной эксплуатации, например коллективом фирмы, за
    счет возможности их более полной загрузки.
    Отдельные пользователи могли загружать полезной работой все
    свободные ЭВМ в сети.

    Кроме этого, для более трудоемких вычислений в парк персональных ЭВМ
    можно было вводить более производительные ЭВМ в качестве рабочих станций.
    В качестве рабочих станций использовались более дорогие и
    производительные РС, ,,бывшие'' мини-ЭВМ или ЭВМ общего назначения.
    Работа в сети упрощало решение проблемы их более полной загрузки работой.

    По сути дела это был возврат к системе коллективного доступа, но в
    рамках сетевой ,,распределенной обработки''.

    Для повышения эффективности использования рабочих станций их стали
    конфигурировать под определенные сервисные услуги и применения.
    Такие рабочие станции стали называть серверами.
    Тип сервера определяется видом ресурса, которым он владеет (файловая
    система, база данных, принтеры, процессоры или прикладные пакеты программ).
    Появились файл-серверы, серверы базы данных, принт-серверы,
    вычислительные серверы, серверы приложений и т.\,д.

    Появилось понятие ,,клиент-сервер''. Но ,,клиент-сервер'' --- это
    название не только режима, но и способа программирования.
    При работе в режиме клиент-сервер программа состоит из двух
    взаимодействующих программ --- программы клиент и программы сервер.

    Программа ,,клиент'' ставится на рабочем месте оператора
    (пользователя), а программа ,,сервер'' на одном из серверов.

    Программа ,,клиент'' выполняет функции посредника между пользователем и сервером.
    Программа сервер --- это целевая программа обработки данных.

    Программа клиент принимает от оператора задание, определяет
    соответствующий сервер, передает ему задание пользователя, принимает
    решение задачи от сервера и отображает его на экране в форме удобной
    для восприятия пользователем.
\section{Системное программное обеспечение компьютерной системы}
  Системное программное обеспечение --- комплекс программных средств,
  которые предназначены для повышения эффективности использования
  мощностей ЭВМ, облегчения её эксплуатации, понижения трудоёмкости
  работы по проектированию и исполнению программных продуктов,
  предоставление пользователю ЭВМ разнообразных услуг.

  В состав СПО КС входят:
  \begin{itemize}
    \item \textbb{Операционные системы} --- обязательный компонент ЭВМ, который
      организует работу и взаимодействие пользователя с КС.
    \item \textbb{Сервисные системы} расширяют возможности ОС, придавая
      пользователю и выполняемым программам наборы дополнительных услуг.
      По этой причине иногда сервисные системы относят к составу ОС.
      \begin{itemize}
        \item Интерфейсные системы;
        \item оболочки ОС;
        \item утилиты.
      \end{itemize}
    \item \textbb{Инструментальные системы} --- объединение разнообразных
      системных программных средств, которые используются для разработки
      программных продуктов, хотя некоторая их часть может
      использоваться для решения прикладных задач.
      Использование большинства инструментальных систем связано с
      разработкой программ; по этой причине они могут считаться
      \textit{системами программирования}.
      Однако к системам программирования традиционно относят такие
      системы, с помощью которых можно запрограммировать и решить любую
      задачу, которая допускает алгоритмическое решение.
      Другие типы инструментальных систем являются специализированными,
      т.\,е. используются для проектирования программного обеспечения
      определённого функционального назначения.
      \begin{itemize}
        \item Системы программирования;
        \item системы управления базами данных;
        \item инструментарий искусственного интеллекта;
        \item текстовые редакторы;
        \item интегрированные системы.
      \end{itemize}
    \item \textbb{Системы технического обслуживания} предназначены для
      облегчения тестирования оборудования и поиска неисправностей.
  \end{itemize}
\section{Место и роль ОС в компьютерных системах}
  \begin{itemize}
    \item Занимает промежуточное место между аппаратурой и пользователем
      ЭВМ;
    \item обеспечивает аппаратуру теми возможностями, которые сложно или
      экономически невыгодно реализовать аппаратным способом;
    \item создаёт условия для эффективной разработки и настройкой
      программ;
    \item понижает стоимость использования вычислительных средств за
      счёт разделения между пользователями ресурсов ЭВМ и накопленной
      информации.
  \end{itemize}

  Таким образом, ОС --- это основа всей системы программного обеспечения
  современной ЭВМ, которая определяет операционную среду, где работают
  пользователи.
\section{Зоопарк ОС. Структура ОС}
  
5.архитектура(подсистемы) ОС. Режимы работы программного обеспечения
6. Основные понятия ОС. Системные вызовы.
7. Аппаратное обеспечение компьютерной системы. Шины(10 видов шин как
минимум)
8. Аппаратное обеспечение компьютерной системы. Память
9. Аппаратное обеспечение компьютерной системы. Устройство ввода/вывода
10.Аппаратное обеспечение компьютерной системы. Процессор
11.Модульный принцип программ. Обработка исходного модуля.
12. Виртуальная память
13. Прерывания, их типы. Механизмы обработки прерывания
14. Защита памяти в ЭВМ
15. Таймер и часы. Микропрограммирование и ОС
\end{document}
