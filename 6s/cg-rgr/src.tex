\documentclass[a4paper,12pt,notitlepage,pdftex,headsepline]{scrartcl}

\usepackage{a4wide}
\usepackage{cmap} % чтобы работал поиск по PDF
\usepackage[utf8]{inputenc}
\usepackage[russian]{babel}
\usepackage[T2A]{fontenc}

\usepackage{textcase}
\usepackage[pdftex]{graphicx}

\usepackage{lscape}

\pdfcompresslevel=9 % сжимать PDF
\usepackage{pdflscape} % для возможности альбомного размещения некоторых страниц
\usepackage[pdftex]{hyperref}
% настройка ссылок в оглавлении для pdf формата
\hypersetup{unicode=true,
            pdftitle={РГР по КГ},
            pdfauthor={Погода Михаил},
            pdfcreator={pdflatex},
            pdfsubject={},
            pdfborder    = {0 0 0},
            bookmarksopen,
            bookmarksnumbered,
            bookmarksopenlevel = 2,
            pdfkeywords={},
            colorlinks=true, % установка цвета ссылок в оглавлении
            citecolor=black,
            filecolor=black,
            linkcolor=black,
            urlcolor=blue}

\usepackage{amsmath}
\usepackage{amssymb}
\usepackage{moreverb}

\author{Михаил Погода}
\title{РГР ро КГ}
\date{\today}

\begin{document}
  \thispagestyle{empty}
  \begin{center}
    \large
    \MakeUppercase{Министерство образования и науки,}

    \MakeUppercase{молодёжи и спорта Украины}

    \MakeUppercase{Национальный технический университет Украины}

    \MakeUppercase{,,Киевский политехнический институт''}

    \addvspace{6pt}

    \normalsize
    Кафедра прикладной математики

    \vfill

    \textbf{Расчётно-графическая работа}

    по курсу ,,Компьютерная графика''

    \addvspace{6pt}

    на тему:
    \texttt{,,Алгоритм Люка для заполнения области''}
  \end{center}

  \vfill

  \begin{flushright}
    Студента

    группы КМ-92

    Погоды Михаила
  \end{flushright}

  \vfill

  \begin{center}
    КИЕВ

    2012
  \end{center}
  \newpage

  \begin{center}\bf \large Краткие теоретические сведения
  \end{center}
    Во всех алгоритмах заполнения области, которые используют построчную
    развёртку, используется память изображения, в которой выполняются
    кодирование и заполнение.

    Во многих случаях при использовании алгоритма контроля чётности можно
    получать ошибочные результаты, вызванные тем, что встреча текущей точки с
    контуром не изменяет контекста.

    Алгоритм Люка позволяет преодолеть эти трудности, использую следующие
    приёмы:
    \begin{itemize}
      \item для горизонтальных участков используется такой код, под действием
        которого отрезок окрашивается, но не изменяет контекст;
      \item для вершины, общей для двух рёбер, находящихся по одну сторону от
        горизонтального отрезка, проходящего через эту вершину, код ведёт к
        окрашиванию, но без изменения контекста;
      \item для перекрытий, которые могут быть вызваны либо пересечением двух
        рёбер, либо недостаточно высокой разрешающей способностью в процессе
        дискретизации, алгоритм Люка не изменяет контекст, если их количество
        --- нечётное.
    \end{itemize}

    Алгоритм, блок-схема которого реализована далее, необходимы следующие
    соглашения:
    \begin{itemize}
      \item структура, хранящая координаты точки, причём функция \texttt{x(P)}
        возвращает абсциссу точки $P$, а функция \texttt{y(P)} --- её
        ординату.
      \item некоторая коллекция точек, причём константа $p_0$ указывает на
        первую точку в этой коллекции, а функция \texttt{next\_p()} возвращает
        следующую точку из этой коллекции.
      \item константы $X_{min}, X_{max}, Y_{min}, Y_{max}$, задающие окно,
        содержащее закрашиваемые многоугольник.
    \end{itemize}
\end{document}
