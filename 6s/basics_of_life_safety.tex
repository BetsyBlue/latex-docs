\documentclass[a4paper,12pt,notitlepage,pdftex,headsepline]{scrartcl}

\usepackage{a4wide}
\usepackage{cmap} % чтобы работал поиск по PDF
\usepackage[utf8]{inputenc}
\usepackage[russian]{babel}
\usepackage[T2A]{fontenc}

\usepackage{textcase}
\usepackage[pdftex]{graphicx}

\usepackage{lscape}

\pdfcompresslevel=9 % сжимать PDF
\usepackage{pdflscape} % для возможности альбомного размещения некоторых страниц
\usepackage[pdftex]{hyperref}
% настройка ссылок в оглавлении для pdf формата
\hypersetup{unicode=true,
            pdftitle={Безопасность жизнедеятельности},
            pdfauthor={Погода Михаил},
            pdfcreator={pdflatex},
            pdfsubject={},
            pdfborder    = {0 0 0},
            bookmarksopen,
            bookmarksnumbered,
            bookmarksopenlevel = 2,
            pdfkeywords={},
            colorlinks=true, % установка цвета ссылок в оглавлении
            citecolor=black,
            filecolor=black,
            linkcolor=black,
            urlcolor=blue}

\usepackage{amsmath}
\usepackage{amssymb}
\usepackage{moreverb}
%for \includepdf
%\usepackage{pdfpages}

\author{Михаил Погода}
\title{Безопасность жизнедеятельности}
\date{\today}

\begin{document}
Полукаров Юрий Алексеевич

Желибо ,,Основы безопасности жизнедеятельности''
\section{Основные понятия дисциплины БЕЗОПАСНОСТЬ ЖИЗНЕДЕЯТЕЛЬНОСТИ}
Жизнь является наивсшей формой существования материи.
Закономерно возникает в определённых условиях в процессе развития материи.

Живым существам свойственны обмен веществ, раздражительность, способность к размножению и росту,
изменчивость, способность приспосабливаться к окружаещей среде.

Окружаящая среда --- природные элементы окружения или исскуственно созданные человеком объекты и явления, с которыми она находится в прямых или относительных взаимоотношениях.

Деятельность --- специфическая способность человека форма активного отношения к окружающей среде, и ёё смысл состоит в необходимом изменении окружающей среде.

Человек --- наивсшая ступень живых организмов и предмет изучения множества областей наук.

Жизнедеятельность --- осмысленная деятельность человека, направленная на ёё самореализацию, с учётом жизненных потребностей и способностей.

Опасность --- негативное свойство материи, которая проявляется в способности причинить вред как живым, так и неживым объектам.

Вред --- качественная или количественная оценка убытка причинённого опасностью.

Источниками опасности могут быть природные процессы и явления, техногенная среда, человеческие действия.

Безопасность --- процесс деятельности, при котором с определённой вероятностью исключается возможность опасности.

Безопасность жизнедеятельности --- наука, предметом изучения которой является безопасность человека при выполнении её любой деятельности.

Исследования в области безопасности жизнедеятельности направлены на выявление основных опасностей и на разработку основных способов защиты от этих опасностей.
\end{document}



















