\documentclass[a4paper,12pt,notitlepage,pdftex,headsepline]{scrartcl}

\usepackage{a4wide}
\usepackage{cmap} % чтобы работал поиск по PDF
\usepackage[utf8]{inputenc}
\usepackage[russian]{babel}
\usepackage[T2A]{fontenc}

\usepackage{textcase}
\usepackage[pdftex]{graphicx}

\usepackage{lscape}

\pdfcompresslevel=9 % сжимать PDF
\usepackage{pdflscape} % для возможности альбомного размещения некоторых страниц
\usepackage[pdftex]{hyperref}
% настройка ссылок в оглавлении для pdf формата
\hypersetup{unicode=true,
            pdftitle={Реферат по политологии},
            pdfauthor={Погода Михаил},
            pdfcreator={pdflatex},
            pdfsubject={},
            pdfborder    = {0 0 0},
            bookmarksopen,
            bookmarksnumbered,
            bookmarksopenlevel = 2,
            pdfkeywords={},
            colorlinks=true, % установка цвета ссылок в оглавлении
            citecolor=black,
            filecolor=black,
            linkcolor=black,
            urlcolor=blue}

\usepackage{amsmath}
\usepackage{amssymb}
\usepackage{moreverb}
%for \includepdf
%\usepackage{pdfpages}

\author{Михаил Погода}
\title{Реферат по политологии}
\date{\today}

\begin{document}
  \thispagestyle{empty}
  \begin{center}
    \large
    \MakeUppercase{Министерство образования и науки,}

    \MakeUppercase{молодёжи и спорта Украины}

    \MakeUppercase{Национальный технический университет Украины}

    \MakeUppercase{,,Киевский политехнический институт''}

    \addvspace{6pt}

    \normalsize
    Кафедра прикладной математики

    \vfill

    \textbf{Реферат}

    по дисциплине ,,Политология''

    на тему: ,,Политические конфликты''
  \end{center}

  \vfill

  \begin{flushright}
    студента\\
    группы КМ-92\\
    Погоды Михаила
  \end{flushright}

  \vfill

  \begin{center}
    КИЕВ

    2012
  \end{center}
  \clearpage
  \tableofcontents
  \clearpage
\section{Введение}
  Конфликт (лат. ,,конфлитус'' --- столкновение) --- столкновение двух или
  более разнонаправленных сил с целью реализации их интересов в условиях
  противодействия; это серьезное разногласие, острый спор, чреватый
  осложнениями и борьбой.
  Конфликтами пронизана вся жизнь человека, конфликты охватывают все сферы
  общества.
  Однако в период построения социалистического государства изучение
  конфликтов, особенно социально-политических, мало занимало
  ученых-обществоведов.
  Считалось, что социалистическое общество может развиваться и развивается
  бесконфликтно, гармонично.
  На самом же деле конфликты не гасились, не разрешались, а заглушались,
  создавая тем самым напряженность в обществе (принцип сжатой пружины).
  Из всех сфер общества самой насыщенной различными видами конфликтов является
  политическая сфера, в которой развертываются многообразные властные
  отношения, представляющие собой отношения господства и подчинения.

  Основным объектом политического конфликта выступает политическая власть как
  способ и средство господства одного социального слоя (класса) над другим.
  Интересы же людей, принадлежащих к этим группам, не только различны, но и
  противоположны: те группы, которые обладают властью, заинтересованы в ее
  удержании, сохранении и укреплении, те же, которые власти лишены и не имеют
  к ней доступа, заинтересованы, чтобы изменить существующее положение,
  добиться перераспределения власти.
  Именно поэтому они и вступают в конкурентные взаимодействия, осознанным
  воплощением которых выступает политический конфликт.
  \clearpage
\section{Исторические концепции политических конфликтов}
  Политический конфликт --- это столкновение противоположных общественных сил,
  обусловленное определенными взаимоисключающими политическими интересами и
  целями.

  Проблема политического конфликта стара, как мир.
  Древние философы, изучая общество, пытались определить источник развития.
  Китайские и древнегреческие философы видели источник всего сущего в
  противоположностях, в их взаимодействии, в борьбе противоположностей.
  В той или иной форме подобные мысли высказывали Анаксимандр, Сократ, Платон,
  Эпикур и др.
  Впервые попытку анализа конфликта, как социального явления, предпринял
  А.~Смит в своей работе ,,Исследования о природе и причинах богатства
  народов'' (1776~г.).
  В основе конфликта, считал А.~Смит, лежат деление общества на классы и
  экономическое соперничество, которое рассматривалось им как важнейшая
  движущая сила общества.

  Важное значение для исследования конфликтов имело учение Гегеля о
  противоречиях и борьбе противоположностей.
  Это учение легло в основу теории К.~Маркса о причинах политических
  конфликтов.
  В соответствии с теорией Маркса политические разногласия обусловлены
  социально-экономическими структурами.
  Общество делится на неравноправные классы, это неравенство порождает
  глубокий антагонизм; в свою очередь, антагонизм является основой
  политической борьбы.
  Политическая борьба есть борьба классовая.

  Во времена Маркса его теория была достаточно точным описанием главных
  тенденций политического конфликта.
  Великие политические битвы 19-го --- начала 20-го веков были в основном
  классовыми в марксистском понимании термина ,,класс''.
  Другие же факторы --- расовые, религиозные, национальные --- были лишь
  маскировкой классовых интересов и имели второстепенное значение, считали
  марксисты.

  В современных условиях эта теория не может адекватно отражать все причины
  политических конфликтов, потому что:
  \begin{enumerate}
    \item марксизм преувеличивает значение классового конфликта в формировании
      политических различий;
    \item марксизм дает слишком узкое определение класса (через отношение к
      средствам производства);
    \item класса, как социальной группы, каким его понимал Маркс, сегодня
      просто не существует.
  \end{enumerate}

  Во второй половине XX-го века наибольшую известность получили взгляды на
  конфликт М.~Дюверже (Франция), Л.~Козера (США), Р.~Дарендорфа (Германия) и
  К.~Боулдинга (США).

  Морис Дюверже построил свою теорию на единстве конфликта и интеграции.
  По его мнению, в любом обществе существуют как конфликт, так и интеграция, и
  эволюция интеграции никогда не снимет все общественные конфликты.

  Льюис Козер считает, что обществу всегда присуще неравенство и
  психологическая неудовлетворенность его членов.
  Это приводит к напряженности, выливающейся в конфликт.

  Ральф Дарендорф обосновал ,,конфликтную модель общества''.
  Согласно этой теории конфликт вездесущ, пронизывает все сферы общества, и
  изменения в обществе происходят под воздействием конфликтов.
  Структурные изменения в обществе происходят из-за неравенства социальных
  позиций людей по отношению к власти, что вызывает трения, антагонизмы,
  конфликты.

  Кеннет Боулдинг считает, что конфликт неотделим от общественной жизни.
  Стремление к борьбе с себе подобными, к эскалации насилия лежит в природе
  человека.
  То есть сущность конфликта лежит в стереотипных реакциях человека.
  В связи с этим Боулдинг полагает, что конфликт можно преодолевать и
  разрешать, манипулируя ценностями, влечениями, реакциями индивидуумов, не
  прибегая к радикальному изменению существующего общественного строя.

  \clearpage
\section{Политический кризис}
  Большой интерес вызывают те конфликты, которые затрагивают основу
  политической системы --- государство.
  Подобные конфликты характеризуются как политические кризисы.

  \textbf{Политический кризис} --- это состояние политической системы
  общества, выражающееся в углублении и обострении имеющихся конфликтов, в
  резком усилении политической напряженности.
  Другими словами, политический кризис можно охарактеризовать как перерыв в
  функционировании какой-либо системы с позитивным для нее или негативным
  исходом.

  Политические кризисы можно разделить на внешнеполитические и
  внутриполитические:
  \begin{itemize}
    \item Внешнеполитические кризисы обусловлены международными противоречиями
      и конфликтами и затрагивают несколько государств.
    \item Внутриполитические кризисы --- это:
      \begin{itemize}
        \item правительственный кризис --- потеря правительством авторитета,
          невыполнение его распоряжений местными исполнительными органами;
        \item парламентский кризис --- расхождение решений законодательной
          власти с мнением большинства граждан страны или изменение
          соотношения сил в парламент;
        \item конституционный кризис --- фактическое прекращение действия
          Основного закона страны;
        \item социально-политический (общенациональный) кризис --- включает в
          себя все три вышеназванных, затрагивает основы общественного
          устройства и вплотную подводит к смене власти.
      \end{itemize}
  \end{itemize}

  Политические конфликты и кризисы соотносятся таким образом, что конфликт
  может быть началом кризиса и кризис может служить основанием конфликта.
  Конфликт по времени и протяженности может включать в себя несколько
  кризисов, и совокупность конфликтов может составлять содержание кризиса.

  \clearpage
\section{Причины политических конфликтов}
  В качестве основной, универсальной причины конфликта можно назвать
  несовместимость претензий сторон при ограниченности возможностей их
  удовлетворения.
  Причинами конфликтов также являются:
  \begin{itemize}
    \item Вопросы власти.
      Люди занимают неравное положение в системе иерархий: одни управляют,
      командуют, другие --- подчиняются.
      Может сложиться ситуация, когда недовольными бывают не только
      подчиненные (несогласие с управлением), но и управляющие
      (неудовлетворительное исполнение).
    \item Нехватка средств к существованию.
      Недостаточно полное или ограниченное получение средств вызывает
      недовольство, протесты, забастовки, митинги и т.\,д., что объективно
      нагнетает напряженность в обществе.
    \item Следствие непродуманной политики.
      Принятие властными структурами поспешного, неотмоделированного решения
      может вызвать недовольство большинства народа и способствовать
      возникновению конфликта.
    \item Несовпадение индивидуальных и общественных интересов.
    \item Различие намерений и поступков отдельных личностей, социальных
      групп, партий.
    \item Зависть.
    \item Ненависть.
    \item Расовая, национальная и религиозная неприязнь и др.
  \end{itemize}

  \clearpage
\section{Типология политических конфликтов}
  Известны разные классификации политических конфликтов, они строятся по
  различным основаниям.
  Например:
  \begin{itemize}
    \item По субъектам:
      \begin{itemize}
        \item межгосударственный;
        \item межнациональный (этнический);
        \item межклассовый;
        \item между социальными группами и общественными организациями;
        \item между расами (расовый) и др.
      \end{itemize}
    \item По сферам отношений между конфликтующими сторонами:
      \begin{itemize}
        \item военный,
        \item финансовый,
        \item таможенный,
        \item правовой и др.
      \end{itemize}
    \item По масштабности:
      \begin{itemize}
        \item мировой;
        \item региональный;
        \item локальный.
      \end{itemize}
    \item По уровню формирования:
      \begin{itemize}
        \item на межличностном уровне;
        \item на групповом уровне;
        \item на уровне подсистем общества или государства;
        \item на региональном и глобальном уровне.
      \end{itemize}
  \end{itemize}

  Кроме того, конфликты классифицируют: по содержанию, по времени, по форме и
  др.

  \clearpage
\section{Виды политических конфликтов}
  Своеобразие современных политических конфликтов приводит к тому, что
  существенно изменяется их конфигурация и направленность.
  Когда преобладали конфликты классового характера, в которые вовлекались
  широкие народные массы, как это было в социальных революциях, где, по словам
  В.\,И.~Ленина, ,,кризис верхов'', когда верхи не могли управлять по-старому,
  сочетался с ,,кризисом низов'', которые не могли мириться со старыми
  политическими порядками и властными структурами, конфликтное противостояние
  неизбежно приобретало ,,вертикальную'' направленность: ,,низы'' восставали
  против ,,верхов''.

  В современных же условиях, когда политические конфликты в большей своей
  части индивидуализируются, а массы чаще всего остаются в стороне в роли
  безучастных или более-менее заинтересованных наблюдателей, конфликтное
  взаимодействие развертывается преимущественно по ,,горизонтали'', где
  противостоят друг другу различные ветви власти или различные политические
  партии и группировки.

  Сказанное о вертикальной и горизонтальной направленности конфликтных
  противостояний дает основание для вывода о разделении политических
  конфликтов на два основных вида:
  \begin{itemize}
    \item ,,вертикальный'' конфликт между существующей в данной социальной
      системе властью и общественно-политическими силами, интересы которых не
      только не представлены в структуре властных органов, но и игнорируются,
      а в некоторых случаях и подавляются.
    \item ,,горизонтальный'' конфликт внутри самих властных структур по поводу
      объема властных полномочий и их распределения между группировками
      властвующей элиты.
  \end{itemize}

  \clearpage
\section{Стадии политического конфликта}
  С точки зрения динамики конфликта он обычно описывается по следующей схеме:
  \begin{enumerate}
    \item накопление противоречий и формирование отношений сторон;
    \item нарастание и эскалация подготовки;
    \item собственно конфликт;
    \item разрешение конфликта.
  \end{enumerate}

  Первая стадия характеризуется накоплением противоречий, выяснением позиций
  сторон, зондированием потенциальных союзников, скрытым накоплением сил.
  На этой стадии --- в случае выявления --- наибольшая вероятность недопущения
  конфликта.

  Вторая стадия является этапом дифференциации и постепенной поляризации
  сторон.
  Все более обостряются противоречия.
  Стороны перестают воспринимать аргументы друг друга.
  Идет активная эскалация силы, поиск и привлечение на свою сторону союзников,
  нейтрализация возможных союзников соперников.

  Третья стадия --- самый острый, самый драматичный этап --- протекание
  конфликта.
  Это собственно этап политической ,,развязки''.
  На этом этапе вероятно перерастание политического конфликта в военный.

  Четвертая стадия --- завершение конфликта.

  \clearpage
\section{Функции политических конфликтов}
  Особенностью функций конфликтов является то, что проявляются они в
  последствиях, по завершению конфликта.

  Функции-последствия политического конфликта:
  \begin{itemize}
    \item политические;
    \item экономические;
    \item военные;
    \item экологические;
    \item демографические.
  \end{itemize}

  Функции конфликта могут быть позитивными и негативными.

  К позитивным можно отнести:
  \begin{itemize}
    \item функция разрядки напряженности между антагонистами.
      Конфликт играет роль ,,последнего клапана'', ,,отводного канала''
      напряженности.
      Общественная жизнь освобождается от накопившихся страстей;
    \item коммуникативно-информационная и связующая функция.
      В ходе столкновения стороны больше узнают друг друга, могут сближаться
      на какой-либо общей платформе;
    \item стимулирующая функция.
      Конфликт выступает движущей силой социальных изменений;
    \item содействие формированию социально необходимого равновесия.
      Своими внутренними конфликтами общество постоянно ,,сшивается
      воедино'';
    \item функция переоценки и изменения прежних ценностей и норм общества.
  \end{itemize}

  К негативным функциям конфликта можно отнести следующее:
  \begin{itemize}
    \item угроза раскола общества;
    \item неблагоприятные изменения во властных отношениях;
    \item раскол в малоустойчивых социальных группах и международных
      организациях;
    \item неблагоприятные демографические процессы и др.
  \end{itemize}

  Задача управления и разрешения конфликта как раз и состоит в том, чтобы не
  допустить его разрастания или снизить негативные последствия.

  \clearpage
\section{Методы урегулирования политических конфликтов}
  По мнению французского политолога Филиппа Бро, фатальность конфликтов и идея
  консенсуса или, по крайней мере, сплоченности, кажутся неразрывно связанными
  в политической практике как на международной арене, так и внутри страны.
  Власть, не справившаяся с конфликтами, неизбежно терпит поражение.

  Сдерживающими факторами конфликтов в политической области является высокий
  уровень социально-экономического развития и политической культуры общества,
  а также доверие к власти и закону.
  Важно отметить и искусство политического маневрирования как средство
  предупреждения конфликта.
  Разрешение конфликта может идти различными путями.
  При определенных условиях это может произойти как бы самопроизвольно, когда
  предмет конфликта исчезает.

  \textit{Каковы же конкретные методы разрешения конфликтов?}
  \begin{itemize}
    \item \textbf{Метод избегания конфликта}.
      Но избегание конфликта не означает его фактического разрешения, т.\,к.
      само противоречие, лежащее в основе конфликта противоборствующих сторон,
      остается.
    \item \textbf{Отрицание или подмена конфликта}, способ отношения к
      конфликту, когда ему дают возможность тлеть и перемещают в другую
      плоскость.
      Например, политический деятель, ведущий избирательную кампанию, чтобы
      набрать максимум голосов должен избегать четкой позиции по реальным
      противоречиям и руководствоваться такими сценариями, которые
      способствуют более широкому объединению и ведут к согласию между
      политическими субъектами.
    \item \textbf{Метод конфронтации}.
      Обязательно выдвигает на авансцену политики явно неразрешимый антагонизм
      во всей его грубой неприкрытости.
      Но конфронтация же может вести к кризису и крушению политического
      режима.
    \item \textbf{Метод откладывания конфликта}.
      Сдача ,,на милость победителя'' --- действие довольно распространенное в
      практике политической борьбы.
      Следует отметить тот факт, что сторона, сдавшая свои позиции, по мере
      накопления сил и изменения ситуации в ее пользу, как правило, делает
      попытку вернуть утраченное в прошлом.
    \item \textbf{Примирение сторон на основе сближения их позиций и интересов
      через посредника}.
      В роли посредника могут выступать согласительные комиссии, менеджеры по
      конфликтам, отдельные политические деятели или страны (в
      межгосударственных конфликтах).
      Этот метод разрешения конфликтов зафиксирован в Гаагской конвенции 1907
      года.
      На американском континенте действует несколько договоров о мирном
      разрешении споров.
      На африканском континенте действует Протокол Комиссии по посредничеству,
      примирению и арбитражу Организации Африканского Единства.
    \item \textbf{Третейское разбирательство или арбитраж}.
      Считается, что этот метод имеет существенные недостатки, т.\,к. его
      применение может привести лишь к затягиванию конфликта на долгие годы.
    \item \textbf{Переговоры}, которые необходимы для того, чтобы избежать
      применения насилия.
      Они возможны, когда между сторонами имеется хотя бы минимальная сфера
      совпадающих интересов.
  \end{itemize}

  Очевидно, что одним из главных условий согласия (консенсуса) является
  терпимость соперников друг к другу, инакомыслию.

  \clearpage
\section{Заключение}
  Подводя итог краткого анализа концепций конфликта, можно сделать вывод, что
  общество сохраняется как единое целое, благодаря присущим ему внутренним
  конфликтам.
  Именно наличие конфликтов, их сложное множественное переплетение
  препятствует расколу общества на два враждебных лагеря, что может привести к
  гражданской войне.

  Политические кризисы и конфликты дезорганизуют, дестабилизируют обстановку,
  но одновременно и служат началом нового этана развития в случае их
  позитивного разрешения.

  Каждый из типов и видов конфликта, обладая определенными особенностями,
  способен сыграть определенную, конструктивную или деструктивную,
  разрушительную роль в развертывании политических процессов.
  Поэтому важно знать эти особенности, чтобы правильно ориентироваться в
  политической ситуации, как правило, весьма изменчивой, динамичной, и
  занимать продуманную политическую позицию.

  \clearpage
\section{Список используемых источников}
  \begin{enumerate}
    \item Курс лекций по дисциплине ,,Политология'', Афанасьев~В.\,В., 2005;
    \item Пугачев~В.\,П., Соловьев~А.\,И. Введение в политологию: Учебник для
      студентов вузов --- 4-е изд. --- М.: Аспект Пресс, 2005;
    \item Лебедева~М.\,М. Политическое урегулирование конфликтов. --- М.:
      Наука, 1999;
    \item \texttt{http://www.uvauga.ru/HSD\_chair/Political\_science/} ---
      электронное учебное пособие ,,Политология''.
  \end{enumerate}
\end{document}
