\documentclass[a4paper,12pt,notitlepage,pdftex,headsepline]{scrartcl}

\usepackage{a4wide}
\usepackage{cmap} % чтобы работал поиск по PDF
\usepackage[utf8]{inputenc}
\usepackage[ukrainian]{babel}
\usepackage[T2A]{fontenc}

\usepackage{textcase}
\usepackage[pdftex]{graphicx}

\usepackage{lscape}

\pdfcompresslevel=9 % сжимать PDF
\usepackage{pdflscape} % для возможности альбомного размещения некоторых страниц
\usepackage[pdftex]{hyperref}
% настройка ссылок в оглавлении для pdf формата
\hypersetup{unicode=true,
            pdftitle={Теория економіки},
            pdfauthor={Погода Михайло},
            pdfcreator={pdflatex},
            pdfsubject={},
            pdfborder    = {0 0 0},
            bookmarksopen,
            bookmarksnumbered,
            bookmarksopenlevel = 2,
            pdfkeywords={},
            colorlinks=true, % установка цвета ссылок в оглавлении
            citecolor=black,
            filecolor=black,
            linkcolor=black,
            urlcolor=blue}

\usepackage{amsmath}
\usepackage{amssymb}
\usepackage{moreverb}
%for \includepdf
%\usepackage{pdfpages}

\author{Михайло Погода}
\title{Грошові системи. Золото та система золотого стандарту. Валютний курс}
\date{\today}

\begin{document}
\maketitle
\section{Грошова система}

  Грошова система --- це визначена державою форма організації грошового обігу, що історично
  склалася й регулюється законами цієї держави.
  Її основу становить сукупність економічних відносин та інститутів, які забезпечують її
  функціонування.
  Кожна промислово розвинена країна має власну грошову систему, яку розвиває й вдосконалює
  для розвитку національної економіки.

  Кожна з нині діючих грошових систем мають багато спільних ознак та включають такі елементи:
  \begin{enumerate}
    \item \textbf{Грошову одиницю} --- встановлений законодавством грошовий знак, що є
      засобом виміру та вираження цін усіх товарів.
      Як правило, грошова одиниця ділиться на дрібніші частини.
      У переважній більшості країн для цього використовується десятинна система поділу.
    \item \textbf{Види державних грошових знаків} --- види грошових знаків, що мають законну
      платіжну силу, у грошовій системі представляють, в основному, кредитні гроші у вигляді
      банкнот і розмінних монет та паперових грошей у формі державних казначейських
      квитків.
    \item \textbf{Масштаб цін}.
      Масштаб цін колись означав вираження суспільної вартості у грошових одиницях, що
      опирався на фіксовану державою вагову кількість грошового металу у грошовій одиниці.
      Якщо на внутрішньому ринку виникає невідповідність між товарною й грошовою масою, то
      національна валюта, як правило, втрачає офіційно зафіксований державою паритет по
      відношенню до іноземних валют, а деякі країни взагалі відмовилися від встановлення
      офіційного масштабу.
      Масштаб цін в цих умовах визначається як певна кількість товарної маси, що приймається
      за одиницю, а остаточно складається під впливом взаємодії попиту й пропозиції.
      Його функцією стало завдання служити засобом виміру вартостей товарів за допомогою
      цін.
    \item \textbf{Валютний курс} --- співвідношення між грошовими одиницями (валютами)
      різних країн, що визначається, їх купівельною силою.
      Валютний курс характеризується еквівалентною сумою, ціною грошової одиниці однієї
      країни, що виражена у грошових одиницях іншої країни.
      Залежно від типу грошової системи, рівня розвитку ринкових відносин, економічного і
      соціально-політичного стану суспільства можуть застосовуватися:
      \begin{itemize}
        \item фіксовані валютні курси;
        \item плаваючі системи валютних курсів;
        \item системи валютних коридорів.
      \end{itemize}
    \item \textbf{Порядок готівкової й безготівкової емісії та обігу грошових знаків.} Такі
      регулювання держава здійснює за допомогою актів внутрішнього законодавства та
      врахування економічного і валютного становища країни.
    \item \textbf{Регламентацію безготівкового грошового обігу} --- це функція держави і
      НБУ, яка реалізується через:
      \begin{itemize}
        \item встановлення порядку використання грошей, що знаходяться на рахунках банків;
        \item держава визначає сфери, у яких платежі виконуються шляхом безготівкового
          перерахування коштів з одного рахунку на інший;
        \item держава законодавчо визначає способи платежу, форми розрахунків, порядок
          платежу тощо.
      \end{itemize}
    \item \textbf{Правила вивозу й ввезення національної валюти та організації міжнародних
      розрахунків.}
      У цій сфері НБУ виконує такі функції:
      \begin{itemize}
        \item здійснює валютну політику на підставі принципів загальної економічної політики
          України;
        \item складає спільно з Кабінетом Міністрів України платіжний баланс України;
        \item контролює дотримання затвердженого Верховною Радою ліміту зовнішнього
          державного боргу України;
        \item визначає ліміти заборгованості в іноземній валюті уповноважених банків
          нерезидентів;
        \item нагромаджує, зберігає і використовує резерви валютних цінностей для здійснення
          державної валютної політики;
        \item видає ліцензії на здійснення валютних операцій та приймає рішення про їх
          скасування;
        \item визначає способи встановлення і використання валютних (обмінних) курсів
          іноземних валют, виражених в іноземній валюті або у розрахункових (клірингових)
          одиницях.
      \end{itemize}
    \item \textbf{Державний орган, який здійснює грошово-кредитне й валютне регулювання}.
      НБУ за основу своєї діяльності визначає:
      \begin{itemize}
        \item забезпечення стабільності національної грошової одиниці гривні;
        \item розробляє і реалізує грошово-кредитну політику та здійснює контроль за
          повсякденною її реалізацією;
        \item стимулює розвиток і зміцнення банківської системи України;
        \item формує забезпечення ефективного і безперебійного функціонування системи
          розрахунків в інтересах вкладників і кредиторів.
      \end{itemize}
  \end{enumerate}

  Оскільки грошові системи --- це складні економічні системи, що перебувають у стані
  розвитку і змін, то їх слід розглядати з різних боків:
  \begin{itemize}
    \item \textit{Залежно від панівних економічних відносин:}
      \begin{itemize}
        \item \textbf{ринкового типу}, який характеризується вільним функціонуванням грошей,
          грошово-кредитним регулюванням на рівні банківської системи, використання
          переважно економічних важелів підтримання стабільності грошового обігу тощо;
        \item \textbf{неринкова грошова система}, якій властиві адміністративно-командні
          методи і важелі управлінням виробництвом та обміном, а панівним було регулювання
          виробництва і обміну для зближення і витіснення Товар-Гроші-Виробництво і
          грошового обігу;
      \end{itemize}
    \item \textit{Залежно від рівня входження національної економіки у світовій ринок і
                  глибини міжнародного поділу праці:}
      \begin{itemize}
        \item \textbf{грошові системи відкритого типу} --- відсутні обмеження у формуванні
          валютних курсів та обмінних операцій, вільне переміщення грошових ресурсів до
          країни та за її межі, в обігу перебуває вільно конвертована валюта, діють інші
          важелі підтримання національного грошового обігу як інтегрованої частини світового
          господарського і грошового обігу;
        \item \textbf{грошові системи закритого типу}.
          В них переважно панують адміністративно-командні важелі управління суспільним
          виробництвом, відсутня вільна конвертованість національної грошової одиниці на
          іноземні валюти, діють численні обмеження у валютних операціях тощо.
      \end{itemize}
    \item \textit{Залежно від форми грошей у обігу:}
      \begin{itemize}
        \item якщо роль загального еквіваленту виконують благородні метали, то такі системи
          грошового обігу називають \textbf{грошовими системами металевого обігу}.
          У них грошовий товар безпосередньо перебуває в обігу і виконує всі функції грошей,
          а кредитні гроші є безперешкодно розмінюваними на дійсні гроші;
        \item \textbf{система обігу кредитних і паперових грошей}, коли благородні метали з
          обігу вилучено, а в обігу перебувають знаки вартості.
      \end{itemize}
  \end{itemize}

\end{document}



















