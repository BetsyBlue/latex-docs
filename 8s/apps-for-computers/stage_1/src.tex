% vim: spelllang=uk
\documentclass[a4paper,12pt,notitlepage,pdftex]{scrartcl}
\usepackage{pdflscape}
\usepackage[left=2.5cm,right=2.5cm]{geometry}
%\usepackage{cmap} % чтобы работал поиск по PDF
\usepackage[utf8]{inputenc}
\usepackage[english,ukrainian]{babel}
\usepackage[T2A]{fontenc}
\usepackage{indentfirst}
\usepackage{concrete}

%\usepackage{textcase}
\usepackage[pdftex]{graphicx}

\pdfcompresslevel=9 % сжимать PDF
\usepackage{pdflscape} % для возможности альбомного размещения некоторых страниц
\usepackage[pdftex]{hyperref}
% настройка ссылок в оглавлении для pdf формата
\hypersetup{unicode=true
           ,pdftitle={Етап 1}
           ,pdfauthor={Погода Михайло}
           ,pdfcreator={pdflatex}
           ,pdfsubject={}
           ,pdfborder={0 0 0}
           ,bookmarksopen
           ,bookmarksnumbered
           ,bookmarksopenlevel=2
           ,pdfkeywords={}
           ,colorlinks=true % установка цвета ссылок в оглавлении
           ,citecolor=black
           ,filecolor=black
           ,linkcolor=black
           ,urlcolor=blue
           }

\usepackage{amsmath}
\usepackage{amssymb}

\begin{document}
\begin{titlepage}
    \Large
    \begin{center}
        Міністерство освіти і науки, молоді та спорту України

        Національний технічний університет України

        ,,Київський політехнічний інститут''

        \vspace*{1cm}

        Факультет прикладної математики

        \vspace*{3.5cm}

        \textbf{I етап курсової роботи}

        з дисципліни ПЗ ЕОМ

        на тему: ,,Мінімізація витрат матеріалів на побудову огороджень''
    \end{center}

    \vspace*{4cm}

    Виконав:\hfill
        \begin{minipage}{0.3\textwidth}
            Погода М.\,В.

            група КМ-91
        \end{minipage}

    \vspace*{1cm}

    \begin{center}
        \Large
        Київ

        2013
    \end{center}
\end{titlepage}

\tableofcontents

\newpage

\section*{Вступ}
\addcontentsline{toc}{section}{Вступ}

    Назва розробки: ,,Мінімізація витрат матеріалів на побудову заборів''.

    Галузь застосування: завдяки оптимізації форми забору можливо значно
    скоротити витрати матеріалів на його побудову..

\section{Підстави для розробки}
    Підставою для розроблення є навчальний план підготовки бакалавра за
    напрямом 6.040301 ,,Прикладна математика'' та навчальна програма
    дисципліни ,,Програмне забезпечення ЕОМ''.

\section{Призначення розробки}
    Метою є створення програмного забезпечення, за допомогою якого можна
    знайти оптимальну форму забору навколо зазначеної території, таку, що
    витрати матеріалу на її побудову були б мінімальними.

    У функціонал програми входить: введення координат території, що повинна
    бути огородженої; знаходження оптимальної форми огородження; відображення
    моделі на екрані; обчислення довжини отриманого огородження.

\section{Вимоги до програмного виробу}
    \paragraph{Вимоги до функціональних характеристик}
        Програмне забезпечення, що розробляється, повинно виконувати такі
        функції та задовольняти таким вимогам:
        \begin{itemize}
            \item можливість введення користувачем даних;
            \item обчислення оптимальної форми огородження та його довжини;
            \item можливість виведення отриманих результатів на екран.
        \end{itemize}

    \paragraph{Вимоги до умов експлуатації}
        Дане програмне забезпечення буде одного разу під час проектування
        огорожі для території.

    \paragraph{Вимоги до складу й параметрів технічних засобів}
        Вимоги до персонального комп’ютера, на якому буде використовуватись
        розроблене програмне забезпечення:
        \begin{itemize}
            \item Intel Pentium 166 MHz або вище
            \item 128 Мб ОЗУ
            \item 10 Мб вільного місця на диску;
            \item VGA або вища роздільна здатність монітора (для подальшої
                роботи з графікою);
            \item клавіатура, миша.
        \end{itemize}
    \paragraph{Вимоги до інформаційної та програмної сумісності}
        Розроблюване програмне забезпечення повинно працювати під керування
        операційної системи Windows XP/7 або GNU/Linux.

\section{Техно"=економічні показники}
    Програмних засобів, що б були призначені для цієї задачі, не було
    знайдено.

\section{Стадії і етапи розробки}

    \begin{tabular}[t]{|p{1em}|p{27em}|p{5em}|}
        \hline
        \No & Назва роботи & Термін виконання\\
        \hline
        1 & Вибір теми & 20.09.12 \\
        \hline
        2 & Огляд літератури. Вивчення математичних методів розв’язування
        задач & 04.10.12 \\
        \hline
        3 & Вибір, обґрунтування, освоєння методу розв’язування задачі.
        Розв’язування контрольних прикладів & 18.10.12 \\
        \hline
        4 & Проектування архітектури розроблюваних програмних засобів &
        25.10.12 \\
        \hline
        5 & Визначення складу та формату вхідних даних та результатів кожної
        програми & 01.11.12 \\
        \hline
        6 & Розробка алгоритму & 15.11.12 \\
        \hline
        7 & Розробка мови управління програмою & 22.11.12 \\
        \hline
        8 & Програмна реалізація & 06.12.12 \\
        \hline
        9 & Оформлення РГР & 13.12.12 \\
        \hline
        10 & Уточнення технічного завдання & 14.02.13 \\
        \hline
        11 & Налагодження програм та експериментальні розрахунки & 21.02.13 \\
        \hline
        12 & Розв'язок контрольних задач на ЕОМ & 7.03.13 \\
        \hline
        13 & Оформлення пояснювальної записки & 14.03.13 \\
        \hline
        14 & Випробування розроблених програм в присутності викладача &
        21.03.13 \\
        \hline
        15 & Захист курсової роботи перед комісією & 28.03.13 \\
        \hline
    \end{tabular}

\section{Порядок контролю і здачі}
    Вірність результатів розробленого програмного забезпечення перевіряється
    на виконанні контрольних прикладів та задачах, запропонованих комісією.

    Розроблене програмне забезпечення представляється комісії для оцінювання.

\end{document}
