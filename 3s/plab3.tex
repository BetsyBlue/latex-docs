\documentclass[a4paper,10pt]{article}
\usepackage[utf8]{inputenc}
\usepackage[russian]{babel}
\usepackage{indentfirst}
\usepackage{amssymb}
\usepackage{amsmath}
\begin{document}
\begin{enumerate}
\item \textit{В чём заключается смысл метода измерения сопротивления с помощью мостика? Какое плечо моста является плечом сравнения?}

Суть ~метода ~заключается в нахождении неизвестных сопротивлений через соотношения известных, которое вычисляется по правилам Киркгоффа.
Плечом сравнения является плечо $R_1$.
\item \textit{Запишите закон Ома для однородного участка цепи, неоднородного участка, полной цепи.}

Закон Ома для однородного участка цепи:
$$\mathcal{I} = \frac{U}{R}$$
Закон Ома для неоднородного участка цепи:
$$\mathcal{I} = \frac{\varphi_1 - \varphi_2 + \mathcal{E}_{12}}{R}$$
Закон Ома для полной цепи:
$$\mathcal{I} = \frac{\mathcal E}{R}$$
\item \textit{Запишите закон Ома в дифференциальной форме.}
$$\vec{j} = \sigma\vec{E}$$
\item \textit{Сформулируйте правила Киркгоффа.}
\begin{enumerate}
\item Алгебраическая сумма сил токов, которые сходятся в узле, равна 0:
$$\sum\limits_{k=1}^n \mathcal{I}_k = 0$$
\item В контуре алгебраическая сумма падений напряжений (произведений сил токов на сопротивления соответственных участков) равна
алгебраической сумме электродвижущих сил, которые действуют в этом контуре:
$$\sum\limits_{k=1}^N \mathcal{I}_kR_k = \sum\limits_{k=1}^N\mathcal{E}_k$$
\end{enumerate}
\item {Какая формула связи сопротивлений моста при условии его равновесия? Выведите её.}

По второму правилу Киркгоффа для acd и cbd контуров:
\begin{equation}\label{a1}
\begin{cases}
  \mathcal{I}_xR_x+\mathcal{I}_rR_r - \mathcal{I}_3R_3 = 0\\
  \mathcal{I}_1R_1 - \mathcal{I}_2R_2 - \mathcal{I}_rR_r = 0
  \end{cases}
\end{equation}
По первому правилу Киркгоффа для узлов c и d:
\begin{equation}\label{a2}
\begin{cases}
\mathcal{I}_x-\mathcal{I}_r-\mathcal{I}_1 = 0\\
\mathcal{I}_r+\mathcal{I}_3-\mathcal{I}_2 = 0
\end{cases}
\end{equation}
Если $\mathcal{I}_r = 0$, то из (\ref{a1}) и (\ref{a2}):
\begin{equation}\label{a3}
\begin{cases}
\mathcal{I}_xR_x-\mathcal{I}_3R_3 = 0\\
\mathcal{I}_1R_1 - \mathcal{I}_2R_2 = 0
\end{cases}
\end{equation}
\begin{equation}\label{a4}
\begin{cases}
\mathcal{I}_x=\mathcal{I}_1\\
\mathcal{I}_3=\mathcal{I}_2
\end{cases}
\end{equation}
Подставим (\ref{a4}) в (\ref{a3}):
\begin{equation}\label{a5}
\mathcal{I}_1R_x=\mathcal{I}_3R_3
\end{equation}
\begin{equation}\label{a6}
\mathcal{I}_1R_1=\mathcal{I}_3R_2
\end{equation}
Поделим (\ref{a5}) на (\ref{a6}):
$$R_x = \frac{R_1R_3}{R_2}$$
\item \textit{Можна ли использовать мостик Уитстона для измерения маленьких сопротивлений? Проанализируйте и обоснуйте ответ.}

Нет. Это метод не можна использовать для измерения маленьких сопротивлений, т.\,к. при этих условиях через схему будет течь слишком
большой ток, что приведёт к появлению значительныйы погрешностей измерений.
\end{enumerate}

\end{document}
