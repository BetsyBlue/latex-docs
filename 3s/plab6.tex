\documentclass[paper=a4,pagesize,draft,10pt]{scrartcl}
\usepackage{ucs}
\usepackage[utf8x]{inputenc}
\usepackage[russian]{babel}

\begin{document}
\begin{enumerate}
\item \textit{Что такое индукция и напряжённость магнитного поля? Какой их физический смысл?}

Напряжённость магнитного поля $\vec{H}$~--- векторная физическая величина, которая является количественной характеристикой магнитного поля, выражающая силу, с которой поле действует на единицу длины прямолинейного проводника с силой тока в одну единицу, расположенного перпендикулярно к направлению магнитных силовых линий.

Магнитная индукция $\vec{B}$~--- векторная величина, которая является силовой характеристикой магнитного поля в данной точке пространства. Показывает с какой силой магнитное поле действует на двигающийся заряд.
\item \textit{В чём суть процесса намагничивания магнетиков? Дайте определение вектора намагничивания.}

Внутри магнетиков при включении внешнего магнитного поля возникает явление электромагнитной инерции.
ЭДС индукции изменяет движение электронов, в атомах и молекулах появляются дополнительные магнитные моменты.
По правилу Ленца эти моменты направлены так, что вещество намагничиваются против внешнего поля так, что возникает диамагнетизм.

Вектор намагничивания $\vec{j}$ является количественной характеристикой намагничивания магнетика; он представляет собой магнитный момент единицы объёма, который возникает во внешнем магнитном поле $\vec{H}$:
$$\vec{j}=\chi\vec{H}$$
\item \textit{Как формулируется теорема про циркуляцию вектора индукции магнитного поля? Какие свойства имеют магнитные поля?}

Циркуляция вектора $\vec{B}$ по некоторому замкнутому контуру $L$ прямо пропорционально алгебраической сумме сил токов, которые протекают в этом контуре:
$$\oint\limits_L \vec{B}d\vec{l} \sim \sum\limits^n_{i=1}\mathcal{I}_i$$

Магнитные поля имеют такие свойства, как индукция $\vec{B}$ и напряжённость $\vec{H}$.
\item \textit{Какие существуют виды магнетиков? Что такое магнитная восприимчивость и магнитная проницаемость магнетиков?}

Существуют такие виды магнетиков:
\begin{itemize}
\item диамагнетики ($\chi_\mu<0$);
\item парамагнетики ($\chi_\mu>0$);
\item ферромагнетики ($\chi_\mu\gg0$).
\end{itemize}

Магнитная восприимчивость магнетиков~--- безразмерная величина, которая характеризует способность вещества намагничиваться под действием внешнего магнитного поля:
$$\chi_\mu=\frac{\vec{j}}{\vec{H}}$$

Магнитная проницаемость~--- величина, которая характеризует связь между магнитной индукцией $\vec{B}$ и напряжённостью магнитного поля:
$$\vec{B_i}=\mu_{ij}\vec{H_j}$$
\item \textit{Что такое ферромагнетики и какие их свойства?}

Ферромагнетики~--- сильно-магнитные вещества, в которых магнитная проницаемость $\mu\gg1$.

Свойства ферромагнетиков:
\begin{itemize}
\item сильно втягиваются в область более сильного магнитного поля;
\item $\mu\gg1$;
\item при не очень высоких температурах ферромагнетики обладают спонтанной намагниченностью, которая сильно изменятся под воздействием внешних воздействий.
\end{itemize}
\item \textit{Как объясняется природа ферромагнетизма?}

Появление магнитных свойств у ферромагнетиков связано с их доменной структурой.
При воздействии на ферромагнетик внешним магнитным полем, векторы намагниченности доменов начинают поворачиваться за силовыми линиями магнитного поля.
Чем сильнее магнитное поле, тем меньше углы между векторами магнитных моментов доменов и силовыми линиями, что вызывает нелинейное возрастание намагниченности.
\item \textit{В чём состоит явление электромагнитной индукции? Объясните механизм возникновения ЭДС индукции.}
\item \textit{Как пользоваться ключом, шунтирующим гальванометр?}

Ключ нужен для того, чтобы уменьшить колебания рамки гальванометра, вызванные индукционным  током.
Это нужно делать после каждого измерения.
\end{enumerate}

\end{document}
