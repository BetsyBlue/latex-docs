\documentclass[a5paper,10pt,notitlepage,pdftex,headsepline]{scrartcl}

\usepackage{cmap}
\usepackage[utf8]{inputenc}
\usepackage[english,russian]{babel}
\usepackage[T2A]{fontenc}

\usepackage{textcase}
\usepackage[pdftex]{graphicx}

\pdfcompresslevel=9
\usepackage[pdftex]{hyperref}
\hypersetup{unicode=true,
            pdftitle={English Copybook},
            pdfauthor={Michael Pogoda},
            pdfcreator={pdflatex},
            pdfsubject={English},
            pdfborder    = {0 0 0},
            bookmarksopen,
            bookmarksnumbered,
            bookmarksopenlevel = 2,
            pdfkeywords={},
            colorlinks=true,
            citecolor=black,
            filecolor=black,
            linkcolor=black,
            urlcolor=blue}

\author{Michael Pogoda}
\title{English copybook}
\date{\today}
\begin{document}
\section{Unit 11 additional exercises}
  \label{sec:unit_11}
  \subsection{Exercise 3}
    \begin{enumerate}
      \item Presentation layer specifies the language, the compression and
        encryption schemes.
      \item Transport layer identifies each segment's checksum and its
        position in the message.
      \item Network layer contains the sequence of packets and the address of
        the receiving computer.
      \item Session layer marks the beginning and end of the message and
        specifies whether the messages will be sent half-duplex or
        full-duplex.
      \item Data-link layer identifies the sending and receiving computers.
    \end{enumerate}
  \subsection{Exercise 4}
    The message is reconverted into bits by the physical layer.
    The data-link layer confirms the arrival of the packets, copy them in, and
    calculates the checksum for each packet.
    The parts of the message are stored by the data-link layer until the
    message is received.
    The incoming packets are recounted by the network later for secure and
    billing purposes.
    The checksum is recalculated by the network layer which also reassembles
    the message.
    The session layer then sends the message to the next layer.
    The message is decrypted and decompressed by the presentation layer.
    The application layer converts the bits into printable characters, and
    directs data to the correct application.
\section{Unit 12}
  \label{sec:unit_12}
  \subsection{Exercise 7}
    A Usenet newsgroup is a repository usually within the Usenet system, for
    messages posted from many users in different locations.
    The term may be confusing to some, because it is usually a discussion
    group.
    Newsgroups are technically distinct from, but functionally similar to,
    discussion forums on the World Wide Web.
    Newsreader software is used to read newsgroups.

    Newsgroups are much like the public message boards on old bulletin board
    systems.
    For those readers not familiar with this concept, envision an electronic
    version of the corkboard in the entrance of your local grocery store.
    Despite the advent of file-sharing technologies such as BitTorrent, as
    well as the increased use of blogs, formal discussion forums, and social
    networking sites, coupled with a growing number of service providers
    blocking access to Usenet (see main article), newsgroups continue to be
    widely used.
  % section unit_12 (end)
\section{Unit 13}
  \label{sec:unit_13}
  \subsection{Exercise 9}
    After I've entered the address of webpage I'm looking for, browser first
    looking in local file /etc/hosts.
    If the domain is in that file, the corresponding IP address will be used.
    If not, the browser will try to ask DNS server for IP address.
    If this try was successful browser will use all the magic behind TCP/IP
    protocol to contact the web server hosting the site.
    Browser than sends requests to web server, and the server responds with
    well-formed HTML pages.
  % section unit_13 (end)
\section{Unit 14}
  \subsection{Exercise 10}
    I've just visited news.bbc.co.uk.
    Its design isn't bright, it's just simple.
    You can freely navigate through the sport|business|entertainment news.
    By visiting the main page it's easy to find out fresh news.
    Actually, I'm trust news by this company, it has recommended as
    non-speculating news site.
    All news are completed by somehow related graphics|photos.
    Also, there is mobile version of this website.
    In my opinion, news.bbc.co.uk is very good news site.
  \subsection{Specialist reading}
    \subsubsection{A}
      \begin{enumerate}
        \item HTML and XML were derived from SGML.
        \item Markup languages are used to structure and format a document.
        \item XML and SGML are both meta languages.
        \item XML is concerned with identity, meaning and structure of the data.
        \item Extensible makes possible for everyone defining own tags.
        \item The ability to define the content in the document makes XML a more
          intelligent markup language than HTML.
        \item Tag <p> marks the start of a new paragraph.
        \item Search engines are able to do a better job with XML documents because
          of intelligent nature of XML content i.\,e.\  they can identify the type of
          data in document.
        \item An e-commerce website is particularly likely to benefit from XML.
      \end{enumerate}
    \subsubsection{B}
      \paragraph{1}
        \begin{enumerate}
          \item false
          \item true
          \item false
          \item false
          \item true
          \item true
        \end{enumerate}
      \paragraph{2}
        \begin{enumerate}
          \item Metadata is data about data.
          \item Metalanguage is a language from which you can create other
            languages.
          \item HTML is an example of page presentation language.
          \item XML is extensible markup language.
          \item Markup language is a coding system used for structuring and
            formatting documents.
        \end{enumerate}
\section{Unit 15}
  \subsection{Exercise 7}
    \begin{itemize}
      \item gateway
      \item bridge
      \item backbone
      \item router
      \item network
      \item LAN
      \item server
      \item client
      \item thin client
      \item hub
    \end{itemize}
  \subsection{Exercise 8}
    \begin{itemize}
      \item bulletin --- board
      \item domain --- name, button
      \item file --- name
      \item graphical --- name, button
      \item mobile --- phone
      \item search --- engine
      \item site --- map
      \item synchronous --- transmission
      \item text --- message
      \item web --- page
    \end{itemize}
\section{Unit 16}
  \subsection{Exercise 12}
    The GPS was developed by the US military to pinpoint locations anywhere in
    the world.
    It consists of 24 earth-orbiting satellites 17,000 km above the earth.
    Each satellite broadcasts a coded radio signal indicating the time and the
    satellite's exact position.
    The satellites have atomic clocks accurate to one second every 70,000
    years.

    A GPS receiver contains a microprocessor which compares signals from at
    least three satellites to calculate the latitude, longitude and altitude
    of the receiver.
    Apart from military uses, GPS is used for orienting hikers, aiding the
    navigation of ships, tracking trucks and buses, and locating stolen cars.

\end{document}
