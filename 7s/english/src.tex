% vim:spelllang=en,ru
\documentclass[a5paper,10pt,notitlepage,pdftex,headsepline]{scrartcl}

\usepackage{fullpage}
\usepackage{cmap}
\usepackage[utf8]{inputenc}
\usepackage[russian,english]{babel}
\usepackage[T2A]{fontenc}
\usepackage{concrete}

\usepackage{textcase}

\pdfcompresslevel=9
\usepackage[pdftex]{hyperref}
\hypersetup{unicode=true,
            pdftitle={English Copybook},
            pdfauthor={Michael Pogoda},
            pdfcreator={pdflatex},
            pdfsubject={English},
            pdfborder    = {0 0 0},
            bookmarksopen,
            bookmarksnumbered,
            bookmarksopenlevel = 2,
            pdfkeywords={},
            colorlinks=true,
            citecolor=black,
            filecolor=black,
            linkcolor=black,
            urlcolor=blue}

\author{Michael Pogoda}
\title{English copybook}
\date{\today}

\begin{document}
\begin{titlepage}
  \maketitle
\end{titlepage}

\section{Unit 16}
  \subsection{Exercise 2}
    \begin{minipage}{0.5\textwidth}
      \begin{description}
        \item[CU] See you;
        \item[AFAIK] As far as I know;
        \item[TY] Thank you;
        \item[L8R] Later;
        \item[GN] Good night;
      \end{description}
    \end{minipage}
    \begin{minipage}{0.5\textwidth}
      \begin{description}
        \item[AFAIR] As far as I remember;
        \item[inb4] In before;
        \item[tl;dr] too long; don't read;
        \item[BCNU] Be seeing you;
        \item[ATB] All the best;
      \end{description}
    \end{minipage}
  \subsection{Exercise 3}
    \begin{minipage}{0.5\textwidth}
      \begin{description}
        \item[GPRS] General Packer Radio Service.
        \item[HTML] HyperText Markup Language.
        \item[SMS] Short Message Service.
      \end{description}
    \end{minipage}
    \begin{minipage}{0.5\textwidth}
      \begin{description}
        \item[WAP] Wireless Application Protocol.
        \item[WML] Wireless Markup Language
        \item[XML] Extensible Markup Language.
      \end{description}
    \end{minipage}
  \subsection{Exercise 4}
    \begin{enumerate}
      \item Some analysts reckon that WAP phones will overtake PCs as the most
        common way of surfing the Internet, although PCs will still be used for
        more complex applications such as spreadsheet and video players.
      \item WAP technology has been made possible by technological advances in
        ,,bandwidths'', the amount of data that can be received or sent within a
        fraction of second.
      \item You can add frequent address in the memory.
      \item Data is packed into packets.
      \item Some phones come with miniature keyboards.
      \item Because the screen of WAP phone is so small.
    \end{enumerate}
  \subsection{Exercise 9}
      \begin{verse}
        Texting is messing\\
        My head and my english.\\
        Try to write essays\\
        They all come out texts.\\
        Gran is not pleased with letters\\
        She's getting,\\
        Swears I wrote better\\
        Before coming to university.\\
        And she's African.
      \end{verse}
      \begin{verse}
        Fourteen: a text message poem\\
        His eyes are bunsen burner blue,\\
        His hair like iron filings\\
        With Electricity going through.\\
        I sit by him in chemistry\\
        It splits my atoms\\
        When he smiles me.
      \end{verse}
\section{Unit 17}
  \subsection{Exercise A}
    \begin{enumerate}
      \item Data protection \& possibility of quick access.
      \item RAID0 is the fastest of all RAID configurations.
      \item If either drive fails, the other continues to provide
        uninterrupted access to data.
      \item To store data, RAID levels higher than 1 require up to about a
        third more disk space.
      \item On all of the available drives.
      \item Every level except of 0.
      \item RAID0.
    \end{enumerate}
  \subsection{Exercise B}
    \begin{minipage}[t]{0.5\textwidth}
      \paragraph{1}
        \begin{description}
          \item[a] III
          \item[b] V
          \item[c] IV
          \item[d] VI
          \item[e] II
          \item[f] I
        \end{description}
    \end{minipage}
    \begin{minipage}[t]{0.5\textwidth}
      \paragraph{2}
        \begin{description}
          \item[a] True.
          \item[b] False.
          \item[c] False.
          \item[d] True.
          \item[e] False.
          \item[f] True.
        \end{description}
    \end{minipage}
  \subsection{Перевод}
    \begin{otherlanguage}{russian}
      Изготовители серверов объединяют жёсткие диски для обеспечения
      необходимой защиты и быстрого доступа к данным.
      Инженеры называют  эту технологию RAID\footnote{redundant array of
      inexpensive disks "--- избыточный массив независимых жёстких дисков}.
      Объединяя диски, пользователи ожидают повышенной скорости доступа
      жёстких дисков меньшего объёма.
      Специальный контроллер жёстких дисков, называемый RAID-контроллер,
      следит за тем, чтобы компьютер видел массив дисков как один диск.

      Схемы RAID нумеруются в зависимости от использующихся методов для
      обеспечения целостности данных и восстановления их в случае ошибки.
    \end{otherlanguage}
\section{Unit 18}
  \subsection{Exercise A}
    \begin{itemize}
      \item a,
      \item c,
      \item public-key cryptography,
      \item decrypt,
      \item b,
      \item information about the company operating the server and the
        server's public key.
    \end{itemize}
  \subsection{Exercise B}
    \begin{minipage}{0.25\textwidth}
      \begin{description}
        \item[a] IV
        \item[b] III
        \item[c] I
        \item[d] II
      \end{description}
    \end{minipage}
    \begin{minipage}{0.25\textwidth}
      \begin{description}
        \item[a] III
        \item[b] IV
        \item[c] VI
        \item[d] I
        \item[e] II
        \item[f] V
      \end{description}
    \end{minipage}
    \begin{minipage}{0.25\textwidth}
      \begin{description}
        \item[a] False
        \item[b] True
        \item[c] False
        \item[d] False
        \item[e] False
        \item[f] False
        \item[g] True
        \item[h] True
      \end{description}
    \end{minipage}
    \begin{minipage}{0.25\textwidth}
      \begin{enumerate}
        \item c
        \item d
        \item b
        \item a
      \end{enumerate}
    \end{minipage}
  \subsection{Перевод}
    \begin{otherlanguage}{russian}
      Защищённая передача данных через интернет используется в трёх случаях.
      \begin{itemize}
        \item Две стороны, учавствующие в обмене (например, электронная почта
          или деловой заказ), не желают, чтобы третья сторона могла
          прочитать их передачу.
          Некоторая форма шифровки данных позволяет избежать этого.
        \item Получатель сообщения должен иметь возможность определить, не
          было ли изменений данных на пути от отправителя.
        \item Обе стороны должны иметь возожность удостоверитьсяв личности
          друг друга.
      \end{itemize}

      Современные методы шифрования данных основаны на технологии,
      называемой шифрованием с открытым ключём.
      Каждый пользователь такой технологии имеет открытый и закрытые ключи.
      Сообщения шифруются и дешифруются с помощью этих ключей.
      Сообщение, зашифрованное с помощью вашего открытого ключа может быть
      расшифровано только системой, знающей ваш закрытый ключ.

      Чтобы эта система работала обе стороны, учавствующие в защищённом
      обменне данных, должны знать открытые ключи друг друга.
      Закрытые ключи являются хорошо защищёнными ключами, которые известны
      только их владельцам.
      Если я хочу отправить вам зашифрованное сообщение, я использую ваш
      открытый ключ чтобы превратить сообщение в белиберду.
      Известно, что только вы сумеете восстановить оригинальное сообщение из
      этой белиберды, так как только вам известен приватный ключ.
      Шифрование с открытым ключём работает и в другую сторону --- только
      ваш открытый ключ может дешифровать сообщение, зашифрованное вашим
      закрытым ключём.

      Чтобы обеспечить целостность данных, отправитель прогоняет каждое
      сообщение через функцию подсчёта контрольной суммы.
      Эта функция возвращает число, называемое
      MAC\footnote{Message-authentication code}.
      Система работает потому, что это практически невозможно, чтобы у
      двух сообщение был одинаковый MAC.
      Также практически невозможно восстановить сообщение из его MAC'а.

      Программное обеспечение шифрует MAC сообщения приватным ключём.
      Затем, само сообщение и его шифрованный MAC шифруются публичным ключём
      получателя.
      После этого сообщение отправляется.

      Как только получатель получается сообщение и дешифрует его своим
      приватным ключём, получая вместе с самим сообщением ещё и
      зашифрованный MAC, который затем дешифруется публичным ключём
      отправителя.
      Если MAC полученного сообщения совпадает с MAC'ом, переданным в
      послании, значит целостность данных не нарушена.

      Из-за динамики интернета, стала почти обязательной авторизация
      пользователей.
      Это можно осуществить с помощью сертификатов.

      Сервер аутенфицируется клиенту, посылая ему незашифрованный
      сертификат.
      Цифровой сертификат содержит информацию про компанию, ответсвенную за
      сервер, включая её публичный ключ.
      Цифровой сертификат подписывается доверенной компанией, а это означает
      что доверенная компания убедилась, что компания действительно владеет
      сервером.
      Если клиент доверяет компании, подписавшей сертификат, он довереяет и
      компании, чей это сертификат.
      Подпись заключается в шифровании MAC сертификата приватным ключём
      доверенной компании, так что его можно расшифровать публичным ключём
      доверенной компании и сравнить с MAC полученного сертификата.

      Скорее всего, методы шифрования изменятся в будущем, но три принципа,
      которые они обеспечиваются не должны измениться.
      Если вы понимаете базовые принципы, вы уже на три шага впереди других.
    \end{otherlanguage}
\section{Unit 19}
  \subsection{Exercise 5}
    \begin{enumerate}
      \item \textit{Anti-virus program}.
        When an user runs anti-=virus software, the software checks files for
        virus coding.
        If coding is matched to a known virus in a virus database, a message
        is displayed to the user that a virus has been found.
        If the user removes the virus or deletes the infected file, the virus
        cannot spread or cause further damage.
      \item \textit{Face recognition}.
        When you approach a high-=security network, key features of your face
        are scanned.
        When the system matches your features to a database record of
        authorised staff, your identify is verified, so you can log on.
        If your identify is not verified, you cannot use the system.
      \item \textit{Voice recognition}.
        Voice-=activated computers without keyboards will become more common.
        When the user wants to log on, she speaks to the computer which
        matches his voice to a database of voice patterns.
        If the user has a cold or sore throat, he is allowed to use the system
        because stress and intonation patterns remain the same.
    \end{enumerate}
  \subsection{Exercise 6}
    \begin{itemize}
      \item \begin{enumerate}
          \item Person looks through the scanner.
          \item Laser scans the retina.
          \item Computer converts the visual data into barcode.
        \end{enumerate}
      \item \begin{enumerate}
          \item Select the language.
          \item Insert credit card.
          \item Insert hand to be scanned, so the computer could check
            whether handprint matches credit card or not.
        \end{enumerate}
    \end{itemize}
  \subsection{Перевод}
    \begin{otherlanguage}{russian}
      Вторичные и автономные хранилища (которые часто называют
      HSM\footnote{Hierarchical Storage Management}) --- современный способ
      решения задачи хранения данных.
      Жёсткие диски дешевеют, однако требования к хранению данных увеличиваются,
      поэтому лучше планировать HSM, которые позволяют добавлять диски к
      системе.

      По сути, HSM --- это автоматическое перераспределение данных между
      носителями.
      Много изготовителей программного и аппаратного обеспечения имеют свои
      реализации HSM, и все они основываются на одних и тех же базовых
      технологиях.

      Наиболее распространённой конфигурацией HSM является та, в которой
      существуют онлайн"=хранилище (жёсткий диск), вторичное хранилище
      (быстрый носитель, с которого можно практически мгновенно получить
      данные) и оффлайн"=хранилище (медленный, но более дешёвый носитель).
      Такой способ является основным в современных системах.

      В основном, такие системы используют оптические носители для вторичных,
      последовательные носители в качестве оффлайн"=хранилищ.

      Данные автоматически переносятся с онлайн"=хранилища на вторичные
      накопители, если он не использовались на протяжении определённого
      промежутка времени.
      Обычно этот промежуток равен трём месяцам.

      Системы вторичных накопителей обычно реализуются как оптические
      диски, которые могут быть перезаписанными.

      Система должна работать полагаясь на том, что пользователь на должен
      знать, что файл был перемещён во вторичное хранилище.
      Поэтому в структуре директорий остаётся маркер для того, чтобы
      пользователь всё ещё мог видеть файл.
      Когда пользователь попробует открыть его, файл будет скопирован со
      вторичного в онлайн"=хранилище, где будет открыт пользователем.
      Всё, что заметит пользователь --- небольшая задержка при открытии файла.

      Перемещение данных со вторичного на автономное хранилище может быть
      осуществлено похожим способом, однако чаще маркер, который остаётся в
      директории просто содержит ссылку.
      Это даёт пользователь возможность сделать запрос на файл управляющей
      программе, и получить информацию типа ,,этот файл был заархивирован на
      автономное хранилище'' и ссылку на место хранения файла.
      Потом эта ссылка посылается управляющей программе, и файлы могут быть
      восстановлены (с носителя) обычным способом.

      Некоторые современные системы могут хранить несколько носителей в
      хранилище, что позволяет оперировать ими как в ,,музыкальном автомате''.
  \end{otherlanguage}
\section{Unit 20}
  \subsection{Exercise 6}
    \begin{enumerate}
      \item find out
      \item hand over
      \item tracked down
      \item break into
      \item log on
      \item go about
      \item phone up
      \item throw away
      \item grown up
      \item hacking into
      \item keep ahead
    \end{enumerate}
  \subsection{Exercise 7}
    \begin{enumerate}
      \item throw away
      \item hack (break) into
      \item grown up
      \item set about
      \item calls you, hand it over
      \item shut down
      \item ran up
      \item find out
      \item write down
      \item check out
    \end{enumerate}
\section{Unit 21}
  \subsection{Exercise 10}
    \begin{itemize}
      \item \textit{Parallel implementation} means that both systems will run
        for a determined period of time, after which the old system will be
        disabled.
        The advantage is that if we have some troubles with new system --- we can
        roll back to the old one.
        Disadvantage is that both systems should be run maintained and
        synchronised with each other.
      \item \textit{Phased implementation}: gradually the old system is being
        upgraded to new by parts, so users use the same system with new
        features, while the haven't completely replaced the old system.
        Advantage is that it is easier for users to get used to the new
        system.
        Disadvantage is complexity of such systems and that some changed
        cannot be made to the system.
    \end{itemize}
\section{Unit 21}
  \subsection{Перевод}
    \begin{otherlanguage}{russian}
      Одной из основных мотиваций использования ООП является работа с
      мультимедийными приложениями, в которых разнообразные типы данных, такие
      как звук и видео, могут быть запакованными вместе в исполняемый модуль.
      Другая причина --- написание программного кода, который был бы более
      интуитивно понятным и многоразовым.

      Возможно, ключевой особенностью ООП является инкапсуляция --- объединение
      данных и команд программы в модули, которые называются ,,объектами''.

      Меню также является объектом, как варианты выбора в нём.
      Каждый раз пользователь выбирает объект , инструкции внутри объекта
      запускаются со свойствами или данными, которые содержит объект, чтобы
      перейти к следующему шагу.

      Другой ключевой особенностью ООП является наследование.
      Оно позволяет ООП-=разработчикам определять класс объектов, например
      ,,четырёхугольники'', и их более конкретный случай ,,квадраты'' (какк
      ,,четырёхугольник'' с одинаковыми сторонами.
      Таким образом, все свойства ,,четырёхугольника'' (имеет четыре стороны и
      четыре угла) автоматически наследуются ,,квадратами''.

    \end{otherlanguage}
\section{Unit 22}
  \subsection{Exercise A}
    \begin{enumerate}
      \item To improve your marketability to poptential employers by upgrading
        your skill-set.
      \item Whose training should you undertake?
        Whose certificates should you get?
      \item Microsoft Word
      \item Attending a training course, self-study
      \item The amount of work you'll have to do to get up to speed for the
        exams and the difference between passing or failing the exam.
      \item Because you need to get used to answering the requisite number of
        questions withing the allowed time to fit into it.
    \end{enumerate}
  \subsection{Exercise B}
    \begin{enumerate}
      \item \begin{enumerate}
          \item MCPS
          \item MCSE
          \item MCT
          \item MCSD
        \end{enumerate}
      \item \begin{enumerate}
          \item true
          \item false
          \item false
          \item true
          \item false
          \item true
          \item true
        \end{enumerate}
    \end{enumerate}
  \subsection{Перевод}
    \begin{otherlanguage}{russian}
      Представим, что вы --- инженер технической поддержки.
      Вы застряли на работе, которая вам не нравится и желаете перемен.
      Один из способов что-либо изменить --- улучшить ваши навыки для того,
      чтобы повысить уровень вашего соответствия нуждам рынка работы.

      Если вы отважились на это, то какой курс выбрать?
      Если вам необходимы сертификаты, то именно ли эти?
      И даже если вы их получите, насколько вы можете быть уверены, что ваша
      зарплата возрастёт?
      Одно из решений --- набор сертификатов, предложенных Microsoft.

      Microsoft предлагает широкий набор сертификационных программ, нацеленных
      на всех, начиная от пользователя конкретной программы (например,
      Microsoft Word) и заканчивая тем, кто хочет стать инженером технической
      поддержки/сотрудником службы технической поддержки.
      Также существует множество других сертификатов.
      Если вы обладатель любого такого сертификата, то вы можете с гордостью
      называть себя Microsoft-=сертифицированным специалистом.

      Выбрав своё направление, вы должны определиться насколько
      квалифицированным вы являетесь основываясь на ваших знаниях и опыте.
      Необходимо ли вам посещать курсы или достаточно самостоятельного
      обучения?
      Сколько времени вы можете выделить для этого?
      Будет ли платить ваш работодатель за эти курсы?

      Ключевым вопросом является ваш опыт.
      Он влияет не только на объём работы, которую необходимо проделать во
      время подготовки к экзаменам, но может и определить, успешно ли вы
      сдадите тесты.

      Пока вы заняты изучением всего необходимого для сертификации, пробные
      экзамены являются хорошей находкой.
      Они демонстрируют типы вопросов, с которыми вы столкнётесь на экзамене,
      ознакомят вас со структурой экзамена.
      Основным фактом, на который вам необходимо обратить внимание, является
      тот факт, что на экзамене время ограничено, поэтому вам необходимо
      привыкнуть отвечать на определённое количество вопросов за выделенное
      время.

      Если вы решите, что вам могут помочь учебные курсы, не выбирайте курс
      только по его названию.
      Узнайте, что именно предлагает этот курс, и существуют ли какие-либо
      условия для поступающих.
      Также узнайте, что делает компания, проводящая курс, если абитуриент не
      имеет минимально-=необходимых знаний для этого курса.

      Когда экзамены заменяются новыми, вам необходимо обновить свой
      сертификат.
      В конце концов, за то, чтобы ваш сертификат отвечал современному
      стандарту, отвечаете только вы.
      Иначе вы теряет сертификацию пока не сдадите новый экзамен.

      Как рабочий службы поддержки, вы получите удовлетворённость собой, зная
      тот факт, что вы прошли сложный тест, и знание того, что ваш менеджер по
      персоналу боится того факта, что вас могут перевербовать в любой момент.

      Существуют следующие квалификационные направления:
      \begin{itemize}
        \item Сертифицированные системный инженер Microsoft (MCSE).
          Эти специалисты проектируют, устанавливают, поддерживают та
          диагностируют неполадки в информационных системах.
        \item Сертифицированный специалист по продуктам Microsoft (MCPS).
          Эти специалисты знают всё хотя бы про одну операционную систему от
          Microsoft.
          Другие специализируются на других продуктах Microsoft.
      \end{itemize}
    \end{otherlanguage}
\section{Unit 23}
  \subsection{Exercise A}
    \begin{enumerate}
      \item Connecting interfaces compatibility problem.
      \item Ericsson, IBM, Intel, Nokia and Toshiba.
      \item Mobility, low power usage
      \item Operating requences
      \item PDA, Tablet PC's, Laptops, Phones
      \item  40-bit encryption and frequency hopping.
      \item Because it would increase its power consumption and cost.
    \end{enumerate}
  \subsection{Exercise B}
    \begin{enumerate}
      \item \begin{enumerate}
          \item III
          \item V
          \item I
          \item VI
          \item VII
          \item II
          \item IV
        \end{enumerate}
      \item \begin{enumerate}
          \item false
          \item false
          \item true
          \item true
          \item false
          \item false
          \item true
        \end{enumerate}
    \end{enumerate}
\section{Unit 24}
  \subsection{Exercise A}
    \begin{enumerate}
      \item That anything is possible if you really put your mind to it.
      \item All the details of his passport, bank account, medical records and
        driving license.
      \item Shopping.
      \item Software programs, networks, telephones \& machines with a degree
        of intelligence built in.
      \item Increasing the speed of evolution.
      \item The deluge of information.
    \end{enumerate}
  \subsection{Перевод}
    \begin{otherlanguage}{russian}
      Пообщаться с проффесором Кокрэйном --- это, возможно, лучший способ
      путешествовать во времени, не покидая при этом современной пространство,
      так как его видение будущего растянуто далеко в XXI век.
      Он твёрдо уверен, что возможно всё, если к этому подойти с умом.

      Изготовленные для XXI века кольцо с печаткой Питера Кокрена создано на
      основании чипа, который содержит всю информацию с его паспорта,
      банковского счёта, медических записей и водительских прав.
      По словам самого Кокрена, это должно будет произвести революции в
      будущем.

      Однако и это ещё не всё.
      BT сделали шаг вперёд в развитии искусственного интеллекта и создали
      машины, которые решают собственные проблемы.
      ,,Мы создали решение, про которое человек не мог и мечтать.
      Мы имеем решение, однако, в то же время, мы не понимаем, как оно
      работает, но оно работает.
      Мы эффективно увеличиваем скорость эволюции'', говорит Кокрен.
    \end{otherlanguage}
\section{Text 1}
  \begin{otherlanguage}{russian}
    Конкурент \emph{Dropbox} заявляет, что можно обмениваться данными в десять раз
    быстрее, чем раньше, благодаря разработанному ими методу понижения
    нагрузки на сервера.

    \emph{Box} --- компания облачного хранения данных, которая обслуживает
    такие компании, как \emph{LinkedIn}, \emph{McAfee}, \emph{DirectTV}, ---
    запустила сервис, который должен обеспечить скорость обмена данными с
    облаком в десять раз более быструю.

    Сервис Box позволяет бизнес-клиентам и индивидуальным клиентам хранить их
    файлы в облаке, аналогично конкурентам \emph{Dropbox} и
    \emph{GoogleDrive}.
    Но Уитни Боук, генеральный директор компании Box, сказала, что новые
    возможности Box опережает остальные облачные сервиса.

    ,,Мы владеем, фактически, самым быстрым сервисом на рынке'', сказала она.

    Новая функция, \emph{Box Accelerator}, --- ими разработанный метод
    снижения нагрузки на сервера, и он применяется во всемирной сети Box.
    ,,Новый сервис, который будет бесплатным, определяет, который из серверов
    сети передаёт данные быстрее других'', сказала Боук.
    Например, если пользователь находится в Париже и сервер в Париже
    перегружен, Box перенаправляет информацию в Германию для быстрого
    получения результатов.

    Боук сказала, что тесты показали повышения скорости загрузки как минимум
    вдвое, и максимум в десять раз.
    По словам Боук, Box создали этот сервис в основном для компаний, которые
    предоставляют услуги, но и индивидуальные клиенты также получают
    достоинства новой технологии, так как она применяется для всех аккаунтов
    Box.
  \end{otherlanguage}
\end{document}
