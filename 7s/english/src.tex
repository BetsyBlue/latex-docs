\documentclass[a5paper,10pt,notitlepage,pdftex,headsepline]{scrartcl}

\usepackage{fullpage}
\usepackage{cmap}
\usepackage[utf8]{inputenc}
\usepackage[russian,english]{babel}
\usepackage[T2A]{fontenc}
\usepackage{concrete}

\usepackage{textcase}

\pdfcompresslevel=9
\usepackage[pdftex]{hyperref}
\hypersetup{unicode=true,
            pdftitle={English Copybook},
            pdfauthor={Michael Pogoda},
            pdfcreator={pdflatex},
            pdfsubject={English},
            pdfborder    = {0 0 0},
            bookmarksopen,
            bookmarksnumbered,
            bookmarksopenlevel = 2,
            pdfkeywords={},
            colorlinks=true,
            citecolor=black,
            filecolor=black,
            linkcolor=black,
            urlcolor=blue}

\author{Michael Pogoda}
\title{English copybook}
\date{\today}

\makeatletter
\newcommand{\rmnum}[1]{\romannumeral #1}
\newcommand{\Rmnum}[1]{\expandafter\@slowromancap\romannumeral #1@}
\makeatother

\begin{document}
\begin{titlepage}
  \maketitle
\end{titlepage}

\section{Unit 16}
  \subsection{Exercise 2}
    \begin{minipage}{0.5\textwidth}
      \begin{description}
        \item[CU] See you;
        \item[AFAIK] As far as I know;
        \item[TY] Thank you;
        \item[L8R] Later;
        \item[GN] Good night;
      \end{description}
    \end{minipage}
    \begin{minipage}{0.5\textwidth}
      \begin{description}
        \item[AFAIR] As far as I remember;
        \item[inb4] In before;
        \item[tl;dr] too long; don't read;
        \item[BCNU] Be seeing you;
        \item[ATB] All the best;
      \end{description}
    \end{minipage}
  \subsection{Exercise 3}
    \begin{minipage}{0.5\textwidth}
      \begin{description}
        \item[GPRS] General Packer Radio Service.
        \item[HTML] HyperText Markup Language.
        \item[SMS] Short Message Service.
      \end{description}
    \end{minipage}
    \begin{minipage}{0.5\textwidth}
      \begin{description}
        \item[WAP] Wireless Application Protocol.
        \item[WML] Wireless Markup Language
        \item[XML] Extensible Markup Language.
      \end{description}
    \end{minipage}
  \subsection{Exercise 4}
    \begin{enumerate}
      \item Some analysts reckon that WAP phones will overtake PCs as the most
        common way of surfing the Internet, although PCs will still be used for
        more complex applications such as spreadsheet and video players.
      \item WAP technology has been made possible by technological advances in
        ,,bandwidths'', the amount of data that can be received or sent within a
        fraction of second.
      \item You can add frequent address in the memory.
      \item Data is packed into packets.
      \item Some phones come with miniature keyboards.
      \item Because the screen of WAP phone is so small.
    \end{enumerate}
\section{Unit 17}
  \subsection{Exercise A}
    \begin{enumerate}
      \item Data protection \& possibility of quick access.
      \item RAID0 is the fastest of all RAID configurations.
      \item If either drive fails, the other continues to provide
        uninterrupted access to data.
      \item To store data, RAID levels higher than 1 require up to about a
        third more disk space.
      \item On all of the available drives.
      \item Every level except of 0.
      \item RAID0.
    \end{enumerate}
  \subsection{Exercise B}
    \begin{minipage}[t]{0.5\textwidth}
      \paragraph{1}
        \begin{description}
          \item[a] \Rmnum{3}
          \item[b] \Rmnum{5}
          \item[c] \Rmnum{4}
          \item[d] \Rmnum{6}
          \item[e] \Rmnum{2}
          \item[f] \Rmnum{1}
        \end{description}
    \end{minipage}
    \begin{minipage}[t]{0.5\textwidth}
      \paragraph{2}
        \begin{description}
          \item[a] True.
          \item[b] False.
          \item[c] False.
          \item[d] True.
          \item[e] False.
          \item[f] True.
        \end{description}
    \end{minipage}
  \subsection{Перевод}
    \begin{otherlanguage}{russian}
      Изготовители серверов объединяют жёсткие диски для обеспечения
      необходимой защиты и быстрого доступа к данным.
      Инженеры называют  эту технологию RAID\footnote{redundant array of
      inexpensive disks "--- избыточный массив независимых жёстких дисков}.
      Объединяя диски, пользователи ожидают повышенной скорости доступа
      жёстких дисков меньшего объёма.
      Специальный контроллер жёстких дисков, называемый RAID-контроллер,
      следит за тем, чтобы компьютер видел массив дисков как один диск.

      Схемы RAID нумеруются в зависимости от использующихся методов для
      обеспечения целостности данных и восстановления их в случае ошибки.
    \end{otherlanguage}
\section{Unit 18}
  \subsection{Exercise A}
    \begin{itemize}
      \item a,
      \item c,
      \item public-key cryptography,
      \item decrypt,
      \item b,
      \item information about the company operating the server and the
        server's public key.
    \end{itemize}
  \subsection{Exercise B}
    \begin{minipage}{0.25\textwidth}
      \begin{description}
        \item[a] \Rmnum{4}
        \item[b] \Rmnum{3}
        \item[c] \Rmnum{1}
        \item[d] \Rmnum{2}
      \end{description}
    \end{minipage}
    \begin{minipage}{0.25\textwidth}
      \begin{description}
        \item[a] \Rmnum{3}
        \item[b] \Rmnum{4}
        \item[c] \Rmnum{6}
        \item[d] \Rmnum{1}
        \item[e] \Rmnum{2}
        \item[f] \Rmnum{5}
      \end{description}
    \end{minipage}
    \begin{minipage}{0.25\textwidth}
      \begin{description}
        \item[a] False
        \item[b] True
        \item[c] False
        \item[d] False
        \item[e] False
        \item[f] False
        \item[g] True
        \item[h] True
      \end{description}
    \end{minipage}
    \begin{minipage}{0.25\textwidth}
      \begin{enumerate}
        \item c
        \item d
        \item b
        \item a
      \end{enumerate}
    \end{minipage}
  \subsection{Перевод}
    \begin{otherlanguage}{russian}
      Защищённая передача данных через интернет используется в трёх случаях.
      \begin{itemize}
        \item Две стороны, учавствующие в обмене (например, электронная почта
          или деловой заказ), не желают, чтобы третья сторона могла
          прочитать их передачу.
          Некоторая форма шифровки данных позволяет избежать этого.
        \item Получатель сообщения должен иметь возможность определить, не
          было ли изменений данных на пути от отправителя.
        \item Обе стороны должны иметь возожность удостоверитьсяв личности
          друг друга.
      \end{itemize}

      Современные методы шифрования данных основаны на технологии,
      называемой шифрованием с открытым ключём.
      Каждый пользователь такой технологии имеет открытый и закрытые ключи.
      Сообщения шифруются и дешифруются с помощью этих ключей.
      Сообщение, зашифрованное с помощью вашего открытого ключа может быть
      расшифровано только системой, знающей ваш закрытый ключ.

      Чтобы эта система работала обе стороны, учавствующие в защищённом
      обменне данных, должны знать открытые ключи друг друга.
      Закрытые ключи являются хорошо защищёнными ключами, которые известны
      только их владельцам.
      Если я хочу отправить вам зашифрованное сообщение, я использую ваш
      открытый ключ чтобы превратить сообщение в белиберду.
      Известно, что только вы сумеете восстановить оригинальное сообщение из
      этой белиберды, так как только вам известен приватный ключ.
      Шифрование с открытым ключём работает и в другую сторону --- только
      ваш открытый ключ может дешифровать сообщение, зашифрованное вашим
      закрытым ключём.

      Чтобы обеспечить целостность данных, отправитель прогоняет каждое
      сообщение через функцию подсчёта контрольной суммы.
      Эта функция возвращает число, называемое
      MAC\footnote{Message-authentication code}.
      Система работает потому, что это практически невозможно, чтобы у
      двух сообщение был одинаковый MAC.
      Также практически невозможно восстановить сообщение из его MAC'а.

      Программное обеспечение шифрует MAC сообщения приватным ключём.
      Затем, само сообщение и его шифрованный MAC шифруются публичным ключём
      получателя.
      После этого сообщение отправляется.

      Как только получатель получается сообщение и дешифрует его своим
      приватным ключём, получая вместе с самим сообщением ещё и
      зашифрованный MAC, который затем дешифруется публичным ключём
      отправителя.
      Если MAC полученного сообщения совпадает с MAC'ом, переданным в
      послании, значит целостность данных не нарушена.

      Из-за динамики интернета, стала почти обязательной авторизация
      пользователей.
      Это можно осуществить с помощью сертификатов.

      Сервер аутенфицируется клиенту, посылая ему незашифрованный
      сертификат.
      Цифровой сертификат содержит информацию про компанию, ответсвенную за
      сервер, включая её публичный ключ.
      Цифровой сертификат подписывается доверенной компанией, а это означает
      что доверенная компания убедилась, что компания действительно владеет
      сервером.
      Если клиент доверяет компании, подписавшей сертификат, он довереяет и
      компании, чей это сертификат.
      Подпись заключается в шифровании MAC сертификата приватным ключём
      доверенной компании, так что его можно расшифровать публичным ключём
      доверенной компании и сравнить с MAC полученного сертификата.

      Скорее всего, методы шифрования изменятся в будущем, но три принципа,
      которые они обеспечиваются не должны измениться.
      Если вы понимаете базовые принципы, вы уже на три шага впереди других.
    \end{otherlanguage}
\end{document}
