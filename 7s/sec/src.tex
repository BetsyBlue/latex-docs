% vim:spelllang=uk
\documentclass[a4paper,10pt,notitlepage,pdftex,headsepline]{scrartcl}

\usepackage{a4wide}
\usepackage{cmap} % чтобы работал поиск по PDF
\usepackage[utf8]{inputenc}
\usepackage[ukrainian]{babel}
\usepackage[T2A]{fontenc}
\usepackage{concrete}

\pdfcompresslevel=9 % сжимать PDF
\usepackage{pdflscape} % для возможности альбомного размещения некоторых страниц
\usepackage[pdftex]{hyperref}
% настройка ссылок в оглавлении для pdf формата
\hypersetup{unicode=true,
            pdftitle={Основи охорони праці},
            pdfauthor={Михайло Погода},
            pdfcreator={pdflatex},
            pdfsubject={},
            pdfborder    = {0 0 0},
            bookmarksopen,
            bookmarksnumbered,
            bookmarksopenlevel = 2,
            pdfkeywords={},
            colorlinks=true, % установка цвета ссылок в оглавлении
            citecolor=black,
            filecolor=black,
            linkcolor=black,
            urlcolor=blue}

\author{Михайло Погода}
\title{Основи охорони праці}
\date{\today}

\begin{document}
\begin{titlepage}
  \maketitle
\end{titlepage}

\tableofcontents
\newpage

\section{Загальні питання охорони праці}

  Шкідливий виробничий фактор "--- небажане явище, яке супроводжує виробничий
  процес і вплив якого на працюючого може привести до погіршення самопочуття,
  зниження працездатності, захворювання та навіть смерті як результату
  захворювання.

  Небезпечний виробничий фактор "--- небажане явище, яке супроводжує
  виробничий процес і дія якого за певних умов може привести до травми або
  іншого раптового погіршення здоров’я працівника й навіть до раптової смерті.

  За походженням ці фактори можна поділити на:
  \begin{itemize}
    \item Фізичні.
    \item Хімічні.
      Їх поділяють на:
      \begin{itemize}
        \item токсичні,
        \item задушуючі,
        \item наркотичні,
        \item подразнюючи,
        \item сенсибілізуючи,
        \item канцерогенні,
        \item мутагенні,
        \item такі, що впливають на репродуктивну функцію.
      \end{itemize}

      По шляхам проникнення їх поділяють на:
      \begin{itemize}
        \item органи дихання,
        \item шлунково-кишечний тракт,
        \item шкіряні покриви та слизові оболонки.
      \end{itemize}
    \item Біологічні "--- патогенні мікроорганізми та продукти їх
      життєдіяльності, а також макроорганізми.
    \item Психо-фізіологічні.
      Їх поділяють на:
      \begin{itemize}
        \item фізичні перенавантаження,
        \item нервово-психічні перенавантаження,
        \item емоційні перенавантаження,
        \item монотонні перенавантаження.
      \end{itemize}
    \item Cоціальні:
      \begin{itemize}
        \item неякісна організація роботи,
        \item понаднормовона работа,
        \item погані відносини в колективі,
        \item зміни біоритмів,
        \item насильство.
      \end{itemize}
  \end{itemize}

  Охорона праці --- система правових, соціально-економічних,
  організаційно-технічних, санітарно-гігієнічних і лікувально-профілактичних
  заходів і засобів спрямованих на збереження життя, здоров’я та
  працездатності людини в процесі трудової діяльності.

  Структура до модулю:
  \begin{itemize}
    \item правові та організаційні основи,
    \item фізіологія, гігієна праці та виробнича санітарія,
    \item виробнича безпека,
    \item пожежна безпека на виробництві.
  \end{itemize}

\section{Правові та організаційні основи охорони праці}
  \begin{itemize}
    \item Закон України про охорону праці.
      Був введений в 1992 року.
      Має 8 розділів та 48 статей.
      \begin{itemize}
        \item Гарантії прав на охорону праці (розділ 2, статті 5--12).
        \item Організація охорони праці (розділ 3, статті 13--24).
        \item Стимулювання охорони праці (розділ 4, статті 25--26).
        \item Нормативно-правові акти з охорони праці (розділ 5, статті
          27--30).
        \item Державне управління охороною праці (розділ 6, статті 31--37).
        \item Державний нагляд і громадський контроль за охороною праці
          (розділ 7, статті 38--42).
        \item Відповідальність працівників за порушення законодавства про
          охорону праці (розділ 8, статті 43--44).
        \item Система стандартів безпеки праці.
      \end{itemize}
    \item Конституція України.
      \begin{description}
        \item[43] Право на належні безпечні умови праці.
        \item[46] Соціальних захист, що включає право забезпечення у разі
          повної, часткової або тимчасової втрати працездатності.
        \item[49] Право на охорону здоров’я, медичну допомогу та медичне
          страхування.
        \item[57] Право знати свої права.
      \end{description}
    \item Кодекс законів про працю.
    \item Закон України про пожежну безпеку.
    \item Закон о радіоційній безпеки України.
    \item Закон України про загально-обов’язкове державне соціальне страхування
      від нещасного випадку на виробництві та професійного захворювання, які
      спричинили втрату працездатності.
  \end{itemize}
\end{document}
