% vim:spelllang=uk
\documentclass[a4paper,10pt,notitlepage,pdftex,headsepline]{scrartcl}

\usepackage{cmap} % чтобы работал поиск по PDF
\usepackage[utf8]{inputenc}
\usepackage[ukrainian]{babel}
\usepackage[T2A]{fontenc}
\usepackage{concrete}
\usepackage{fullpage}

\pdfcompresslevel=9 % сжимать PDF
\usepackage{pdflscape} % для возможности альбомного размещения некоторых страниц
\usepackage[pdftex]{hyperref}
% настройка ссылок в оглавлении для pdf формата
\hypersetup{unicode=true,
            pdftitle={Основи охорони праці},
            pdfauthor={Михайло Погода},
            pdfcreator={pdflatex},
            pdfsubject={},
            pdfborder    = {0 0 0},
            bookmarksopen,
            bookmarksnumbered,
            bookmarksopenlevel = 2,
            pdfkeywords={},
            colorlinks=true, % установка цвета ссылок в оглавлении
            citecolor=black,
            filecolor=black,
            linkcolor=black,
            urlcolor=blue}

\author{Михайло Погода}
\title{Основи охорони праці}
\date{\today}

\begin{document}
\begin{titlepage}
  \maketitle
\end{titlepage}

\tableofcontents
\newpage

\section{Загальні питання охорони праці}
  \emph{Шкідливий виробничий фактор} "--- небажане явище, яке супроводжує
  виробничий процес і вплив якого на працюючого може привести до погіршення
  самопочуття, зниження працездатності, захворювання та навіть смерті як
  результату захворювання.

  \emph{Небезпечний виробничий фактор} "--- небажане явище, яке супроводжує
  виробничий процес і дія якого за певних умов може привести до травми або
  іншого раптового погіршення здоров’я працівника й навіть до раптової смерті.

  За походженням ці фактори можна поділити на:
  \begin{itemize}
    \item Фізичні.
    \item Хімічні.
      Їх поділяють на:
      \begin{itemize}
        \item токсичні,
        \item задушуючі,
        \item наркотичні,
        \item подразнюючи,
        \item сенсибілізуючи,
        \item канцерогенні,
        \item мутагенні,
        \item такі, що впливають на репродуктивну функцію.
      \end{itemize}

      По шляхам проникнення їх поділяють на:
      \begin{itemize}
        \item органи дихання,
        \item шлунково"=кишечний тракт,
        \item шкіряні покриви та слизові оболонки.
      \end{itemize}
    \item Біологічні "--- патогенні мікроорганізми та продукти їх
      життєдіяльності, а також макроорганізми.
    \item Психо"=фізіологічні.
      Їх поділяють на:
      \begin{itemize}
        \item фізичні перенавантаження,
        \item нервово"=психічні перенавантаження,
        \item емоційні перенавантаження,
        \item монотонні перенавантаження.
      \end{itemize}
    \item Cоціальні:
      \begin{itemize}
        \item неякісна організація роботи,
        \item понаднормовона работа,
        \item погані відносини в колективі,
        \item зміни біоритмів,
        \item насильство.
      \end{itemize}
  \end{itemize}

  \emph{Охорона праці} --- система правових, соціально"=економічних,
  організаційно"=технічних, санітарно"=гігієнічних і лікувально"=профілактичних
  заходів і засобів спрямованих на збереження життя, здоров’я та
  працездатності людини в процесі трудової діяльності.

  Структура до модулю:
  \begin{itemize}
    \item правові та організаційні основи,
    \item фізіологія, гігієна праці та виробнича санітарія,
    \item виробнича безпека,
    \item пожежна безпека на виробництві.
  \end{itemize}
\section{Правові та організаційні основи охорони праці}
  \begin{itemize}
    \item Закон України про охорону праці.
      Був введений в 1992 року.
      Має 8 розділів та 48 статей.
      \begin{itemize}
        \item Гарантії прав на охорону праці (розділ 2, статті 5--12).
        \item Організація охорони праці (розділ 3, статті 13--24).
        \item Стимулювання охорони праці (розділ 4, статті 25--26).
        \item Нормативно"=правові акти з охорони праці (розділ 5, статті
          27--30).
        \item Державне управління охороною праці (розділ 6, статті 31--37).
        \item Державний нагляд і громадський контроль за охороною праці
          (розділ 7, статті 38--42).
        \item Відповідальність працівників за порушення законодавства про
          охорону праці (розділ 8, статті 43--44).
        \item Система стандартів безпеки праці.
      \end{itemize}
    \item Конституція України.
      \begin{description}
        \item[43] Право на належні безпечні умови праці.
        \item[46] Соціальних захист, що включає право забезпечення у разі
          повної, часткової або тимчасової втрати працездатності.
        \item[49] Право на охорону здоров’я, медичну допомогу та медичне
          страхування.
        \item[57] Право знати свої права.
      \end{description}
    \item Кодекс законів про працю.
    \item Закон України про пожежну безпеку.
    \item Закон о радіаційній безпеки України.
    \item Закон України про загально"=обов’язкове державне соціальне страхування
      від нещасного випадку на виробництві та професійного захворювання, які
      спричинили втрату працездатності.
  \end{itemize}

  Закон про охорону праці за порушення норм охорони праці передбачає:
  \begin{itemize}
    \item \emph{Адміністративну відповідальність} --- відповідно статті 41
      Кодексу України тягне за собою адміністративну відповідальність у
      вигляді накладання штрафів на працівників та посадових осіб
      підприємства.
    \item \emph{Дисциплінарна відповідальність} --- відповідно статті 47
      КЗпПУ\footnote{Кодекс Законів про Працю України} встановлює два види
      дисциплінарного стягнення: догана та звільнення з роботи.
    \item \emph{Матеріальна відповідальність} --- регламентуються КЗпПУ.
      Загальними підставами для накладання матеріальної відповідальності на
      працівника є наявність прямої дійсної шкоди, провина працівника,
      протиправні дії працівника, \ldots
    \item \emph{Кримінальна відповідальність} --- ККУ\footnote{Кримінальний
      Кодекс України} 271--275.
      Якщо була загроза загибелі людей або інших тяжких наслідків.
  \end{itemize}

\section{Організація охорони праці на підприємстві}
  Три центри управління охороною праці:
  \begin{itemize}
    \item Державне управління.
    \item Управління з боку власника підприємства.
    \item Управління з боку працівника підприємства.
  \end{itemize}

  СУОП\footnote{Система управління охороною праці} має наступні основні
  функції:
  \begin{itemize}
    \item прогнозування та планування робіт, їх фінансування;
    \item організація та координація робіт (розробка стандарту СУОП);
    \item аналіз та оцінка стану умов і безпеки праці;
    \item стимулювання робіт по вдосконалення охорони праці.
  \end{itemize}

  Основні завдання управління охороною праці:
  \begin{itemize}
    \item навчання питань з охорони праці;
    \item забезпечення безпечності технологічних процесів, виробничого
      устаткування, будівель і споруд;
    \item забезпечення працівників засобами індивідуального захисту.
  \end{itemize}

  Основною формую планування є розроблення комплексного плану підприємства
  щодо покращення стану охорони праці.

  До основних форм контролю за станом охорони праці належать:
  \begin{itemize}
    \item оперативний контроль;
    \item контроль, що проводиться службою охорони праці підприємства;
    \item громадській контроль;
    \item адміністративно"=громадській трьох"=ступеневий контроль;
    \item відомчий контроль.
  \end{itemize}

  Відповідно до типового положення на підприємстві з числом працюючих 50 і
  більше чоловік створюється служба охорони праці.

  Служба охорони праці підпорядковується безпосередньо керівнику підприємства.

  Служба охорони праці виконує такі основні функції:
  \begin{itemize}
    \item опрацьовує ефективну цілісну систему управління охороною праці;
    \item проводить оперативно"=методичне керівництво роботи з охорони праці;
    \item складає разом зі структурним підрозділом підприємства комплексні
      заходи щодо до досягнення встановлених норм;
    \item проводить для працівників вступний інструктаж з охорони праці;
    \item організує забезпечення правилами, нормами, стандартами всі
      структурні підрозділи.
  \end{itemize}
\section{Навчання з питань охорони праці}
  Навчання і перевірка знань з охорони праці відповідно до ДНАОП000-4.12-99
  --- перелік осіб та посад що зобов’язані проходити перевірку знань з охорони
  праці.

  Інструктажі з охорони праці:
  \begin{itemize}
    \item \emph{Вступний інструктаж}.
      Проводиться з:
      \begin{itemize}
        \item з усіма працівниками, що приймаються на постійну чи тимчасову
          роботу;
        \item з працівниками інших організацій, які прибули на підприємство та
          беруть участь у виробничому процесі або виконують інші роботи для
          підприємства;
        \item з учнями та студентами, які прибули на підприємства для
          проходження практики;
        \item з відвідувачами у разі екскурсії на підприємстві;
        \item з усіма вихованцями, учнями, студентами\ldots при оформленні до
          закладу освіти.
      \end{itemize}
    \item Первинний інструктаж.
      Проводиться безпосередньо на робочому місці до початку роботи з:
      \begin{itemize}
        \item новоприйнятим на підприємство;
        \item працівником, що переводиться з одного цеху до іншого;
        \item працівником, що буде виконувати нову для нього роботу;
        \item відрядженим працівником, що бере безпосередню участь у
          виробничому процесі цього підприємства.
      \end{itemize}
    \item Повторний інструктаж.
      Проводиться за працівниками на роботах з
      \begin{itemize}
        \item підвищеною небезпекою один раз на три місяці,
        \item для решти працівників --- один раз на шість місяців.
      \end{itemize}
    \item Позаплановий інструктаж.
      \begin{itemize}
        \item при введені в дію нових нормативних актів,
        \item при зміні технологічних процесів.
      \end{itemize}
    \item Цільовий інструктаж.
      Проводиться з працівниками
      \begin{itemize}
        \item при виконанні разових робіт, не передбачених трудовою угодою,
        \item при ліквідації аварії чи стихійного лиха,
        \item при проведенні робіт, на які оформлюються наряд допусків.
      \end{itemize}
  \end{itemize}
\section{Захист від іонізуючих випромінювань}
  \emph{Іонізуючу випромінювання} --- це випромінювання, яке під час взаємодії
  з речовиною безпосередньо або непрямо викликає іонізацію та збуджує атоми та
  молекули.

  Джерела іонізації можуть бути:
  \begin{itemize}
    \item \emph{Природні}.
      До них належать космічні промені та радіоактивні речовинию.
    \item \emph{Штучні}.
      До них належать реактори атомних, пункти захоронення радіоактивних
      відходів, рентгенівські установки, заводи радіохімічних технологій,
      тощо.
  \end{itemize}

  Іонізуючі випромінювання підрозділяються на:
  \begin{itemize}
    \item \emph{Електромагнітне (фотонне)}.
      Які в свою чергу підрозділяються на:
      \begin{itemize}
        \item \emph{$\gamma$"=випромінювання} --- короткохвильове
          випромінювання з довжиною хвилі 0.1 нм.
        \item \emph{Рентгенівське випромінювання} --- довжина хвилі
          $10^{-5}--10^{-6}$ м.
      \end{itemize}
    \item \emph{Корпускулярне}.
      Які в свою чергу підрозділяються на:
      \begin{itemize}
        \item $\alpha$"=випромінювання --- швидкість розповсюдження ---
          $2\cdot 10^{7}$ м/с.
        \item $\beta$"=випромінювання --- швидкість розповсюдження
          наближається до $3\cdot 10^8$ м/с.
      \end{itemize}
  \end{itemize}

  Ступінь впливу залежить від виду випромінювання, величини поглиненої
  тканиною енергії, терміну дії, розміру опроміненої поверхні й індивідуальних
  особливостей людини.

  Нормування іонізуючого випромінювання: основні документи
  НРБУ-97\footnote{Норма радіаційної безпеки України} та
  ОСПУ-97\footnote{Основні Санітарні Правила Роботи з Радіоактивними
  Речовинами}.

  Засоби та заходи захисту від іонізуючого випромінювання:
  \begin{itemize}
    \item Зменшення потужності джерела випромінювання до
      мінімально"=необхідних значень, зменшивши його активність.
    \item Використання ізотопів меншої активності.
    \item Скорочення часу роботи з джерелом випромінювання.
    \item Віддалення робочого місця від джерела.
    \item Використання роботизованих комплексів.
    \item Застосування устаткування з автоматичним і дистанційним керуванням.
    \item Екранування джерел випромінювання.
    \item Застосування засобів індивідуального захисту.
    \item Впровадження організаційних, санітарно"=гігієнічних і
      лікувально"=профілактичних заходів.
    \item Дозометричний контроль.
  \end{itemize}

  Безпека на АЕС включає у себе
  \begin{itemize}
    \item йодну профілактику,
    \item респіратори (строк дії --- 2 години),
    \item прийом всередину сорбентів 2--3 рази в тиждень.
  \end{itemize}

  Сприяють виведенню ізотопів плавлені сирки, напівкопчені ковбаси, мармелад,
  желе.
\section{Комплексне дослідження електробезпеки}
  \emph{Електробезпека} --- система організаційних та технічних заходів і
  засобів, що забезпечують захист людей від шкідливого та небезпечного впливу
  електричного струму, електричної дуги, електромагнітного поля і статичної
  електрики.

  Основними причинами електротравматизму на виробництві є:
  \begin{itemize}
    \item випадкове доторкання до неізольованих струмопровідних частин
      електроустаткування;
    \item використання несправних ручних електроінструментів;
    \item застосування нестандартних або несправних переносних світильників
      напругою 220 чи 127 В;
    \item робота без надійних захисних засобів та запобіжних пристосувань;
    \item доторкання до незаземлених корпусів електроустаткування, що
      опинилися під напругою внаслідок пошкодження ізоляції;
    \item недотримання правил улаштування технічної експлуатації, та правил
    техніки безпеки при експлуатації.
  \end{itemize}

  Проходячи через організм людини електричний струм справляє на нього
  термічну, електролітичну, механічну та біологічну дії.

  \emph{Електротравма} --- травма, яка спричинена дією електричного струму чи
  електричної дуги.

  Чинники, що впливають на наслідки ураження електричним струмом:
  \begin{itemize}
    \item сила струму,
    \item напруга,
    \item опір тіла людини,
    \item вид та частота струму,
    \item тривалість дії струму,
    \item шлях проходження струму через тіло людини,
    \item індивідуальні особливості тіла людини,
    \item умови навколишнього середовища.
  \end{itemize}

  Розділяють три основні порогові значення сили струму:
  \begin{itemize}
    \item \emph{пороговий відчутний струм} --- найменше значення електричного
      струму, що викликає при проходженні через організм людини відчутні
      подразнення;
    \item \emph{пороговий невідпускаючий струм} --- найменше значення
      електричного струму, яке викликає судоми;
    \item \emph{пороговий фибриляційний струм} --- життєво"=небезпечний струм.
  \end{itemize}

  \emph{Напруга доторкання} --- напруга між двома точками електричного кола,
  до яких одночасно доторкається людина.
\end{document}
