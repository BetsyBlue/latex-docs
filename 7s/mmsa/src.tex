% vim:spelllang=ru
\documentclass[a4paper,12pt,notitlepage,pdftex,headsepline]{scrartcl}

\usepackage{a4wide}
\usepackage{cmap}
\usepackage[utf8]{inputenc}
\usepackage[russian]{babel}
\usepackage[T2A]{fontenc}
\usepackage{concrete}

\usepackage{textcase}

\usepackage[pdftex]{graphicx}

\usepackage{lscape}

\pdfcompresslevel=9 % сжимать PDF
\usepackage{pdflscape} % для возможности альбомного размещения некоторых страниц
\usepackage[pdftex]{hyperref}
% настройка ссылок в оглавлении для pdf формата
\hypersetup{unicode=true,
            pdftitle={Конспект по ММСА},
            pdfauthor={Михаил Погода},
            pdfcreator={pdflatex},
            pdfsubject={},
            pdfborder    = {0 0 0},
            bookmarksopen,
            bookmarksnumbered,
            bookmarksopenlevel = 2,
            pdfkeywords={},
            colorlinks=true, % установка цвета ссылок в оглавлении
            citecolor=black,
            filecolor=black,
            linkcolor=black,
            urlcolor=blue}

\usepackage{amsmath}
\usepackage{amssymb}

\author{Михаил Погода}
\title{Конспект по ММСА}
\date{\today}

\begin{document}
\begin{titlepage}
  \maketitle
\end{titlepage}

\tableofcontents
\newpage

\section{Введение}
  Предметом исследования операций является изучение сложных организационных
  систем на основе системного подхода и оптимизация управления как каждой
  отдельной подсистемы, так и всей системы в целом, для получения решений,
  которые наилучшим образом с точки зрения одного или нескольких критериев
  удовлетворяет целям всей организации.

  Этапы исследования операций:
  \begin{enumerate}
    \item Постановка задачи заказчиком.
    \item Формализация задачи.
    \item Проверка и корректировка модели.
    \item Нахождение оптимального решения задачи на основании уточнённой модели
      с помощью того или иного метода оптимизации.
    \item Практическое применение полученных результатов.
  \end{enumerate}
\section{Теория массового обслуживания}
  Системой массового обслуживания называется система, предназначенная для
  удовлетворения потока заявок с помощью одного или нескольких каналов
  обслуживания, которые выполняют ряд стандартных для данной системы операций
  обслуживания.

  Элементы системы обслуживания:
  \begin{enumerate}
    \item Поток заявок.
    \item Очередь.
    \item Каналы обслуживания.
    \item Выходной поток.
  \end{enumerate}

  Классификация входного потока:
  \begin{itemize}
    \item конечный;
    \item бесконечный.
  \end{itemize}

  Как приходят заявки:
  \begin{itemize}
    \item по одной;
    \item группой.
  \end{itemize}

  По количеству каналов:
  \begin{itemize}
    \item одно-канальный;
    \item многоканальный.
  \end{itemize}

  По типу обслуживания:
  \begin{itemize}
    \item без ожидания;
    \item с очередями.
  \end{itemize}

  По тому, как влияют количество заявок:
  \begin{itemize}
    \item разомкнутая СМО\footnote{Система массового обслуживания};
    \item замкнутая СМО.
  \end{itemize}

  Характеристики СМО:
  \begin{itemize}
    \item абсолютное пропускная способность;
    \item относительная пропускная способность;
    \item количество каналов;
    \item время обслуживания;
    \item \ldots
  \end{itemize}

  Правила составления уравнений Колмогорова:
  \begin{itemize}
    \item Система уравнений Колмогорова имеет форму Коши. Каждое уравнение
      составляется с помощью рассмотрения вероятности состояния, представленного
      соответствующей вершиной в размеченном графе. Число уравнений системы равно
      числу вершин графа.
    \item Число слагаемых правой части каждого уравнения равно числу дуг,
      принадлежащих этой вершине.
    \item Дугам, которые выходят из данной вершине соответствуют отрицательные
      слагаемые.
    \item Каждое слагаемое равно произведению вероятности состояния
      соответствующей началу рассматриваемой дуги на плотность вероятности
      перехода по дуге.
  \end{itemize}
\end{document}
