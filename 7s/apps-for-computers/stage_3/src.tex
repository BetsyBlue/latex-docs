% vim:spelllang=ru,en
\documentclass[a4paper,12pt,notitlepage,headsepline,pdftex]{scrreprt}

\usepackage{cmap} % чтобы работал поиск по PDF
\usepackage[T2A]{fontenc}
\usepackage[utf8]{inputenc}
\usepackage[english,russian]{babel}
\usepackage{concrete}
\usepackage{cite}
\usepackage{url}

\usepackage{textcase}
\usepackage[pdftex]{graphicx}

\usepackage{lscape}

\pdfcompresslevel=9 % сжимать PDF
\usepackage{pdflscape} % для возможности альбомного размещения некоторых страниц
\usepackage[pdftex]{hyperref}
% настройка ссылок в оглавлении для pdf формата
\hypersetup{unicode=true,
            pdftitle={3ой этап выполнения проекта по ПО ЭВМ},
            pdfauthor={Погода Михаил},
            pdfcreator={pdflatex},
            pdfsubject={},
            pdfborder    = {0 0 0},
            bookmarksopen,
            bookmarksnumbered,
            bookmarksopenlevel = 2,
            pdfkeywords={},
            colorlinks=true, % установка цвета ссылок в оглавлении
            citecolor=black,
            filecolor=black,
            linkcolor=black,
            urlcolor=blue}

\usepackage{amsmath}
\usepackage{amssymb}
\usepackage{moreverb}
\usepackage{indentfirst}
\usepackage{misccorr}

\usepackage{xtab}
\usepackage{nccfoots}

\begin{document}
\begin{titlepage}
  \begin{center}
    \large
    \MakeUppercase{Министерство образования и науки,}

    \MakeUppercase{молодёжи и спорта Украины}

    \mbox{\MakeUppercase{Национальный технический университет Украины}}

    \MakeUppercase{,,Киевский политехнический институт''}

    \addvspace{6pt}

    \normalsize
    Кафедра прикладной математики

    \vfill

    \textbf{Етап №3}

    выполнения курсового проекта

    по дисциплине ,,Программное обеспечение ЭВМ''

    \emph{Выбор, обоснование и освоение метода решения задачи.
          Решение контрольных примеров}
  \end{center}

  \vfill

  \noindent
  Выполнил\\
  студент группы КМ-92\\
  Погода~М.\,В.\\
  \vfill

  \begin{center}
    КИЕВ

    2012
  \end{center}
\end{titlepage}

\emph{Алгоритм Чана} является комбинацией двух более медленных алгоритмов
(сканирование по Грэхему ($O(n\log n)$) и заворачивание по Джарвису ($O(n h)$).
Недостатком сканирования по Грэхему является необходимость сортировки всех
точек по полярному углу, что занимает достаточно много времени $O(n\log n)$.
Заворачивание по Джарвису требует перебора всех точек для каждой из $h$ точек
выпуклой оболочки, что в худшем случае занимает $O(n^2)$.

\section*{Описание метода}
  Идея алгоритма Чана заключается в изначальном делении всех точек на группы
  по $m$ штук в каждой.
  Соответственно, количество групп равно $\displaystyle r = \left[\frac{n}{m}\right]$.
  Для каждой из групп строится выпуклая оболочка сканированием по Грэхему за
  $O(m\log m)$, то есть для всех групп понадобится $O(rm\log m) = O(n\log m)$
  времени.

  Затем, начиная с самой левой нижней точки, для получившихся в результате
  разбиения оболочек строится общая выпуклая оболочка по Джарвису.
  При этом следующая подходящая для выпуклой оболочки точка находится за
  $O(r\log m)$, так как для того, чтобы найти точку с максимальным тангенсом
  по отношению к рассматриваемой точке в $m$-угольнике достаточно затратить
  $(O(\log m)$ (точки в $m$-угольнике были отсортированы по полярному углу во
  время выполнения алгоритма сканирования Грэхема).
  В итоге, на обход требуется $\displaystyle O\left(hr\log m\right) =
  O\left(\frac{hn}{m} \log m\right)$ времени.

  То есть алгоритм Чана работает за $\displaystyle O\left( \left(n +
  \frac{hn}{m}\right) \log m\right)$, при этом, если получить $m = h$, то за
  $O\left(n\log h\right)$.

\section*{Выбор числа точек $m$}
  Ясно, что обход по Джарвису, а следовательно и весь алгоритм, будет
  корректно работать только если $m \geqslant h$, но как заранее узнать
  сколько точек будет в выпуклой оболочке?
  Надо запускать алгоритм несколько раз, подбирая $m$ и, если $m < h$, то
  алгоритм будет возвращать ошибку.
  Если делать подбор бинарным поиском по $n$, то в итоге получится время
  $\displaystyle O\left( \left(n\log h\right)\log n\right) = O\left(n\log
  n\right)$, что достаточно долго.

  Процесс можно ускорить, если начать с маленького значения $m$ и после
  значительно его увеличивать, пока не получится $m \geqslant h$.
  Например, взять $\displaystyle 2^{2^t}$.
  При этом $t$-я итерация займет $\displaystyle O\left( n\log 2^{2^t} \right)
  = O\left( n 2^t \right)$.
  Процесс поиска завершится, когда $\displaystyle 2^{2^t} \geqslant h$, то
  есть $t = \lceil \lg\lg h \rceil$ ($\lg$ --- двоичный логарифм).

  В итоге:
  \[
    \sum_{t = 1}^{\lg\lg h} n 2^t = n\sum_{t = 1}^{\lg\lg h} 2^t \leqslant n
    2^{1 + \lg\lg h} = 2 n \lg h = O(n\log h)
  \]

\section*{Контрольные примеры}
  \subsection*{Пример №1}
    \[
      A = \left\{
        (0, 0), (1, 0), (2, 0),
        (0, 1), (1, 1), (2, 1),
        (0, 2), (1, 2), (2, 2)
      \right\}
    \]

    \[ \mathop{Conv} A = \left\{
        (0, 0), (2, 0), (2, 2), (0, 2)
       \right\}
    \]
  \subsection*{Пример №2}
    \[
      A = \left\{
        (0, 0), (1, 0), (2, 0),
        (0, 1), (1, 1),
        (0, 2)
      \right\}
    \]

    \[
      \mathop{Conv} A = \left\{
        (0, 0), (2, 0), (0, 2)
      \right\}
    \]
\end{document}
