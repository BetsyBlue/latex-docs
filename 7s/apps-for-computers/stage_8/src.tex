% vim:spelllang=ru,en
\documentclass[a4paper,12pt,notitlepage,headsepline,pdftex]{scrartcl}

\usepackage{cmap} % чтобы работал поиск по PDF
\usepackage[T2A]{fontenc}
\usepackage[utf8]{inputenc}
\usepackage[english,russian]{babel}
\usepackage{concrete}
\usepackage{cite}
\usepackage{url}

\usepackage{textcase}
\usepackage[pdftex]{graphicx}

\usepackage{lscape}

\pdfcompresslevel=9 % сжимать PDF
\usepackage{pdflscape} % для возможности альбомного размещения некоторых страниц
\usepackage[pdftex]{hyperref}
% настройка ссылок в оглавлении для pdf формата
\hypersetup{unicode=true,
            pdftitle={8ой этап выполнения проекта по ПО ЭВМ},
            pdfauthor={Погода Михаил},
            pdfcreator={pdflatex},
            pdfsubject={},
            pdfborder    = {0 0 0},
            bookmarksopen,
            bookmarksnumbered,
            bookmarksopenlevel = 2,
            pdfkeywords={},
            colorlinks=true, % установка цвета ссылок в оглавлении
            citecolor=black,
            filecolor=black,
            linkcolor=black,
            urlcolor=blue}

\usepackage{amsmath}
\usepackage{amssymb}
\usepackage{moreverb}
\usepackage{indentfirst}
\usepackage{misccorr}

\usepackage{xtab}
\usepackage{nccfoots}

\usepackage{listings}
\lstloadlanguages{C++}
\lstset{language=C++,basicstyle=\scriptsize,frame=tb,commentstyle=\itshape,stringstyle=\bfseries,extendedchars=false}

\begin{document}
\begin{titlepage}
  \begin{center}
    \large
    \MakeUppercase{Министерство образования и науки,}

    \MakeUppercase{молодёжи и спорта Украины}

    \mbox{\MakeUppercase{Национальный технический университет Украины}}

    \MakeUppercase{,,Киевский политехнический институт''}

    \addvspace{6pt}

    \normalsize
    Кафедра прикладной математики

    \vfill

    \textbf{Етап №8}

    выполнения курсового проекта

    по дисциплине ,,Программное обеспечение ЭВМ''

    \emph{Программная реализация}
  \end{center}

  \vfill

  \noindent
  Выполнил\\
  студент группы КМ-92\\
  Погода~М.\,В.\\
  \vfill

  06.12.2012

  \vfill

  \begin{center}
    КИЕВ

    2012
  \end{center}
\end{titlepage}

Для представления точки на плоскости используется АТД\footnote{Абстрактный тип
данных} \texttt{PointF}, интерфейс которого описан в файле
\textit{pointf.hxx}, а реализация --- в файле \textit{pointf.cxx}.

Сама же реализация метода поиска выпуклой оболочки находится в классе
\texttt{Convex\_Hull} (файлы \textit{convex\_hull.hxx} и
\textit{convex\_hull.cxx}).

\section*{pointf.hxx}
\lstinputlisting{pointf.hxx}
\section*{pointf.cxx}
\lstinputlisting{pointf.cxx}
\section*{convex\_hull.hxx}
\lstinputlisting{convex_hull.hxx}
\section*{convex\_hull.cxx}
\lstinputlisting{convex_hull.cxx}

\end{document}
