% vim:spelllang=ru,en
\documentclass[a4paper,12pt,notitlepage,headsepline,pdftex]{scrartcl}

\usepackage{cmap} % чтобы работал поиск по PDF
\usepackage[T2A]{fontenc}
\usepackage[utf8]{inputenc}
\usepackage[english,russian]{babel}
\usepackage{concrete}
\usepackage{cite}
\usepackage{url}

\usepackage{textcase}
\usepackage[pdftex]{graphicx}

\usepackage{lscape}

\pdfcompresslevel=9 % сжимать PDF
\usepackage{pdflscape} % для возможности альбомного размещения некоторых страниц
\usepackage[pdftex]{hyperref}
% настройка ссылок в оглавлении для pdf формата
\hypersetup{unicode=true,
            pdftitle={ПОЭВМ Лаба №7},
            pdfauthor={Погода Михаил},
            pdfcreator={pdflatex},
            pdfsubject={},
            pdfborder    = {0 0 0},
            bookmarksopen,
            bookmarksnumbered,
            bookmarksopenlevel = 2,
            pdfkeywords={},
            colorlinks=true, % установка цвета ссылок в оглавлении
            citecolor=black,
            filecolor=black,
            linkcolor=black,
            urlcolor=blue}

\usepackage{amsmath}
\usepackage{amssymb}
\usepackage{moreverb}
\usepackage{indentfirst}
\usepackage{misccorr}

\usepackage{xtab}
\usepackage{nccfoots}
\usepackage{listings}

\lstloadlanguages{matlab}
\lstset{language=matlab,basicstyle=\scriptsize,frame=tb,commentstyle=\itshape,stringstyle=\bfseries,extendedchars=false}
\begin{document}
\begin{titlepage}
  \begin{center}
    \large
    \MakeUppercase{Министерство образования и науки,}

    \MakeUppercase{молодёжи и спорта Украины}

    \mbox{\MakeUppercase{Национальный технический университет Украины}}

    \MakeUppercase{,,Киевский политехнический институт''}

    \addvspace{6pt}

    \normalsize
    Кафедра прикладной математики

    \vfill

    \textbf{Отчёт}

    Лабораторная работа \No 7

    по дисциплине ,,Программное обеспечение ЭВМ''

    \emph{,,Язык программирования matlab''}
  \end{center}

  \vfill

  \noindent
  \begin{minipage}{0.3\textwidth}
    Выполнил

    студент группы КМ-92

    Погода~М.\,В.
  \end{minipage}
  \hfill
  \begin{minipage}{0.4\textwidth}
    Проверила:

    Ковальчук"=Химюк~Л.\,А.
  \end{minipage}
  \vfill

  \begin{center}
    КИЕВ

    2012
  \end{center}
\end{titlepage}
  В рамках этой лабораторной работы были реализованы метод Свена, позволяющий
  найти интервал неопределённости, метод золотого сечения, позволяющий
  уточнить полученный интервал неопределённости, а также метод Пауэлла,
  позволяющий найти локальный оптимум функции от многих переменных.

  \section*{sven.m}
    \lstinputlisting{/home/projects/mo/sven.m}
  \section*{golden\_ratio\_method.m}
    \lstinputlisting{/home/projects/mo/golden_ratio_method.m}
  \section*{powell\_method.m}
    \lstinputlisting{/home/projects/mo/method.m}
\end{document}
