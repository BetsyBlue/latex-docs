% vim:spelllang=ru,en
\documentclass[a4paper,12pt,notitlepage,headsepline,pdftex]{scrartcl}

\usepackage{cmap} % чтобы работал поиск по PDF
\usepackage[T2A]{fontenc}
\usepackage[utf8]{inputenc}
\usepackage[english,russian]{babel}
\usepackage{lmodern}
\usepackage{cite}
\usepackage{url}

\usepackage{textcase}
\usepackage[pdftex]{graphicx}

\usepackage{lscape}

\pdfcompresslevel=9 % сжимать PDF
\usepackage{pdflscape} % для возможности альбомного размещения некоторых страниц
\usepackage[pdftex]{hyperref}
% настройка ссылок в оглавлении для pdf формата
\hypersetup{unicode=true,
            pdftitle={5ой этап выполнения проекта по ПО ЭВМ},
            pdfauthor={Погода Михаил},
            pdfcreator={pdflatex},
            pdfsubject={},
            pdfborder    = {0 0 0},
            bookmarksopen,
            bookmarksnumbered,
            bookmarksopenlevel = 2,
            pdfkeywords={},
            colorlinks=true, % установка цвета ссылок в оглавлении
            citecolor=black,
            filecolor=black,
            linkcolor=black,
            urlcolor=blue}

\usepackage{amsmath}
\usepackage{amssymb}
\usepackage{moreverb}
\usepackage{indentfirst}
\usepackage{misccorr}

\usepackage{xtab}
\usepackage{nccfoots}

\begin{document}
\begin{titlepage}
  \begin{center}
    \large
    \MakeUppercase{Министерство образования и науки,}

    \MakeUppercase{молодёжи и спорта Украины}

    \mbox{\MakeUppercase{Национальный технический университет Украины}}

    \MakeUppercase{,,Киевский политехнический институт''}

    \addvspace{6pt}

    \normalsize
    Кафедра прикладной математики

    \vfill

    \textbf{Етап №5}

    выполнения курсового проекта

    по дисциплине ,,Программное обеспечение ЭВМ''

    \emph{Определение состава и формата исходных данных и результатов для
    программы}
  \end{center}

  \vfill

  \noindent
  Выполнил\\
  студент группы КМ-92\\
  Погода~М.\,В.\\
  \vfill

  \begin{center}
    КИЕВ

    2012
  \end{center}
\end{titlepage}

Поиск выпуклой оболочки на плоскости заключается в выборе из множества всех
точек таких точек, которые создают выпуклую оболочку этого множества.

Таким образом,
\begin{equation}
  \mathop{conv} A : \mathbb{R}^2 \rightarrow \mathbb{R}^2
  \label{eq:1}
\end{equation}

Для представления точки на плоскости можно использовать абстрактный тип
данных, хранящий её координаты.

Таким образом, программа принимает на вход программа получает множество
(массив, список, коллекцию) пар вещественных чисел, и возвращает тоже
множество таких пар.

Между подпрограммами существуют такие связи:
\begin{itemize}
  \item Управляющая подпрограмма обрабатывает сигналы от других модулей и
    передаёт им соответствующие сигналы управления;
  \item Подпрограмма ввода принимает сигнал управления и данные, вводимые
    пользователем (список точек либо имя файла, содержащего точки),
    обрабатывает эти данные (распознает вводимые пользователем точки/считывает
    данные из файла) и передаёт управляющей программе множество точек;
  \item Подпрограмма вывода принимает сигнал управления и данные, преобразует
    их в необходимый вид (бинарный/текстовый) и выводит их на устройство
    ввода;
  \item Подпрограмма графического интерфейса пользователя принимает
    управляющие сигналы и различные действия пользователя;
  \item Подпрограмма ведения отчёта принимает сообщения о ходе работы
    программы от других модулей программы, формирует из них отчёт.
\end{itemize}
\end{document}
