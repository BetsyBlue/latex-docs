% vim:spelllang=uk
\documentclass[a4paper,12pt,notitlepage,pdftex,headsepline]{scrartcl}

\usepackage{cmap}
\usepackage[utf8]{inputenc}
\usepackage[english,ukrainian]{babel}
\usepackage[T2A]{fontenc}
\usepackage{a4wide}
\usepackage{concrete}

\pdfcompresslevel=9
\usepackage[pdftex]{hyperref}
\hypersetup{unicode=true,
            pdftitle={Економіка підприємства та організація виробництва},
            pdfauthor={},
            pdfcreator={pdflatex},
            pdfsubject={},
            pdfborder    = {0 0 0},
            bookmarksopen,
            bookmarksnumbered,
            bookmarksopenlevel = 2,
            pdfkeywords={},
            colorlinks=true,
            citecolor=black,
            filecolor=black,
            linkcolor=black,
            urlcolor=blue}

\author{Погода Михайло}
\title{Економіка підприємства та організація виробництва}
\date{\today}

\begin{document}
\begin{titlepage}
  \maketitle
\end{titlepage}

\tableofcontents
\newpage

\section{Вступ}
  Основною ланкою економіки держави є підприємство.
  Конкурентоспроможність держави "--- це насамперед конкурентоспроможність її
  підприємств, які формують економічну ,,піраміду'' держави.

  Для підтримання належного рівня конкурентоспроможності власники, керівники,
  усі працівники підприємства повинні піклуватися про ефективне використання
  матеріальних, трудових, фінансових, інформаційних ресурсів, застосовувати
  високопродуктивне обладнання, прогресивну технологію.

  Організація виробництва "--- координація та оптимізація в часі та просторі
  процесів праці й матеріальних елементів виробництва з метою створення
  організаційних передумов для досягнення максимальних результатів діяльності
  підприємства.

  Діяльність підприємств в Україні регулюється Господарським кодексом України.

\section{Підприємство в системі ринкових відносин}

  Підприємство --- самостійних суб’єкт господарювання, що має права
  юридичної особи та здійснює виробничу, науково-дослідницьку та комерційну
  діяльність з метою одержання відповідного прибутку.

  Підприємство має головну мету:
  \begin{itemize}
    \item розширення ринків збуту;
    \item вихід на зовнішні ринки;
    \item поліпшення природного середовища
    \item та інші.
  \end{itemize}

  Підприємство набуває прав юридичної особи від дня його державної
  реєстрації.

  Підприємство діє на підставі статуту.
  У статуті підприємства визначаються власники та найменування підприємства,
  його місцезнаходження, предмет і цілі діяльності, органи управління,
  порядок утворення майна підприємства, умови реорганізації та припинення
  діяльності підприємства.

  Майно підприємства становлять
  \begin{itemize}
    \item основні фонди;
    \item оборотні кошти;
    \item нематеріальні активи;
    \item також інші цінності, що відображаються в самостійному балансі
      підприємства.
  \end{itemize}

  Підприємство має
  \begin{itemize}
    \item самостійний баланс,
    \item розрахунковий рахунок,
    \item печатки зі своїм найменуванням,
    \item свій товарний знак (для промислового підприємства).
  \end{itemize}

  Продукція --- результат виробничої діяльності підприємства.
  Її поділяють на товари та послуги.
  Види продукції залежать від галузі промисловості.

  Обсяг продукції підприємства вимірюється в натуральних (штуки, м$^3$,
  \ldots), трудових нормогодин та вартісних показниках.

  До вартісних показників обсягу продукції належать:
  \begin{itemize}
    \item товарна продукція;
    \item валова продукція;
    \item реалізована продукція;
    \item чиста продукція.
  \end{itemize}

  \subsection{Організаційно-правові форми підприємства}
  Характеристику організаційно-правових форм підприємств можна дати,
  користуючись такими кваліфікаційними ознаками:
  \begin{itemize}
    \item форма власності майна підприємства;
    \item форма вкладення капіталу;
    \item національна належність капіталу;
    \item правовий статус і форми господарювання.
  \end{itemize}
\section{Ресурси підприємства та їх використання}
  Ресурси:
    \begin{itemize}
      \item Основні виробничі фонди.
      \item Оборотні засоби.
      \item Праця.
        \begin{itemize}
          \item Персонал.
          \item Продуктивність праці.
          \item Заробітна плата.
        \end{itemize}
    \end{itemize}
\section{Витрати та ціноутворення}
  \begin{itemize}
    \item Собівартість продукції \textbf{КОНТРОЛЬНА РОБОТА}.
    \item Ціноутворення.
  \end{itemize}
\section{Результати діяльності підприємства}
  \begin{itemize}
    \item Прибуток підприємства та його використання.
    \item Рентабельність.
  \end{itemize}
\section{Організація виробництва}
  Сітьове планування:
    \begin{itemize}
      \item Параметри сетьових моделей та їх розрахунок.
      \item Правила побудови сетьових моделей.
      \item Методи розрахунку сетьового графіку.
      \item Аналіз сетьових моделей.
      \item Оптимізація сетьового графіку.
    \end{itemize}
\end{document}
