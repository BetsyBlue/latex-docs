% vim:spelllang=uk
\documentclass[a5paper,10pt,notitlepage,pdftex,headsepline]{scrartcl}

\usepackage{cmap}
\usepackage[utf8]{inputenc}
\usepackage[english,russian,ukrainian]{babel}
\usepackage[T2A]{fontenc}
\usepackage{concrete}
\usepackage{fullpage}

\pdfcompresslevel=9
\usepackage[pdftex]{hyperref}
\hypersetup{unicode=true,
            pdftitle={Економіка підприємства та організація виробництва},
            pdfauthor={},
            pdfcreator={pdflatex},
            pdfsubject={},
            pdfborder    = {0 0 0},
            bookmarksopen,
            bookmarksnumbered,
            bookmarksopenlevel = 2,
            pdfkeywords={},
            colorlinks=true,
            citecolor=black,
            filecolor=black,
            linkcolor=black,
            urlcolor=blue}

\author{Погода Михайло}
\title{Економіка підприємства та організація виробництва}
\date{\today}

\usepackage{amsmath}
\begin{document}
  \maketitle

\section{Вступ}
  Основною ланкою економіки держави є підприємство.
  Конкурентоспроможність держави "--- це насамперед конкурентоспроможність її
  підприємств, які формують економічну ,,піраміду'' держави.

  Для підтримання належного рівня конкурентоспроможності власники, керівники,
  усі працівники підприємства повинні піклуватися про ефективне використання
  матеріальних, трудових, фінансових, інформаційних ресурсів, застосовувати
  високопродуктивне обладнання, прогресивну технологію.

  Організація виробництва "--- координація та оптимізація в часі та просторі
  процесів праці й матеріальних елементів виробництва з метою створення
  організаційних передумов для досягнення максимальних результатів діяльності
  підприємства.

  Діяльність підприємств в Україні регулюється Господарським кодексом України.

\section{Підприємство в системі ринкових відносин}

  Підприємство --- самостійних суб’єкт господарювання, що має права
  юридичної особи та здійснює виробничу, науково"=дослідницьку та комерційну
  діяльність з метою одержання відповідного прибутку.

  Підприємство має головну мету:
  \begin{itemize}
    \item розширення ринків збуту;
    \item вихід на зовнішні ринки;
    \item поліпшення природного середовища
    \item та інші.
  \end{itemize}

  Підприємство набуває прав юридичної особи від дня його державної
  реєстрації.

  Підприємство діє на підставі статуту.
  У статуті підприємства визначаються власники та найменування підприємства,
  його місцезнаходження, предмет і цілі діяльності, органи управління,
  порядок утворення майна підприємства, умови реорганізації та припинення
  діяльності підприємства.

  Майно підприємства становлять
  \begin{itemize}
    \item основні фонди;
    \item оборотні кошти;
    \item нематеріальні активи;
    \item також інші цінності, що відображаються в самостійному балансі
      підприємства.
  \end{itemize}

  Підприємство має
  \begin{itemize}
    \item самостійний баланс,
    \item розрахунковий рахунок,
    \item печатки зі своїм найменуванням,
    \item свій товарний знак (для промислового підприємства).
  \end{itemize}

  Продукція --- результат виробничої діяльності підприємства.
  Її поділяють на товари та послуги.
  Види продукції залежать від галузі промисловості.

  Обсяг продукції підприємства вимірюється в натуральних (штуки, м$^3$,
  \ldots), трудових нормогодин та вартісних показниках.

  До вартісних показників обсягу продукції належать:
  \begin{itemize}
    \item товарна продукція;
    \item валова продукція;
    \item реалізована продукція;
    \item чиста продукція.
  \end{itemize}


  \subsection{Організаційно-правові форми підприємства}
  Характеристику організаційно"=правових форм підприємств можна дати,
  користуючись такими кваліфікаційними ознаками:
  \begin{itemize}
    \item форма власності майна підприємства;
    \item форма вкладення капіталу;
    \item національна належність капіталу;
    \item правовий статус і форми господарювання.
  \end{itemize}
\section{Ресурси підприємства та їх використання}
  \subsection{Основні виробничі фонди}
    \subsubsection{Загальна характеристика виробничих фондів підприємств}
      В процесі виробництва працівники підприємства за допомогою засобів праці
      здійснюють вплив на сировину, матеріали, перетворюючи їх на готову
      продукцію.
      \emph{Засоби праці} (машини, устаткування, будівлі, транспортні засоби)
      разом з \emph{предметами праці} (сировиною, матеріалами,
      напівфабрикатами) засоби виробництва, які за своїми матеріально"=речовим
      складом становлять \textbf{виробничі фонди підприємства}.

      \emph{Виробничі фонди} --- це в сукупності ОВФ\footnote{Основні
      виробничі фонди} та оборотні засоби.

      \emph{Основні виробничі фонди} --- засоби праці, які діють у процесі
      виробництва протягом довготривалого проміжку часу, зберігаючи при цьому
      свою натуральну речову форму та переносять свою вартість на вартість
      виготовленої продукції по частинам шляхом амортизаційних відрахувань.

      Основні фонди розділяються на:
      \begin{itemize}
        \item Основні \emph{виробничі} фонди.
          Сюди відносяться основні фонди, які беруть участь у виробничих
          процесах безпосередньо, або створюють умови для виробництва.
        \item основні \emph{невиробничі} фонди --- об’єкти
          культурно-побутового призначення.
      \end{itemize}
    \subsubsection{Класифікація, структура, облік та оцінка основних
      виробничих фондів}
      В процесі виробництва продукції беруть участь різні групи основних
      фондів, які можна класифікувати наступним чином:
      \begin{itemize}
        \item будівлі,
        \item споруди,
        \item передавальні пристрої,
        \item машини та устаткування,
        \item транспортні засоби,
        \item інструменти,
        \item вироби та господарський інвентар,
        \item \ldots
      \end{itemize}

      У зв’язку з тривалим функціонуванням та поступовим опрацюванням ОВФ
      існує кілька видів оцінки основних фондів, які можна класифікувати:
      \begin{itemize}
        \item \emph{Повна первісна вартість} ($\Phi_\text{пп}$ --- це така
          вартість, яка включає всі витрати на придбання ОВФ в цінах того
          року, в якому вони були придбані.
          $\Phi_\text{пп}$, взяті на баланс підприємства --- \emph{балансова
          вартість}
        \item \emph{Повна відновлювальна вартість} ($\Phi_\text{пв}$) --- це
          вартість виготовлення ОВФ в умовах даного року (сучасних умов).
        \item \emph{Залишкова вартість} ($\Phi_\text{з}$) --- вартість
          техніки, що ще не перенесена на вартість продукції.
      \end{itemize}

      Окрім вище перерахованих вартостей є ще \emph{ліквідаційна вартість} ---
      це вартість основних фондів ОВФ, які вичерпали свій ресурс і вартість
      яких може бути оцінена підприємством, яке купує цю техніку, або вартість
      металобрухту.

    \subsubsection{Знос, амортизація та відтворення основних фондів}
      ОВФ протягом тривалого функціонування зазнають фізичного та морального
      зносу.

      Необхідною умовою відновлення основних фондів є поступове відшкодування
      їхньої вартості яке здійснюється через \emph{амортизаційні відрахування}.

      \emph{Амортизація ОФ} --- процес перенесення авансованої раніше вартості
      основних фондів на вартість продукції з метою її повної відшкодування.

      \emph{Амортизація} --- також метод включення по частинам вартості
      основних фондів (протягом строку їх роботи) у витрати на виробництво
      продукції та наступне використання цих коштів для повного відновлювання
      використаних ОФ.

      Після реалізації продукції частина основної суми, яка відповідає
      перенесеній вартості основних фондів, надходить в амортизаційній фонд.
      Цей фонд використовується для придбання нових основних фондів взамін
      зношених.

      Для відшкодування вартості зношеної частини основних фондів кожне
      підприємство здійснює амортизаційні відрахування, які обчислюються за
      певними нормами, що характеризують щорічній розмір відрахування у
      відсотках до балансової вартості основних фондів.

      Розрахунки норм амортизаційних відрахувань на повне відновлення
      (\emph{реновація}) здійснюють за формулою (з урахуванням ліквідаційної
      вартості)
      \[
        H_p = \frac{\Phi_\text{пп} - \Phi_\text{л}}{T \cdot \Phi_\text{пп}}
      \]

      Де $T$ --- амортизаційний період, $\Phi_\text{л}$ --- ліквідаційна
      вартість.

      Норма амортизаційних відрахувань на реновації:
      \[
        H_a = \frac{A_p}{\Phi_\text{пп}} \cdot 100\%
      \]

      Норми амортизації встановлюються офіційно.
      \begin{itemize}
        \item 2\% --- будинки
        \item 10\% --- автотранспорт
        \item 6\% --- всі інші фонди, що не увійшли до пер
        \item 15\% --- ЕОМ, мікрофони, ПЗ
      \end{itemize}

      \paragraph{Можливі методи амортизації ОФ}
        Розрахунок амортизаційних нарахувань можна здійснювати різними
        методами:
        \begin{itemize}
          \item \emph{Прямолінійній метод} передбачає річну суму амортизації,
            яка обчислюється по  роках і визначається виходячи з $
            \Phi_\text{пп}$.
          \item Окрім прямолінійного методу на практиці застосовуються
            \emph{нелінійні методи} амортизації.
            Застосування цих методів дозволяє отримати більшу частину
            амортизації (до 70\%) вартості основних фондів вже в першу
            половину строку їх використання.
            В другій половині строку основного фонду величина амортизації
            основних фондів, розрахована нелінійними методами, зменшуються.
            Дані методи ще називаються методами \emph{прискореної амортизації}.
        \end{itemize}

    \subsubsection{Показники використання основних фондів}
      Для характеристики ефективності використання основних фондів
      застосовуються такі показники:
      \begin{itemize}
        \item показники \emph{екстенсивного використання} основних фондів;
        \item показники \emph{інтенсивного використання} --- по потужності й
          продуктивності;
        \item показники \emph{інтегрального використання} --- два попередніх
          разом.
      \end{itemize}

      Узагальнюючими показниками є
      \begin{itemize}
        \item фондовіддача $\displaystyle \Phi_\text{в} = \frac{Q}{\Phi}$;
        \item фондоозброєність $\displaystyle \Phi_\text{о} = \frac{\Phi}{r}$,
          де $r$ --- кількість персоналу;
        \item фондомісткість $\displaystyle \Phi_\text{м} =
          \frac{1}{\Phi_\text{в}}$
      \end{itemize}

    \subsubsection{Нематеріальні ресурси та нематеріальні активи підприємства}
  \subsection{Оборотні засоби}
    \subsubsection{Загальна характеристика оборотних засобів}
      \emph{Оборотні засоби} --- авансовані грошові кости в оборотні фонди та
      фонди обігу для забезпечення виробничого процесу на підприємстві.

      Оборотні засоби підприємства
      \begin{itemize}
        \item Оборотні засоби у сфері виробництва (\emph{оборотні фонди}):
          \begin{itemize}
            \item виробничі запаси,
            \item незавершене виробництво,
            \item витрати майбутніх періодів.
          \end{itemize}
        \item Оборотні засоби у сфері обігу (\emph{фонди обігу}):
          \begin{itemize}
            \item готова продукція,
            \item відвантажена продукція,
            \item дебіторська заборгованість,
            \item грошові кошти на поточному рахунку.
          \end{itemize}
      \end{itemize}

      Вони також поділяються на:
      \begin{itemize}
        \item Нормовані оборотні засоби --- всі оборотні фонди й готова
          продукція.
        \item Ненормовані оборотні засоби.
      \end{itemize}

      \emph{Дебіторська заборгованість} --- заборгованість юридичних та
      фінансових осіб підприємству.
      Наприклад, заборгованість покупців за реалізовану продукцію.

      \emph{Кредиторська заборгованість} --- заборгованість підприємства
      юридичним та фінансовим особам.

      \emph{Витрати майбутніх періодів} --- грошові кошти, які мають місце в
      даному періоді часу, але віднесені на собівартість продукції будуть у
      наступні періоди.

    \subsubsection{Нормування оборотних засобів}
      Кожне підприємство для забезпечення ритмічності виробництва повинно мати
      певну величину товарно"=матеріальних цінностей та грошових коштів на
      кожному етапі.
      Для цієї мети необхідно розрахувати нормативи оборотних засобів.

      \emph{Норматив ОЗ} у виробничих запасах визначають множенням
      середньодобового споживання матеріалів у вартісному виразі на норму
      запасу дня.

      \emph{Норма запасу дня} передбачає транспортний, технологічний
      (підготовчий), поточний, резервний запаси матеріальних ресурсів.

      \paragraph{Виробничий запас}
        Середньодобове споживання матеріалів або добова потреба в них може
        бути визначена за формулою $\displaystyle \mathcal{D} =
        \frac{M_\text{заг}}{360}$

        Серед видів запасу матеріалів є:
        \begin{itemize}
          \item \emph{Поточний запас}.
            Його можна визначити множенням $\mathcal{D}$ на $T_\text{поставки}$.
          \item \emph{Мінімальний запас} (страховий, резервний).
            Його можна визначити множення $\mathcal{D}$ на період зриву.
          \item \emph{Максимальний запас}.
            Він дорівнює сумі двох попередніх запасів.
        \end{itemize}

      \paragraph{Нормування ОЗ у незавершеному виробництві}
        Норматив ОЗ у незавершеному виробництві визначається за формулою:
          \[
            H_\text{н.в.} = C_\text{д} T_\text{ц} K_\text{н.в.},
          \]
        где $C_\text{д}$ --- середньоденний випуск товарної продукції за її
        виробничою собівартістю, $T_\text{ц}$ --- середня тривалість
        виробничого циклу на підприємстві, $K_\text{н.в.}$ --- коефіцієнт
        наростання витрат незавершеного виробництва (співвідношення середньої
        собівартості незавершеного виробництва та собівартості готової
        продукції).
        Він характеризує ступінь готовності виробів.
        Його можна визначити за формулою $\displaystyle K_\text{н.в.} =
        \frac{1 + \gamma}{2}$, де $\gamma$ --- відношення матеріальних витрат
        виробу до його виробничої собівартості.
      \paragraph{Норматив ОЗ готової продукції}
        Визначається множенням вартості (виробничої собівартості) одноденного
        випуску готових виробів на норму їх запасу на складі.
    \subsubsection{Ефективне використання оборотних засобів}
      Ефективне використання ОЗ на підприємстві характеризується показниками:
      \begin{itemize}
        \item Коефіцієнт оборотності ОЗ.
          $\displaystyle K_\text{об.} = \frac{Q_\text{р}}{O_\text{н}}$, де
          $Q_\text{р}$ --- вартість реалізованої продукції за певний період,
          $O_\text{н}$ --- середній залишок нормативних ОЗ за той самий
          період.
          Він показує кількість обертів використання ОЗ за секунду.
        \item Коефіцієнт завантаження.
          $\displaystyle K_\text{завант.} = \frac{1}{K_\text{об.}}$.
          Показує скільки обертається коштів.
        \item Тривалість денного обороту ОЗ.
          $\displaystyle T_\text{об.} = \frac{\mathcal{D}}{K_\text{об.}}$, де
          $\mathcal{D}$ --- кількість днів у розрахунковому періоді.
      \end{itemize}

      Крім цих показників застосовуються часткові або одиничні показники
      використання ОЗ.
      Вони розраховуються як норма розходу витрат та коефіцієнт використання
      матеріалів.

      Норма $q_i = m_\text{вир} + m_\text{відх.}$, де $m_\text{вир}$ --- маса
      готового виробу, $m_\text{відх.}$ --- маса відходів.

      Коефіцієнт $\displaystyle K_\text{в} =
      \frac{m_\text{вир}}{m_\text{заг.}}$, де $m_\text{заг.}$ --- маса
      заготовки.
\section{Персонал підприємства і продуктивність праці}
  \subsection{Персонал підприємства. Його склад. Структура}
    \emph{Персонал підприємства} --- це сукупність постійних працівників, які
    отримали професійну підготовку, мають досвід практичної діяльності.

    Усі працівники підприємства поділяються на 2 групи:
    \begin{itemize}
      \item Промислово"=виробничий персонал;
      \item Невиробничий персонал.
    \end{itemize}

    За характером виконання функцій персонал поділяють на категорії:
    \begin{itemize}
      \item робітники;
      \item технічні службовці;
      \item фахівці;
      \item професіонали;
      \item керівники.
    \end{itemize}

    Кадрова політика підприємства полягає у забезпеченні кожного робочого
    місця й посади персоналом відповідної та належної кваліфікації.
  \subsection{Набір та підготовка персоналу підприємства}
    Модель процесу відбору персоналу має наступні етапи:
    \begin{itemize}
      \item попередній відбір персоналу;
      \item анкетування;
      \item співбесіда;
      \item тестування;
      \item перевірка інформації про рівень кваліфікації;
      \item рішення про прийняття на роботу;
      \item рішення про розмір оплати.
    \end{itemize}

    Крім підготовки спеціалістів практикується перепідготовка управлінського
    персоналу поза межами підприємства та семінари, тренінги за ролями, методи
    службової ротації, який включає як підвищення кваліфікації, так і
    переміщення у інші відділи.
  \subsection{Визначення чисельності окремих категорій персоналу}
    Планову чисельність персоналу визначається конкретною специфікою їхньої
    професійної діяльності.
    Для визначення загальної чисельності ПВП використовують \emph{метод
    корегування базової чисельності}~\eqref{eq:1}, а також більш точний метод
    за \emph{місткістю виробничих програм}~\eqref{eq:2}.
    Визначити чисельність основних працівників за нормами виробітку можна
    обчисли за формулою~\eqref{eq:3}

    Необхідно знати методику розрахунку ефективного фонду робочого часу
    працівника.

    Баланс робочого часу працівника:
    \begin{itemize}
      \item Кількість календарних днів (365)
      \item Вихідні та святкові дні (115)
      \item Номінальний фонд робочого часу (250)
      \item Не виходить на роботу
      \item Явочний робочий час (220)
      \item Тривалість робочого дня (8 годин)
      \item Трати робочого часу (0.5 години)
    \end{itemize}

    \begin{align}
      \text{Ч}_\text{пл} &= \frac{\text{Ч}_\text{б} \Delta Q}{100} \pm \Delta
      \text{Ч}\label{eq:1}\\
      \text{Ч}_\text{пл} &= \frac{T_\text{сум}}{T_\text{рн}
      K_\text{вн}}\label{eq:2}\\
      \text{Ч}_\text{пл.р.} &= \frac{N_\text{ол}}{H_\text{в} T_\text{рп}
      K_\text{вн}}\label{eq:3}\\
      \Delta\text{Ч} &= \frac{\left(t_1 - t_2\right)N_\text{пл}}{T_\text{рп}
      K_\text{вн}} K_\text{ч}\label{eq:4}\\
      \Delta\text{Пл} &= \frac{\Delta\text{Ч}_\text{заг}}{\text{Ч}_\text{вих}
      - \Delta\text{Ч}_\text{заг}} \cdot 100\label{eq:5}\\
      \text{Ч}_\text{вих} &= \frac{\text{Ч}_\text{баз} K_o}{100}\label{eq:6}\\
      \Delta\text{Пп} &= \frac{\Delta T_\text{прог}}{100 - \Delta
        T_\text{прог}} \cdot 100\label{eq:7}\\
      \Delta T_\text{прог} &= \frac{\Delta E_\text{роб. час}}{T_\text{прог.б}}
      \cdot 100\label{eq:8}\\
      \Delta Q_\text{пп} &= 100 - \frac{\Delta\text{Ч}}{\Delta Q} \cdot
      100\label{eq:9}
    \end{align}

    \begin{itemize}
      \item $\text{Ч}_\text{пл}$ --- планова чисельність;
      \item $\text{Ч}_\text{б}$ --- базова чисельність;
      \item $\text{Ч}_\text{заг}$ --- загальна чисельність ПВП планового
        обсягу виробництва осіб;
      \item $\text{Ч}_\text{баз}$ --- чисельність ПВП за базовий період осіб;
      \item $\Delta Q$ --- плановий темп зростання обсягу виробництва
        продукції у \%;
      \item $\Delta\text{Ч}$ --- сумарна зміна чисельності за пофакторним
        розрахунком можливого зростання продуктивності праці осіб;
      \item $T_\text{сум}$ --- сумарна трудомісткість виробничої програми
        планового року у нормогодинах;
      \item $T_\text{рп}$ --- розрахунковий ефективний фонд робочого часу
        одного працівника в годину;
      \item $K_\text{вн}$ --- середній коефіцієнт виконання норм на
        підприємстві;
      \item $N_\text{пл}$ --- обсяг виробництва на плановий період;
      \item $H_\text{вир}$ --- годинна норма виробітку одного робітника.
    \end{itemize}

    Чисельність управлінського персоналу (керівники, фахівці, службовці)
    визначають за типовим штатним розкладом, при цьому кількість працівників
    управління у кожній групі визначають згідно з нормативами чисельності в
    організації.
  \subsection{Продуктивність праці. Показники та методи її вимірювання}
    \emph{Продуктивність праці} --- це показник ефективності праці, що
    характеризується обсягом її послуг, а з іншого боку її кількістю праці.

    Розрізняють \emph{індивідуальну}, \emph{локальну} та \emph{суспільні}
    продуктивності праці.

    Залежно від прямого чи оберненого співвідношення обсягу продукції, робіт
    або послуг  так кількості праці затраченої на виробництво її обсягу
    розрізняють 2 показники рівня ПП:
    \begin{itemize}
      \item \emph{Виробіток} $\displaystyle B =\frac{Q}{\text{Ч}_\text{пвп}}$.
        Виробіток може визначатися різними способами, залежно від того, якими
        одиницями вимірюється обсяг продукції та затрати праці.
      \item \emph{Трудомісткість} $\displaystyle T_p =
        \frac{\text{Ч}_\text{пвп}}{Q}$.
    \end{itemize}

    Розрізняють наступні методи вимірювання виробітку і відновлювальні
    показники:
    \begin{itemize}
      \item \emph{Натуральні}, відображають динаміку продуктивності праці,
        достатньо наочні і показникові, але мають недолік --- їх неможливо
        застосовувати до різнорідної продукції.
      \item \emph{Трудові}, для оцінювання рівня виробітку на окремих робочих
        місцях та за наявності незавершеного виробництва, яке неможливо
        виміряти у натуральних та вартісних показниках, які вимірюються у
        нормогодинах.
        Ці показники застосовуються на підприємствах для визначення
        чисельності працівників і повинні бути обґрунтовані.
      \item Найбільш універсальними є \emph{вартісні} показники, я яких обсяг
        продукції вимірюється грошовими одиницями.
        Ці показники дають змогу порівняти продуктивність праці з різною
        продукцією.
    \end{itemize}
  \subsection{Чинники та резерви зростання продуктивності праці}
  Чинники зростання ПП\footnote{Продуктивність праці} бувають:
  \begin{itemize}
    \item \emph{Зовнішні} --- ті, що об’єктивно знаходяться поза сферою
      управління даного підприємства (законодавство, політика, конкуренція).
    \item \emph{Внутрішні} --- ті, на які підприємство може впливати (техніка,
      енергія, персонал тощо).
  \end{itemize}

  Розглянемо узагальнені групи чинників ПП:
  \begin{itemize}
    \item \emph{науково"=технічні}: техніка, технології, інновації;
    \item \emph{організаційні}: вдосконалення системи управління організацією;
    \item \emph{економічні}: вдосконалення методів планування, систем оплати
      праці тощо;
    \item \emph{соціальні}: створення належного морально"=психологічного
      клімату, нематеріальні заохочення;
    \item \emph{структурні};
    \item \emph{галузеві};
    \item \emph{природні};
    \item \emph{інші}.
  \end{itemize}

  У практиці реального управління підприємства є актуальною необхідність
  кількісного визначення впливу окремих показників на рівень ПП,

  Кількісний вплив окремих чинників на зміну продуктивності праці визначається
  наступним методом, який називається \emph{факторним методом прогнозування
    продуктивності праці через економію чисельності працівників під впливом
  різних чинників}.

  Економію чисельності працівників за рахунок науково"=технічних чинників
  (впровадження нової техніки, технології) для підвищення технічного рівня
  виробництва розраховується за формулою~\eqref{eq:4}, де $t_1, t_2$ ---
  трудомісткість виготовлення одиниці продукції до і після впровадження
  технологічного заходу (нормогодин), $K_\text{ч}$ --- коефіцієнт часу, який
  визначається шляхом ділення кількості місяців дії заходу на плановий період,
  $\Delta\text{Ч}_\text{заг}$ --- загальна чисельність економії працівників
  знаходиться за сумою по кожній групі чинників.

  Розрахунок прирості ПП можна зробити за формулою~\eqref{eq:5}, де
  $\text{Ч}_\text{вих}$ --- вихідна чисельність персоналу промислового
  виробництва, $K_\text{о}$ --- коефіцієнт обсягу в плановому періоді.

  Можливу зміну в плановому році ПП знаходять за формулою~\eqref{eq:7}, де
  $\Delta T_\text{прог}$ --- процент зменшення (збільшення) трудомісткості
  виробництва в плановому році.

  $\Delta T_\text{прог}$ визначається за формулою~\eqref{eq:8}, де $\Delta E$
  --- економія (зростання) затрат робочого часу на виконання праці в базовому
  році (нормогодини).

  $\Delta Q_\text{пп}$ --- приріст обсягу виробництва за рахунок підвищення
  продуктивності праці, визначається за формулою~\eqref{eq:9}, де $\Delta Q$
  --- зміна обсягу виробництва в плановому році в порівнянні з базовим.

\section{Оплата праці}
  \subsection{Заробітна плата. Її сутність, функції}
    \emph{Оплата праці} --- грошовий вираз вартості робочої сили, який
    виплачується робітникові за виконану роботу або надані послуги і
    спрямований на мотивацію досягнення високого рівня продуктивності праці.

    Розмір, порядок нарахування і виплати ЗП\footnote{Заробітна плата}
    регулюються чинним законодавством України та відповідними указами та
    постановами.

    Основні види ЗП:
    \begin{itemize}
      \item \emph{номінальна}, означає суму грошей, яку отримують працівники
        за свою працю;
      \item \emph{реальна} засвідчує кількість товарів і послуг, які працівник
        може придбати за зароблену суму грошей.
    \end{itemize}

    Оплата праці складається з:
    \begin{itemize}
      \item \emph{Основна} зарплата (70\%).
        Виплачується за виконану роботу відповідно до встановлених норм праці
        у вигляді тарифних ставок (оклад) і відрядних розцінок для робітників
        та посадових окладів для службовців.
      \item \emph{Додаткова} зарплата (30\%).
        Визначається кінцевими результатами діяльності підприємства і виступає
        в ролі доплат, надбавок та компенсаційні виплати, передбачені чинним
        законодавством, а також премії, пов’язані з виконанням виробничих
        функцій.
    \end{itemize}

    Функції ЗП:
    \begin{itemize}
      \item відтворювальна;
      \item стимулююча;
      \item регулююча;
      \item соціальна.
    \end{itemize}

    ЗП, як форма вартості або ціни послуг, значною мірою залежить від
    результатів виробництва.

    Працівники отримують більш високу ЗП якщо працюють в
    несприятливих/шкідливих умовах або якщо їхня праця пов’язана з ризиком.
  \subsection{Тарифна система оплати праці}
    Важливу роль в організації ЗП на підприємствах має тарифна система.

    \emph{Тарифна система} --- сукупність нормативних матеріалів за допомогою
    яких встановлюється рівень зарплати працівників залежно від їх
    кваліфікації, складності робіт та умов праці.

    До числа основних елементів тарифної системи відносяться:
    \begin{itemize}
      \item \emph{Тарифна сітка} --- це сукупність кваліфікаційних розрядів і
        відповідних їм \emph{тарифних коефіцієнтів}, за допомогою яких
        встановлюється безпосередня залежність ЗП працівників від їх
        кваліфікації.
        \emph{Тарифний коефіцієнт}, як елемент тарифної сітки, характеризує
        співвідношення між тарифної ставкою першого розряду і наступними
        тарифними розрядами.
        Тарифний коефіцієнт означає у скільки разів тарифна ставка конкретного
        розряду більша за конкретну ставку першого розряду, коефіцієнт якого
        приймається за 1.
      \item \emph{Тарифна ставка} --- це другий елемент тарифної системи,
        абсолютний розмір зарплати за одиницю робочого часу (годину, день,
        місяць).
        На основі тарифної сітки та тарифної ставки працівника першого розряду
        розраховується тарифні ставки кожного наступного розряду.
      \item \emph{Довідник кваліфікаційних характеристик професій працівників}
        --- систематизований збірник за конкретними професіями, що включені до
        діючого класифікатора професій, тобто це нормативний документ, за
        допомогою якого встановлюється кваліфікаційний статус усіх категорій
        працюючих.
        Таким чином, ці довідники містять систему ,,цензів'', яким повинні
        відповідати працівники певних професій.
      \item \emph{Посадові оклади}.
      \item \emph{Надбавки ЗП}.
    \end{itemize}
  \subsection{Система оплати праці}
\section{Собівартість продукції підприємства}
  % 25.10
  Розглянемо склад та методики розрахунку калькуляції.
  До статті ,,сировина та матеріали'' включається сировина та матеріали, що
  входять до складу продукції.
  Ці витрати визначаються виходячи з норм витрати матеріалів на одиницю
  продукції, цін на матеріали та витрати на транспортну заготівельні роботи.

  Покупні комплектуючі вироби, напівфабрикати. В цю статтю включається їх
  вартість, що підлягають монтажу або додатковій обробці.
  Витрати за цією статтею безпосередньо відносяться на собівартість окремих
  виробів і визначають аналогічні витрати на матеріали.

  Стаття ,,паливо та енергія на технологічні цілі''.
  Сюди відносяться витрати на всі види палива та енергії, що безпосередньо
  використовуються на виробництво продукції.
  Витрати на паливо визначаються аналогічно витратам на матеріали.
  Витрати на енергію визначаються виходячи з витрат на одиницю продукції та
  тарифів на енергію.

  Стаття ,,зворотні відходи''.
  Зворотні відходи безпосередньо визначаються по кожному виробу та
  вираховуються за можливою ціною їх реалізації.

  Стаття ,,Основна заробітна плата робітників''.
  Містить витрати на виплати основної зарплати робітникам, зайнятих
  виробництвом продукції, та включається до собівартості виробу згідно
  прийнятим на підприємстві системам оплати праці.
  Ці витрати за відрядною оплатою праці визначають тарифними ставками та
  нормами витрат часу на виробництвом одиниці продукції.

  Стаття ,,Додаткова зарплата'', до якої відносяться витрати на виплати
  персоналу підприємства додаткової зарплати за працю понад установленої
  норми тощо.
  Ці витрати встановлюються у відсотках до основної зарплати робітників.

  Стаття ,,відрахування на соціальні заходи''.
  Розрахунок даної статті калькуляції здійснюється від суми всіх витрат на
  оплату праці робітників.

  Стаття ,,витрати на утримання та експлуатацію устаткування'' включають
  амортизаційні відрахування від вартості виробничого устаткування, витрати на
  проведення поточного ремонту, платежі за користування наданими в оперативну
  оренду основними засобами.
  На одиницю продукцію ці витрати повинні розподілятися виходячи з величини
  витрати за годину роботи устаткування та тривалості його роботи.
  Існує спрощений метод, який визначається за процентними ставками пропорційно
  до основної зарплати робітників.

  Стаття ,,Загально"=виробничі витрати''.
  Загально"=виробничі витрати, пов’язані з управлінням виробництвом (утримання
  працівників апарату управлінням виробництвом), оплата службових відряджень
  персоналу виробництва, амортизаційні відрахування основних фондів, поточний
  ремонт виробничих приміщень, забезпечення техніки безпеки тощо.
  Ці витрати на одиницю продукції у складі калькуляції визначаються з
  використанням бази розподілу основної зарплати робітників, витрат на
  утримання та експлуатацію устаткування.

  Стаття ,,Втрати від браку''.
  Це вартість остаточно забракованої продукції, витрати на усунення браку
  (якщо цей брак не по вині робочого).

  Стаття ,,інші виробничі витрати'' --- витрати на перевірку виробів на
  відповідність стандартам, проведення періодичних випробувань, монтаж або
  демонтаж виробу, що випробовується.

  Стаття ,,адміністративні витрати''.
  За своїм змістом має аналогію із статтею ,,загально"=виробничі витрати'' і
  містить витрати на оплату праці апарату управління підприємством, утримання,
  ремонт та обслуговування загально"=заводських \dots, підготовка та
  перепідготовка кадрів тощо.

  Стаття ,,загально"=виробничі витрати, або витрати на збут''.
  До цієї статті належать витрати на реалізацію продукції: складські,
  пакувальні, транспортні роботи, а також маркетингові витрати, сплата
  експертного мита тощо.
  Ці витрати відносяться на собівартість відповідних видів продукції
  пропорційно виробничій собівартості.

  Калькуляція собівартості одиниці продукції може бути складена коли повністю
  розроблена конструкція виробу, технологія його виробництва, визначено
  підприємство де буде виготовлятися дані вироби.

  Наближені методи розрахунку собі

    Параметричні
    метод коре

  Напрями зниження собівартості продукції

  %01.11
\section{Ціноутворення}
  \subsection{Ціна: поняття, функції, класифікація}
    \emph{Ціна} --- це грошовий вираз вартості товару, кількість грошей, що
    сплачується або одержується за одиницю товару або послуги.
    Одночасно ціна відображає споживчі властивості (корисність товару),
    ступінь рідкісності товару, силу конкуренції та інші суб’єктивні моменти.

    Ціни у діяльності підприємства виконують три основні функції:
    \begin{itemize}
      \item обліково"=вимірювальна;
      \item стимулююча;
      \item розподільча.
    \end{itemize}

    Модель процесу ціноутворення передбачає такі етапи:
    \begin{enumerate}
      \item Визначення цілей цінової політики.
      \item Розроблення стратегії ціноутворення.
      \item Оцінка витрат виробництва.
      \item Виявлення чинників, які впливають на ціну.
      \item Вибір методу ціноутворення, що дає змогу отримати декілька
        варіантів можливого рівня цін.
      \item Встановлення остаточної ціни, пов’язано з необхідністю повного
        покриття усіх витрат на виробництво, а також для забезпечення
        отриманого прибутку для розширеного відтворення.
    \end{enumerate}

    Ціноутворення --- процес обґрунтування, затверження та перегляду цін і
    тарифів, визначення їх рівня, співвідношення та структури
  \subsection{Гуртові та роздрібні ціни на продукцію та їх формування}
    Ціна на будь-який товар складається з окремих елементів, основними з них є
    собівартість і прибуток.
    Їх наявність у ціні є обов’язковою.
    Крім того до складу ціни можуть входити:
    \begin{itemize}
      \item Податок на Додану Вартість (ПДВ);
      \item Акцизний збір;
      \item Націнки постачальницько"=збитових організацій;
      \item Торгівельні надбавки або знижки.
    \end{itemize}

    Ціна виробника продукції (оптова ціна) формується на стадії виробництва
    продукції або послуги, має проміжний характер, а також компенсує витрати
    на виробництво (собівартість), а також забезпечує плановий прибуток
    підприємству.

    В основі формування оптових (гуртових) цін підприємства лежить калькуляція
    усіх видів витрат, до якої додається величина прибутку, а також ПДВ та
    акцизний збір.

    ПДВ є видом податку, який встановлюється у вигляді процентної надбавки до
    цін і встановлюється за цією ставкою до всього обороту продукції.

    Акцизний збір є видом специфічних акцизів, що встановлюється за
    індивідуальними ставками до кожного товару.

    Оптова ціна промисловості включає оптову ціну підприємства, а також
    постачальницько"=збитову націнку.

    Постачальницько"=збитова націнка включає витрати та прибуток
    постачальницько"=збитових організацій.

    Роздрібна ціна включає оптову ціну промисловості, а також торговельну
    надбавку (знижку).

    Торговельна надбавка покриває витрати торговельних організацій і
    забезпечує їм одержання прибутку
  \subsection{Франкування цін}
    Сучасну ціноутворення передбачає систему франкування цін.

    Франко --- вид торговельної угоди купівлі"=продажу, коли в ціну товару
    включають витрати на його страхування та доставку у місце зазначене в
    договорі.

    Таким чином ціні ,,франко'' --- оптові ціни, які встановлюються з
    урахування передбаченого контрактом порядку відшкодування транспортних
    витрат на доставку товарів покупцеві.

    Так, наприклад, є такі види цін:
    \begin{itemize}
      \item Франко --- станція відправлення --- споживач сплачує необхідні витрати
        на доставку продукції від станції відправлення до місця в якому він
        бажає зберігати товар.
        Також включаються витрати на вантажні роботи.
      \item Франко --- станція призначення --- включаються всі витрати
        пов’язані із доставкою до місця знаходження споживача.
      \item Франко --- кордон.
        Використовується у зовнишньо"=економічних угодах.
        Окрім транспортних витрат на доставку товару до відповідного кордону
        включає витрати на його страхування та митні витрати.
    \end{itemize}

    Тарифи вантажного та пасажирського транспорту передбачають оплату за
    переміщення вантажів і пасажирів, яку беруть транспортні організації з
    відправників вантажів та населення.

    Демпінгова ціна --- ринкова ціна, яку свідомо занижують суб’єкти
    ціноутворення порівняно з формованим на ринку рівнем цін.
    Таку ціну використовують з метою витиснення з ринку конкурентів.
  \subsection{Методи ціноутворення}

\section{Витрати та ціноутворення}
  \begin{itemize}
    \item Собівартість продукції \textbf{КОНТРОЛЬНА РОБОТА}.
    \item Ціноутворення.
  \end{itemize}
\section{Результати діяльності підприємства}
  \begin{itemize}
    \item Прибуток підприємства та його використання.
    \item Рентабельність.
  \end{itemize}
\section{Організація виробництва}
  \begin{otherlanguage}{russian}
    Построение сетевого графика начинается с описания работ, которые необходимо
    выполнить в данном большом комплексе работ.

    Устанавливается логическая последовательность выполнения работ, строится
    сеть начиная с исходного (начального) события.
    Необходимо для каждой работы определить время выполнения.
    Зная длительность работы можно рассчитать все необходимые параметры,
    характеризующие свойства сети.

    После расчёта времени за каждой работой закреплеяются исполнители.
    Важным этапом СПУ является определение параметров сетевого графика, для
    этого каждое событие разбивается на четыре сектора.

    Следующим этапом СПУ является анализ сетевого графика.
    Определяется степень напряжённости каждой работы, а также в целом сетевого
    графика.

    Завершающим этапом сетевого планирования является оптимизация сетевого
    графика.
    Для этого строится документ под названием ,,Карта проекта комплекса работ''.

    Данная карта проекта состоит из двух частей.
    Верхняя часть называется трансфомированный, или преобразованный сетевой
    график в масштабе времени.
    Нижняя часть называется диаграммой загрузки исполнителей.
  \end{otherlanguage}
\end{document}
