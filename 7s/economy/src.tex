% vim:spelllang=uk
\documentclass[a4paper,10pt,notitlepage,pdftex,headsepline]{scrartcl}

\usepackage{cmap}
\usepackage[utf8]{inputenc}
\usepackage[english,ukrainian]{babel}
\usepackage[T2A]{fontenc}
\usepackage{concrete}
\usepackage{fullpage}

\pdfcompresslevel=9
\usepackage[pdftex]{hyperref}
\hypersetup{unicode=true,
            pdftitle={Економіка підприємства та організація виробництва},
            pdfauthor={},
            pdfcreator={pdflatex},
            pdfsubject={},
            pdfborder    = {0 0 0},
            bookmarksopen,
            bookmarksnumbered,
            bookmarksopenlevel = 2,
            pdfkeywords={},
            colorlinks=true,
            citecolor=black,
            filecolor=black,
            linkcolor=black,
            urlcolor=blue}

\author{Погода Михайло}
\title{Економіка підприємства та організація виробництва}
\date{\today}

\usepackage{amsmath}
\begin{document}
\begin{titlepage}
  \maketitle
\end{titlepage}

\tableofcontents
\newpage

\section{Вступ}
  Основною ланкою економіки держави є підприємство.
  Конкурентоспроможність держави "--- це насамперед конкурентоспроможність її
  підприємств, які формують економічну ,,піраміду'' держави.

  Для підтримання належного рівня конкурентоспроможності власники, керівники,
  усі працівники підприємства повинні піклуватися про ефективне використання
  матеріальних, трудових, фінансових, інформаційних ресурсів, застосовувати
  високопродуктивне обладнання, прогресивну технологію.

  Організація виробництва "--- координація та оптимізація в часі та просторі
  процесів праці й матеріальних елементів виробництва з метою створення
  організаційних передумов для досягнення максимальних результатів діяльності
  підприємства.

  Діяльність підприємств в Україні регулюється Господарським кодексом України.

\section{Підприємство в системі ринкових відносин}

  Підприємство --- самостійних суб’єкт господарювання, що має права
  юридичної особи та здійснює виробничу, науково-дослідницьку та комерційну
  діяльність з метою одержання відповідного прибутку.

  Підприємство має головну мету:
  \begin{itemize}
    \item розширення ринків збуту;
    \item вихід на зовнішні ринки;
    \item поліпшення природного середовища
    \item та інші.
  \end{itemize}

  Підприємство набуває прав юридичної особи від дня його державної
  реєстрації.

  Підприємство діє на підставі статуту.
  У статуті підприємства визначаються власники та найменування підприємства,
  його місцезнаходження, предмет і цілі діяльності, органи управління,
  порядок утворення майна підприємства, умови реорганізації та припинення
  діяльності підприємства.

  Майно підприємства становлять
  \begin{itemize}
    \item основні фонди;
    \item оборотні кошти;
    \item нематеріальні активи;
    \item також інші цінності, що відображаються в самостійному балансі
      підприємства.
  \end{itemize}

  Підприємство має
  \begin{itemize}
    \item самостійний баланс,
    \item розрахунковий рахунок,
    \item печатки зі своїм найменуванням,
    \item свій товарний знак (для промислового підприємства).
  \end{itemize}

  Продукція --- результат виробничої діяльності підприємства.
  Її поділяють на товари та послуги.
  Види продукції залежать від галузі промисловості.

  Обсяг продукції підприємства вимірюється в натуральних (штуки, м$^3$,
  \ldots), трудових нормогодин та вартісних показниках.

  До вартісних показників обсягу продукції належать:
  \begin{itemize}
    \item товарна продукція;
    \item валова продукція;
    \item реалізована продукція;
    \item чиста продукція.
  \end{itemize}


  \subsection{Організаційно-правові форми підприємства}
  Характеристику організаційно-правових форм підприємств можна дати,
  користуючись такими кваліфікаційними ознаками:
  \begin{itemize}
    \item форма власності майна підприємства;
    \item форма вкладення капіталу;
    \item національна належність капіталу;
    \item правовий статус і форми господарювання.
  \end{itemize}
\section{Ресурси підприємства та їх використання}
  \subsection{Основні фонди}
    Типи оцінок фондів:
    \begin{itemize}
      \item Повна первісна вартість.
      \item Повна відновлювальна вартість.
      \item Залишкова вартість --- вартість техніки, що ще не перенесена на
        вартість продукції.
    \end{itemize}

    \subsubsection{Знос, амортизація основних виробничих фондів}
      Окрім ППВ, ПВВ, ЗВ є ще ліквідаційна вартість --- вартість основних
      фондів, які вичерпали свій ресурс і вартість яких може бути оцінена
      підприємством, яке купує цю техніку, або вартість металобрухту.

      Основні виробничі фонди протягом тривалого функціонування зазнають
      фізичного та морального зносу.
      Необхідною умовою відновлення основних фондів є поступове відшкодування
      їхньої вартості яке здійснюється через амортизаційні відрахування.

      \emph{Амортизація} основних фондів --- процес перенесення авансованої
      раніше вартості основних фондів на вартість продукції з метою її повної
      відшкодування.
      Амортизація --- також метод включення по частинам вартості основних фондів
      (протягом строку їх роботи) у витрати на виробництво продукції та наступне
      використання цих коштів для повного відновлювання використаних основних
      фондів.

      Після реалізації продукції частина основної суми, яка відповідає
      перенесеній вартості основних фондів, надходить в амортизаційній фонд.
      Цей фонд використовується для придбання нових основних фондів взамін
      зношених.

      Для відшкодування вартості зношеної частини основних фондів кожне
      підприємство здійснює амортизаційні відрахування, які обчислюються за
      певними нормами, що характеризують щорічній розмір відрахування у
      відсотках до балансової вартості основних фондів.

      Розрахунки норм амортизаційних відрахувань на повне відновлення
      (\emph{реновація}) здійснюють за формулою (з урахуванням ліквідаційної
      вартості)
      \[
        H_p = \frac{\Phi_\text{пп} - \Phi_\text{л}}{T \cdot \Phi_\text{пп}}
        \]\[H_a = \frac{A_p}{\Phi_\text{пп}} \]

      Норми амортизації встановлюються офіційно.
      \begin{itemize}
        \item 2\% --- будинки
        \item 10\% --- автотранспорт
        \item 6\% --- всі інші фонди, що не увійшли до пер
        \item 15\% --- ЕОМ, мікрофони, ПЗ
      \end{itemize}

      Прямолінійній метод передбачає річну суму амортизації, яка обчислюється по
      роках і визначається виходячи з $ \Phi_\text{пп}$.

      Окрім прямолінійного методу на практиці застосовуються нелінійні методи
      амортизації.
      Застосування цих методів дозволяє отримати більшу частину амортизації (до
      70\%) вартості основних фондів вже в першу половину строку їх
      використання.
      В другій половині строку основного фонду величина амортизації основних
      фондів, розрахована нелінійними методами, зменшуються.
      Дані методи ще називаються методами \emph{прискореної амортизації}.

    \subsubsection{Показники використання основних виробничих фондів}
      Для характеристики ефективності використання основних фондів
      застосовуються такі показники:
      \begin{itemize}
        \item показники \emph{екстенсивного використання} основних фондів;
        \item показники \emph{інтенсивного використання} --- по потужності й
          продуктивності;
        \item показники \emph{інтегрального використання} --- два попередніх
          разом.
      \end{itemize}

      Узагальнюючими показниками є фондовіддача, фондомісткість й
      фондоозброєність.

  \subsection{Оборотні засоби}
    \subsubsection{Загальна характеристика оборотних засобів}
      Оборотні засоби --- авансовані грошові кости в оборотні фонди та фонди
      обігу для забезпечення виробничого процесу на підприємстві.

      Оборотні засоби підприємства
      \begin{itemize}
        \item Оборотні засоби у сфері виробництва (\emph{оборотні фонди}):
          \begin{itemize}
            \item виробничі запаси,
            \item незавершене виробництво,
            \item витрати майбутніх періодів.
          \end{itemize}
        \item Оборотні засоби у сфері обігу (\emph{фонди обігу}):
          \begin{itemize}
            \item готова продукція,
            \item відвантажена продукція,
            \item дебіторська заборгованість,
            \item грошові кошти на поточному рахунку.
          \end{itemize}
      \end{itemize}

      Вони також поділяються на:
      \begin{itemize}
        \item Нормовані оборотні засоби --- всі оборотні фонди й готова
          продукція.
        \item Ненормовані оборотні засоби.
      \end{itemize}

      Дебіторська заборгованість --- заборгованість юридичних та фінансових
      осіб підприємству.
      Наприклад, заборгованість покупців за реалізовану продукцію.


      Витрати майбутніх періодів --- грошові кошти, які мають місце в даному
      періоді часу, але віднесені на собівартість продукції будуть у наступні
      періоди.

    \subsubsection{Нормування оборотних засобів}
      Кожне підприємство для забезпечення ритмічності виробництва повинно мати
      певну величину матеріальних цінностей та грошових коштів на кожному
      етапі.
      Для цієї мети необхідно розрахувати нормативи оборотних засобів.

      Норматив оборотних засобів у виробничих запасах визначають множення
      середньодобового споживання матеріалів у вартісному виразі на норму
      запасу дня.
      Норма запасу дня передбачає транспортний, технологічний (підготовчий),
      поточний, резервний запаси матеріальних ресурсів.
    \subsubsection{Ефективне використання оборотних засобів}
  \subsection{Праця}
    \begin{itemize}
      \item Персонал.
      \item Продуктивність праці.
      \item Заробітна плата.
    \end{itemize}
\section{Витрати та ціноутворення}
  \begin{itemize}
    \item Собівартість продукції \textbf{КОНТРОЛЬНА РОБОТА}.
    \item Ціноутворення.
  \end{itemize}
\section{Результати діяльності підприємства}
  \begin{itemize}
    \item Прибуток підприємства та його використання.
    \item Рентабельність.
  \end{itemize}
\section{Організація виробництва}
  Сітьове планування:
    \begin{itemize}
      \item Параметри сетьових моделей та їх розрахунок.
      \item Правила побудови сетьових моделей.
      \item Методи розрахунку сетьового графіку.
      \item Аналіз сетьових моделей.
      \item Оптимізація сетьового графіку.
    \end{itemize}
\end{document}
