% vim:spelllang=ru
\documentclass[a4paper,10pt,notitlepage,pdftex,headsepline]{scrartcl}

\usepackage{cmap}
\usepackage[utf8]{inputenc}
\usepackage[russian]{babel}
\usepackage[T2A]{fontenc}
\usepackage{concrete}

\pdfcompresslevel=9
\usepackage[pdftex]{hyperref}
\hypersetup{unicode=true,
            pdftitle={ДКР №1 по экономике предприятий и организации
            производства},
            pdfauthor={Михаил Погода},
            pdfcreator={pdflatex},
            pdfsubject={},
            pdfborder    = {0 0 0},
            bookmarksopen,
            bookmarksnumbered,
            bookmarksopenlevel = 2,
            pdfkeywords={},
            colorlinks=true,
            citecolor=black,
            filecolor=black,
            linkcolor=black,
            urlcolor=blue}

\author{Погода Михаил}
\title{ДКР №1 по экономике предприятий и организации производства}
\date{Вариант №2}

\usepackage{amsmath}
\begin{document}
  \maketitle

\section{Сырьё и материалы}
  Расходы на приобретение материалов вычисляются согласно норм их
  расходования и цен с учётом транспортно"=заготовительных затрат:
  \begin{equation}
    C_\text{м} = \left(1 + k_\text{т.з.}\right) \sum_{i = 1}^n q_\text{вм}^i
    \text{Ц}_\text{м}^i
    \label{eq:1}
  \end{equation}

  В этом случае существует всего один вид материала.
  $k_\text{т.з.} = 20\% = 0.2$ --- коэффициент транспортно"=заготовительных
  затрат.
  $\text{Ц}_\text{м} = 4\,\frac{\text{грн}}{\text{кг}}$ --- цена за единицу
  материала.
  $q_\text{вм} = 5\,\frac{\text{кг}}{\text{издел}}$ --- норма расхода материала
  на единицу продукции.

  \begin{equation}
    C_\text{м} = 1.2 \cdot 5 \cdot 4 = 24\,\frac{\text{грн}}{\text{издел}}
    \label{eq:r1}
  \end{equation}
\section{Покупные комплектующие изделия, полуфабрикаты, работы и услуги
производственного характера сторонних предприятий и организаций}
  \begin{equation}
    C_\text{пф} = \left( 1 + k_\text{т.з.}\right)\sum_{i = 1}^n q_\text{пф}^i
    \text{Ц}_\text{пф}^i
    \label{eq:2}
  \end{equation}

  В нашем случае один вид полуфабрикатов.
  $k_\text{т.з.} = 20\% = 0.2$ --- коэффициент транспортно"=заготовительных
  затрат.
  $\text{Ц}_\text{пф} = 1.3\,\frac{\text{грн}}{\text{шт}}$ --- цена за единицу
  полуфабриката.
  $q_\text{пф} = 11\,\frac{\text{шт}}{\text{издел}}$ --- норма траты
  полуфабрикатов на единицу продукции.
  \begin{equation}
    C_\text{пф} = 1.2 \cdot 1.3 \cdot 11 =
    17.16\,\frac{\text{грн}}{\text{издел}}
    \label{eq:r2}
  \end{equation}
\section{Возвратные отходы}
  \begin{equation}
    C_\text{в} = \sum_{i = 1}^n q_\text{в}^i \text{Ц}_\text{в}^i
    \label{eq:3}
  \end{equation}

  $q_\text{в}$ --- количество возвратных отходов на единицу продукции.
  $\text{Ц}_\text{в} = 1.5\,\frac{\text{грн}}{\text{кг}}$ --- цена единицы
  вида возвратных отходов.

  На одну единицу продукции необходимо 5 кг материала, $k_\text{исп} = 0.83$.

  $q_\text{в} = 5 \cdot \left( 1 - 0.83 \right) \cdot 1.5 = 5 \cdot 0.17 \cdot
  1.5 = 1.28\,\frac{\text{кг}}{\text{издел}}$
  \begin{equation}
    C_\text{в} = 0.85 \cdot 1.5 = 1.275\,\frac{\text{грн}}{\text{издел}}
    \label{eq:r3}
  \end{equation}
\section{Стоимость затрат на топливо, энергию и другие технологические цели}
  Невозможно рассчитать прямо, поэтому эта стоимостью переносится на другие
  расходы по частям.
\section{Основная заработная плата}
  \begin{equation}
    C_\text{з.о.} = \sum_{i = 1}^n C_T^i t_\text{ш}^i
    \label{eq:4}
  \end{equation}

  $C_T = 9.5\,\frac{\text{грн}}{\text{ч}}$ --- почасовая ставка для вида работы.
  $t_\text{ш} = 7\,\frac{\text{ч}}{\text{издел}}$ --- норма времени для вида
  работы.

  \begin{equation}
    C_\text{з.о.} = 9.5 \cdot 7 = 66.5\,\frac{\text{грн}}{\text{издел}}
    \label{eq:r4}
  \end{equation}
\section{Дополнительная заработная плата}
  Расходы по этой статье определяются в процентах относительно основной
  заработной платы.
  Норматив дополнительной заработной платы $\alpha_\text{дз} = 10\% = 0.1$.
  \begin{equation}
    C_\text{дз} = 0.1 \cdot 66.5 = 6.65\,\frac{\text{грн}}{\text{издел}}
    \label{eq:r5}
  \end{equation}
\section{Отчисления на социальное страхование}
  Отчисления на социальное страхование составляет $38\%$ от суммы основной и
  дополнительной заработных плат.

  \begin{equation}
    C_\text{соц} = 0.38 \cdot \left( 6.65 + 66.5 \right) =
    27.8\,\frac{\text{грн}}{\text{издел}}
    \label{eq:r6}
  \end{equation}
\section{Расходы на содержание и эксплуатацию оборудования, амортизация}
  \begin{equation}
    C_\text{о} = \frac{C_\text{зп} \cdot \alpha_\text{об}}{100} =
    \frac{66.5 \cdot 120}{100} = 79.8\,\frac{\text{грн}}{\text{издел}}
    \label{eq:r7}
  \end{equation}
\section{Общепроизводственные расходы}
  Эта статья включает в себя затраты на управление производством, некоторой
  части амортизационных затрат.
  \begin{equation}
    C_\text{ор} = \left( C_\text{зп} + C_\text{о} \right) \cdot
    \frac{\alpha_\text{о.р}}{100} = \left( 66.5 + 79.8 \right) \cdot
    \frac{250}{100} = 365.75\,\frac{\text{грн}}{\text{издел}}
    \label{eq:r8}
  \end{equation}
\section{Производственная себестоимость}
  \begin{equation}
    C_\text{пр} = 24 + 17.16 + 1.28 + 66.5 + 6.65 + 27.8 + 79.8 + 365.75 =
    588.93\,\frac{\text{грн}}{\text{издел}}
    \label{eq:r9}
  \end{equation}
\section{Административные расходы}
  \begin{equation}
    C_\text{адм} = \left( C_\text{зп} + C_\text{о}
    \right)\cdot\frac{\alpha_\text{зх}}{100} = \left( 66.5 + 79.8 \right)
    \cdot 1.5 = 219.45\,\frac{\text{грн}}{\text{издел}}
    \label{eq:r10}
  \end{equation}
\section{Внепроизводственные расходы}
  \begin{equation}
    C_\text{вн.пр} = \frac{C_\text{пр}\cdot\alpha_\text{вн.пр}}{100} = 588.93
    \cdot 0.05 = 29.45\,\frac{\text{грн}}{\text{издел}}
    \label{eq:r11}
  \end{equation}
\section{Полная себестоимость единицы продукции}
  \begin{equation}
    C = C_\text{пр} + C_\text{адм} + C_\text{вн.пр} = 588.93 + 219.45 + 29.45
    = 837.83\,\frac{\text{грн}}{\text{издел}}
    \label{eq:r12}
  \end{equation}
\end{document}
