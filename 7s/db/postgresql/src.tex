% vim:spelllang=ru,en
\documentclass[a4paper,12pt,notitlepage,headsepline,pdftex]{scrartcl}

\usepackage{cmap} % чтобы работал поиск по PDF
\usepackage[T2A]{fontenc}
\usepackage[utf8]{inputenc}
\usepackage[english,russian]{babel}
\usepackage{concrete}
\usepackage{cite}
\usepackage{url}

\usepackage{textcase}
\usepackage[pdftex]{graphicx}

\usepackage{lscape}

\pdfcompresslevel=9 % сжимать PDF
\usepackage{pdflscape} % для возможности альбомного размещения некоторых страниц
\usepackage[pdftex]{hyperref}
% настройка ссылок в оглавлении для pdf формата
\hypersetup{unicode=true,
            pdftitle={PostgreSQL},
            pdfauthor={Погода Михаил},
            pdfcreator={pdflatex},
            pdfsubject={},
            pdfborder    = {0 0 0},
            bookmarksopen,
            bookmarksnumbered,
            bookmarksopenlevel = 2,
            pdfkeywords={},
            colorlinks=true, % установка цвета ссылок в оглавлении
            citecolor=black,
            filecolor=black,
            linkcolor=black,
            urlcolor=blue}

\usepackage{amsmath}
\usepackage{amssymb}
\usepackage{moreverb}
\usepackage{indentfirst}
\usepackage{misccorr}

\usepackage{xtab}
\usepackage{nccfoots}

\begin{document}
\begin{titlepage}
  \begin{center}
    \large
    \MakeUppercase{Министерство образования и науки,}

    \MakeUppercase{молодёжи и спорта Украины}

    \mbox{\MakeUppercase{Национальный технический университет Украины}}

    \MakeUppercase{,,Киевский политехнический институт''}

    \addvspace{6pt}

    \normalsize
    Кафедра прикладной математики

    \vfill

    \textbf{Реферат}

    по дисциплине ,,Базы данных и информационные системы''

    на тему: \textit{,,PostgreSQL''}
  \end{center}

  \vfill

  \noindent
  Выполнил\\
  студент группы КМ-92\\
  Погода~М.\,В.\\
  \vfill

  \begin{center}
    КИЕВ

    2012
  \end{center}
\end{titlepage}

\tableofcontents
\clearpage

\section{Введение}
  \emph{PostgreSQL}\footnote{произносится ,,Пост-Грес-Кью-Эль'', в
    профессиональной среде коротко называется ,,постгрес''} --- свободная
  объектно"=реляционная система управления базами данных (СУБД).

  PostgreSQL ведет свою ,,родословную'' от некоммерческой СУБД
  \textbf{Postgres}, разработанной, как и многие open-source проекты, в
  Калифорнийском университете в Беркли.
  К разработке Postgres, начавшейся в 1986 году, имел непосредственное
  отношение Майкл Стоунбрейкер, руководитель более раннего проекта
  \textbf{Ingres}, на тот момент уже приобретённого компанией \emph{Computer
  Associates}.
  Само название ,,Postgres'' расшифровывалось как ,,Post Ingres'',
  соответственно, при создании Postgres были применены многие уже ранее
  сделанные наработки.

  Стоунбрейкер и его студенты разрабатывали новую СУБД в течение восьми лет, с
  1986 по 1994 год.
  За этот период в синтаксис были введены процедуры, правила, пользовательские
  типы и многие другие компоненты.
  Работа не прошла даром "--- в 1995 году разработка снова разделилась:
  Стоунбрейкер использовал полученный опыт в создании коммерческой СУБД
  \textbf{Illustra}, продвигаемой его собственной одноимённой компанией
  (приобретённой впоследствии компанией Informix), а его студенты разработали
  новую версию Postgres "--- \emph{Postgres95}, в которой язык запросов
  \emph{POSTQUEL} --- наследие Ingres --- был заменен на
  \textbf{SQL}\footnote{Structured Query Language --- язык структурированных
  запросов}.

  В этот момент разработка Postgres95 была выведена за пределы университета и
  передана команде энтузиастов.
  С этого момента СУБД получила имя, под которым она известна и развивается в
  текущий момент "--- PostgreSQL.\cite{ruwiki}
  \pagebreak

\section{Возможности PostgreSQL}
  PostgreSQL считается самой совершенной СУБД, распространяемой на условиях
  открытых исходных текстов.
  В PostgreSQL реализованы многие возможности, обычно присутствующие только в
  коммерческих СУБД, таких как \emph{DB2} и \emph{Oracle}.
  Ниже перечислены основные возможности PostgreSQL.

  \begin{itemize}
    \item \emph{Объектно"=реляционная модель.}
      Работа с данными в PostgreSQL основана на объектно"=реляционной модели,
      что позволяет задействовать сложные процедуры и системы правил.
    \item полное соответствие принципам \emph{ACID} --- атомарность,
      непротиворечивость, изолированность, сохранность данных.
      \begin{itemize}
        \item \textbf{Atomicity} --- транзакция рассматривается как единая
          логическая единица, все ее изменения или сохраняются целиком, или
          полностью откатываются.
        \item \textbf{Consistency} --- транзакция переводит базу данных из
          одного  непротиворечивого состояния (на момент старта транзакции) в
          другое непротиворечивое состояние (на момент завершения транзакции).
          Непротиворечивым считается состояние базы, когда выполняются все
          ограничения физической и логической целостности базы данных, при
          этом допускается нарушение ограничений целостности в течение
          транзакции, но на момент завершения все ограничения целостности, как
          физические, так и логические, должны быть соблюдены.
        \item \textbf{Isolation} --- изменения данных при конкурентных
          транзакциях изолированы друг от друга на основе системы
          версионности.
        \item \textbf{Durability} --- PostgreSQL заботится о том, что
          результаты успешных транзакций гарантировано сохраняются на жесткий
          диск вне зависимости от сбоев аппаратуры.
      \end{itemize}
    \item \emph{Простота расширения.}
      В PostgreSQL поддерживаются пользовательские операторы, функции, методы
      доступа и типы данных.
    \item \emph{Полноценная поддержка SQL.}
    \item \emph{Проверка целостности ссылок.}
      PostgreSQL поддерживает проверку целостности ссылок, обеспечивающую
      правильность данных в базе.
    \item \emph{Гибкость API\footnote{Application programming interface ---
        \label{p:langs}
      Интерфейс программирования приложений}}.
      В настоящее время, существуют программные интерфейсы для:

      \begin{minipage}{0.26\textwidth}
        \begin{itemize}\tt
          \item Object Pascal,
          \item Python,
          \item Perl,
          \item PHP,
        \end{itemize}
      \end{minipage}
      \begin{minipage}{0.25\textwidth}
        \begin{itemize}\tt
          \item ODBC,
          \item Java/JDBC,
          \item Ruby,
          \item TCL,
        \end{itemize}
      \end{minipage}
      \begin{minipage}{0.25\textwidth}
        \begin{itemize}\tt
          \item C/C++,
          \item Lua,
          \item Javascript,
          \item LOLCODE,
        \end{itemize}
      \end{minipage}
      \begin{minipage}{0.20\textwidth}
        \begin{itemize}\tt
          \item OpenCL,
          \item Parrot,
          \item R,
          \item Scheme.
        \end{itemize}
      \end{minipage}
    \item \emph{Процедурные языки запросов}.
      В PostgreSQL предусмотрена поддержка внутренних процедурных языков, в
      том числе, специализированной языка \texttt{PL/pgSQL}, являющегося
      аналогом \texttt{PL/SQL}, процедурного языка \emph{Oracle}.
      Одним из преимуществ PostgreSQL является возможность использования
      \texttt{Perl}, \texttt{Python} и \texttt{TCL} в качестве внутренних
      процедурных языков.
    \item \emph{MVCC\footnote{Multi-Version Concurrency Control}.}
      Используется для поддержания согласованности данных в конкурентных
      условиях, в то время как в традиционных базах данных используются
      блокировки.
      MVCC означает, что каждая транзакция видит копию данных (версию базы
      данных) на время начала транзакции, несмотря на то, что состояние базы
      могло уже измениться.
      Это защищает транзакцию от несогласованных изменений данных, которые
      могли быть вызваны (другой) конкурентной транзакцией, и обеспечивает
      изоляцию транзакций.
      Основной выигрыш от использования MVCC по сравнению с блокировкой
      заключается в том, что блокировка, которую ставит MVCC для чтения не
      конфликтует с блокировкой на запись, и поэтому чтение никогда не
      блокирует запись и наоборот.
      Конкурентные операции записи ,,мешают'' друг другу только при работе с
      одной и той же записью.
    \item \emph{Клиент"=сервер.}
      В PostgreSQL используется клиент"=серверная модель.
      Сессия PostgreSQL состоит из двух взаимодействующих процессов (программ):
      \begin{itemize}
        \item \textbf{Процесс"=сервер}, который управляет файлами базы данных,
          принимает соединения от клиентов и выполняет операции над базой
          данных по запросам клиентов.
          Эта серверная программа называется \texttt{postgres}.
        \item \textbf{Пользовательский клиент"=процесс} (frontend), который,
          собственно, работает с базой данных.
          Клиентские приложения могут сильно отличаться друг от друга:
          клиент может быть как приложением с текстовым интерфейсом, так и с
          графическим интерфейсом;
          как веб-сервером, генерирующим страницу на основании данных, так и
          специализированным программой обслуживания базы данных.
          Некоторые клиентские программы поставляются вместе с сервером,
          большинство же разрабатываются пользователями.
      \end{itemize}

      Клиент и сервер могут находиться как на одной машине, так и на разных.
      В случае разных хостов, они общаются через соединение
      \texttt{TCP/IP\footnote{Transmission Control Protocol/Internet Protocol
      --- Протокол управления передачей данных}}.

      Сервер PostgreSQL может обслуживать множество конкурирующих соединений
      клиентов.
      Для этого сервер создаёт по процессу (,,форкает'') для каждого клиента.
      После этого клиент и новый процесс сервера общаются без вмешательства
      главного процесса \texttt{postgres}.
      Таким образом, процесс \texttt{postgres} всегда запущен в ожидании
      соединений с клиентами, в то время как его сабпроцессы работают
      непосредственно с клиентами.
      Вся эта архитектура совершенно прозрачна для клиента.\cite{ofdoc}

    \item \emph{WAL\footnote{Write Ahead Logging --- Опережающая регистрация
      изменений}.} --- общепринятый механизм протоколирования всех транзакций,
      что позволяет восстановить систему после возможных сбоев.
      Основная идея WAL состоит в том, что все изменения должны записываться в
      файлы на диск только после того, как эти записи журнала, описывающие эти
      изменения будут и гарантировано записаны на диск.
      Это позволяет не сбрасывать страницы данных на диск после фиксации
      каждой транзакции, так как мы знаем и уверены, что сможем всегда
      восстановить базу данных используя журнал транзакций.
    \item \emph{PITR\footnote{Point in Time Recovery}} --- возможность
      восстановления базы данных (используя WAL) на любой момент в прошлом,
      что позволяет осуществлять непрерывное резервное копирование кластера
      PostgreSQL.
    \item \emph{Репликация} также повышает надежность PostgreSQL.
      Существует несколько систем репликации, например, Slony (тестируется
      версия 1.1), который является свободным и самым используемым решением,
      поддерживает master-slaves репликацию. Ожидается, что Slony-II будет
      поддерживать multi-master режим.
    \item \emph{Целостность данных} является сердцем PostgreSQL.
      Помимо MVCC, PostgreSQL поддерживает целостность данных на уровне схемы
      --- это внешние ключи\footnote{foreign keys},
      ограничения\footnote{Constraints}.
    \item \emph{Модель развития PostgreSQL}, которая абсолютно прозрачна для
      любого, так как все планы, проблемы и приоритеты открыто обсуждаются.
      Пользователи и разработчики находятся в постоянном диалоге через мэйлинг
      листы.
      Все предложения, патчи проходят тщательное тестирование до принятия их в
      программное дерево.
      Большое количество бета-тестеров способствует тестированию версии до
      релиза и вычищению мелких ошибок.
    \item \emph{Открытость кодов} PostgreSQL означает их абсолютную
      доступность для любого, а либеральная BSD лицензия не накладывает
      никаких ограничений на использование кода.
    \item \emph{Поддержка индексов}
      \begin{itemize}
        \item \textbf{Стандартные индексы} --- B-tree, hash, R-tree, GiST (обобщенное
          поисковое дерево)
        \item \textbf{Частичные индексы\footnote{partial indices}} --- можно
          создавать индекс по ограниченному подмножеству значений, например,

          \verb'create index idx_partial on foo (x) where x > 0;'
        \item \textbf{Функциональные индексы\footnote{expressional indices}}
          позволяют создавать индексы используя значения функции от параметра,
          например,

          \verb'create index idx_functional on foo ( length(x) );'
      \end{itemize}
    \item \emph{Планировщик запросов} основывается на стоимости различных
      планов, учитывая множество факторов.
      Он предоставляет возможность пользователю отлаживать запросы и
      настраивать систему.
    \item \emph{Система блокировок} поддерживает блокировки на нижнем уровне,
      что позволяет сохранять высокий уровень конкурентности при защите
      целостности данных.
      Блокировка поддерживается на уровне таблиц и записей. На нижнем уровне,
      блокировка для общих ресурсов оптимизирована под конкретную ОС и
      архитектуру.
    \item \emph{Управление буферами и кэширование} используют сложные
      алгоритмы для поддержания эффективности использования выделенных
      ресурсов памяти.
    \item \emph{Tablespaces\footnote{табличные пространства}} для управления
      хранения данных на уровне объектов, таких как базы данных, схемы,
      таблицы и индексы.
      Это позволяет гибко использовать дисковое пространство и повышает
      надежность, производительность, а также способствует масштабируемости
      системы.
    \item \emph{Масштабируемость} основывается на описанных выше возможностях.
      Низкая требовательность PostgreSQL к ресурсам и гибкая система
      блокировок обеспечивают его шкалирование, в то время как индексы и
      управление буферами обеспечивают хорошую управляемость системы даже при
      высоких загрузках.
  \end{itemize}\cite{prpsql}

  СУБД PostgreSQL работает на таких операционных системах, как
  \begin{itemize}
    \item GNU/Linux,
    \item *BSD,
    \item UNIX,
    \item Android,
    \item Windows
  \end{itemize}\cite{enwiki}

  \pagebreak

\section{Типы данных}
  Так как язык SQL относится к \emph{языкам с сильной типизацией}, с любым
  объектом данных, представленным в PostgreSQL, связывается определённый тип,
  даже если на первый взгляд это и не очевидно.
  Тип данных одновременно определяет и ограничивает разновидности операций,
  которые могут выполняться с этими данными.

  Типы не только ассоциируются со всеми данными, но и играют важную роль при
  создании таблиц.
  Таблицы состоят из одного или нескольких полей.
  При создании таблицы каждому полю, помимо имени, назначается определённый
  тип данных.

  В таблице~\ref{tab:types} перечислены встроенные типы данных.
  Большинство имён из колонки \emph{псевдоним} используются внутри реализации
  PostgreSQL.
  Типы, выделенные \textbf{полужирным}, являются частью стандарта
  SQL.\cite{upandrunning}

  \topcaption{Типы данных PostgreSQL}\label{tab:types}
  \tablefirsthead{\hline%
                  \textbf{Тип данных} & \textbf{Псевдоним} &%
                  \textbf{Описание}\\  \hline}
  \tablehead{\multicolumn{3}{c}%
            {\tablename\ \thetable{} "--- продолжение}\\%
            \hline%
            \textbf{Тип данных} & \textbf{Псевдоним} & \textbf{Описание}\\%
            \hline}
  \tablelasthead{\multicolumn{3}{c}%
                  {\tablename\ \thetable{} "--- окончание}\\%
                \hline%
                \textbf{Тип данных} & \textbf{Псевдоним} &%
                \textbf{Описание}\\ \hline }
  \tabletail{\hline%
              \multicolumn{3}{|r|}%
              {Продолжение на следующей странице}\\%
             \hline}
  \tablelasttail{\hline}
  \begin{center}
    \begin{mpxtabular}{|p{0.31\textwidth}|p{0.22\textwidth}|p{0.47\textwidth}|}
      \texttt{\bf boolean} & \texttt{bool} & отдельная логическая величина\\
      \texttt{\bf bit [ (n) ]} & & битовая последовательность фиксированной %
              длины (ровно \texttt{n} бит)\\
      \texttt{\bf bit varying [ (n) ]} & \texttt{varbit} & битовая %
              последовательность переменной длины (до \texttt{n} бит)\\
      \hline
      \texttt{character [ (n) ]} & \texttt{\bf char [ (n) ]} & символьная %
              строка фиксированной длины (ровно \texttt{n} символов)\\
      \texttt{\bf character varying [ (n) ]} & \texttt{\bf varchar [ (n) ]} &%
              символьная строка переменной длины (до \texttt{n} символов)\\
      \texttt{text} & & символьная строка переменной или неограниченной %
              длины\\
      \hline
      \texttt{\bf smallint} & \texttt{int2} & 2"=байтовое целое со знаком\\
      \texttt{\bf integer} & \texttt{int, int4} & 4"=байтовое целое со %
              знаком\\
      \texttt{\bf bigint} & \texttt{int8} & 8"=байтовое целое со знаком\\
      \texttt{\bf real} & \texttt{float4} & 4"=байтовое вещественное число\\
      \texttt{\bf double precision} & \texttt{float8} & 8"=байтовое %
              вещественное число\\
      \texttt{\bf numeric [ (p, s) ]} & \texttt{\bf decimal [ (p, s) ]} & %
              число из {\tt p} цифр, содержащее {\tt s} цифр в дробной части\\
      \texttt{money} & & фиксированная точность, представление денежных %
              величин\\
      \texttt{smallserial} & \texttt{serial2} & 2"=байтовое целое с %
              автоматическим приращением\\
      \texttt{serial} & \texttt{serial4} & 4"=байтовое целое с %
              автоматическим приращением\\
      \texttt{bigserial} & \texttt{serial8} & 8"=байтовое целое с %
              автоматическим приращением\\
      \hline
      \texttt{\bf date} & & календарная дата (день, месяц и год)\\
      \texttt{\bf time [ (p) ]} & & время суток\\
      \texttt{\bf time [ (p) ] with time zone} & \texttt{timetz} & время %
              суток с информацией о часовом поясе\\
      \texttt{\bf timestamp [ (p) ]} & & дата и время\\
      \texttt{\bf timestamp [ (p) ] with time zone } & \texttt{timestamptz} &%
              дата и время с информацией о часовом поясе\\
      \texttt{\bf interval [ fields ] [ (p) ]} & & произвольный интервал %
              времени\\
      \hline
      \texttt{box} & & прямоугольник на плоскости\\
      \texttt{circle} & & круг с заданным центром и радиусом\\
      \texttt{line} & & прямая на плоскости\\
      \texttt{lseg} & & отрезок на плоскости\\
      \texttt{path} & & геометрическая фигура на плоскости\\
      \texttt{point} & & точка на плоскости\\
      \texttt{polygon} & & многоугольник на плоскости\\
      \hline
      \texttt{cidr} & & спецификация сети \texttt{IPv4} или \texttt{IPv6}\\
      \texttt{inet} & & сетевой адрес \texttt{IPv4} или \texttt{IPv6} с %
              необязательными битами подсети\\
      \texttt{macaddr} & & MAC\protect\Footnote{a}{Media Access %
                                                   Control}"=адрес\\
      \hline
      \texttt{bytea} & & бинарные данные (массив байтов)\\
      \texttt{tsquery} & & запрос поиска текста\\
      \texttt{tsvector} & & документ поиска текста\\
      \texttt{txid\_snapshot} & & пользовательский ID снимка транзакции\\
      \texttt{uuid} & & универсальный уникальный идентификатор\\
      \texttt{\bf xml} & & данные XML\Footnote{b}{Extensible Markup %
                                                  Language}\\
      \texttt{json} & & данные JSON\Footnote{c}{JavaScript Object Notation}\\
    \end{mpxtabular}
  \end{center}

  Ещё PostgreSQL поддерживает
  \begin{itemize}
    \item Массивы согласно стандарту \verb'SQL:2003'
    \item Большие объекты (Large Objects) позволяют хранить в базе данных
      бинарные данные размером до 2Gb.
    \item ГИС (GIS) типы в PostgreSQL являются доказательством расширяемости
      PostgreSQL и позволяют эффективно работать с трехмерными данными.
      Подробности можно найти на сайте проекта PostGis.
    \item \emph{Композитные тип\footnote{composite types}} объединяют один или
      несколько элементарных типов и позволяют пользователям манипулировать со
      сложными объектами.
  \end{itemize}

  \pagebreak
\section{Команды и функции}
  Любая концептуальная информация о реляционных базах данных и таблицах
  приносит пользу лишь в том случае, если Вы знаете, как организовать
  взаимодействие с данными.
  Язык SQL состоит из структурированных команд, предназначенных для
  добавления, модификации и удаления данных из базы.
  Эти команды образуют основу для взаимодействия с сервером PostgreSQL.

  \subsection{Команды}
    \begin{table}[h]
      \centering
      \caption{Основные действия PostgreSQL}\label{tab:commands}
      \begin{tabular}[c]{l|l}
        \hline
        \textbf{Действие} & \textbf{Описание}\\
        \hline
        \texttt{CREATE DATABASE} & Создание новой базы данных\\
        \texttt{CREATE INDEX} & Создание нового индекса для столбца таблицы\\
        \texttt{CREATE SEQUENCE} & Создание новой последовательности в базе
                                  данных\\
        \texttt{CREATE TABLE} & Создание новой таблицы в базе данных\\
        \texttt{CREATE TRIGGER} & Создание нового определения триггера\\
        \texttt{CREATE VIEW} & Создание нового представления для таблицы\\
        \texttt{SELECT} & Выборка записей из таблицы\\
        \texttt{INSERT} & Вставка записей в таблицу\\
        \texttt{UPDATE} & Модификация данных\\
        \texttt{DELETE} & Удаление записей из таблицы\\
        \texttt{DROP DATABASE} & Уничтожение базы данных\\
        \texttt{DROP INDEX} & Удаление индекса столбца\\
        \texttt{DROP SEQUENCE} & Уничтожение генератора последовательности\\
        \texttt{DROP TABLE} & Уничтожение таблицы\\
        \texttt{DROP TRIGGER} & Уничтожение определения триггера\\
        \texttt{DROP VIEW} & Уничтожение представления\\
        \texttt{CREATE USER} & Создание новой учётной записи пользователя\\
        \texttt{ALTER USER} & Модификация учётной записи\\
        \texttt{DROP USER} & Удаление учётной записи\\
        \texttt{GRANT} & Предоставление прав доступа к объекту базы данных\\
        \texttt{REVOKE} & Лишение прав доступа к объекту базы данных\\
        \texttt{CREATE FUNCTION} & Создание функции SQL\\
        \texttt{CREATE LANGUAGE} & Создание определения языка\\
        \texttt{CREATE OPERATOR} & Создание оператора SQL\\
        \texttt{CREATE TYPE} & Создание типа данных SQL\\
        \hline
      \end{tabular}
    \end{table}

    Операция \texttt{SELECT} поддерживает:
    \begin{itemize}
      \item JOIN.
        Возможные варианты синтаксиса:
        \begin{itemize}
          \item \verb't1 { [INNER] | { LEFT | RIGHT | FULL } [OUTER] } JOIN t2 ON be'
          \item \verb't1 { [INNER] | { LEFT | RIGHT | FULL } [OUTER] } JOIN t2 USING ( list )'
          \item \verb't1 NATURAL { [INNER] | { LEFT | RIGHT | FULL } [OUTER] } JOIN t2'
        \end{itemize}
      \item Псевдонимы таблиц/столбцов.
      \item Подзапросы (в поле \verb'FROM' может быть другой запрос).
      \item Табличные функции.
      \item Клаузы \verb'GROUP BY' и \verb'HAVING' в поле \verb'WHERE'.
      \item Клауза \verb'DISTINCT' --- фильтрация дублирующихся строчек.
      \item Клауза \verb'ORDER BY', имеющая синтаксис:

        \verb'SELECT ... FROM ... ORDER BY se [ASC | DESC] [NULLS {FIRST | LAST}] [, ...]'
    \end{itemize}

  \subsection{Операторы}
    \begin{itemize}
      \item Логические:
        \verb'AND, OR, NOT'
      \item Сравнения:
        \begin{itemize}
          \item \verb'<, >, <=, >=, =, <>' или \verb'!=' (оператор
                      \verb'!=' преобразуется в \verb'<>' во время разбора
                                              (парсинга) строки)
          \item \verb'a [NOT] BETWEEN x AND y', что равносильно %
            \verb'a >= x AND a <= y'
          \item \verb'IS [NOT] NULL'
          \item \verb'ISNULL' --- нестандартный оператор, эквивалентный
            \verb'IS NULL'
          \item \verb'NOTNULL' --- нестандартный оператор, эквивалентный
            \verb'IS NOT NULL'.
            \verb'NULL <> NULL', однако можно установить параметр
            \emph{transform\_null\_equals} и сервер PostgreSQL будет
            автоматически трансформировать выражения типа \verb'x = NULL' в
            \verb'x IS NULL'.
          \item \verb'IS [NOT] DISTINCT FROM'
        \end{itemize}
      \item Математические:
        \begin{itemize}
          \item \verb'+, -, *, /, %, ^'
          \item \verb'|/' --- квадратный корень
          \item \verb'||/' --- кубический корень
          \item \verb'!'
          \item \verb'!!' --- префиксный факториал
          \item \verb'@' --- модуль
        \end{itemize}
      \item Побитовые:
        \begin{itemize}
          \item \verb'&, |'
          \item \verb'#' --- XOR
          \item \verb'~' --- NOT
          \item \verb'<<, >>'
        \end{itemize}
      \item Строковые:
        \begin{itemize}
          \item \verb'string || string', \verb'string || non-string',
            \verb'non-string || string' --- конкатенация
        \end{itemize}
    \end{itemize}

  \subsection{Математические функции}
    Среди математических функций, реализованных в PostgreSQL, можно отметить
    следующие:
    \verb'abs', \verb'ceiling', \verb'cbrt', \verb'degrees', \verb'div',
    \verb'exp', \verb'floor', \verb'ln', \verb'log', \verb'mod', \verb'pi',
    \verb'power', \verb'radians', \verb'random', \verb'round', \verb'setseed',
    \verb'sign', \verb'trunc', \verb'width_bucket', \verb'acos', \verb'asin',
    \verb'atan', \verb'atan2', \verb'cos', \verb'cot', \verb'sin', \verb'tan'.

  \subsection{Функции для работы со строками}
    В стандарте SQL определены некоторые функции для работы со строками так,
    что они используют некоторые ключевые слова, а не запятые, для разделения
    аргументов.
    PostgreSQL предоставляет также возможности этих функций с обычным
    синтаксисом вызова функций.

    В PostgreSQL есть следующие функции для работы со строками:
    \verb'bit_length', \verb'char_length', \verb'character_length',
    \verb'lower', \verb'octet_length', \verb'overlay', \verb'postion',
    \verb'substring', \verb'trim', \verb'upper', \verb'ascii', \verb'btrim',
    \verb'chr', \verb'concat', \verb'concat_ws', \verb'convert',
    \verb'convert_from', \verb'convert_to', \verb'encode', \verb'format',
    \verb'initcap', \verb'left', \verb'length', \verb'lpad', \verb'ltrim',
    \verb'md5', \verb'pg_client_encoding', \verb'strpos', \verb'quote_ident',
    \verb'quote_literal', \verb'quote_nullable', \verb'regexp_mathes',
    \verb'regexp_replace', \verb'to_hex', \verb'regexp_split_to_array',
    \verb'regexp_split_to_table', \verb'repeat', \verb'replace',
    \verb'reverse', \verb'right', \verb'rpad', \verb'rtrim',
    \verb'split_part', \verb'substr', \verb'to_ascii',
    \verb'translate'.

  \pagebreak

\section{Особенности языка запросов}
  Кроме основных возможностей, присущих любой SQL базе данных, PostgreSQL
  поддерживает:\cite{cookbook}
  \begin{itemize}
    \item Очень высокий уровень соответствия ANSI SQL 92, ANSI SQL 99 и ANSI
      SQL 2003.
    \item \emph{Схемы}, которые обеспечивают пространство имен на уровне
      SQL.
      Схемы содержат таблицы, в них можно определять типы данных, функции и
      операторы.
      Используя полное имя объекта можно одновременно работать с несколькими
      схемами.
      Схемы позволяют организовать базы данных совокупность нескольких
      логических частей, каждая их которых имеет свою политику доступа, типы
      данных.
      Для приложений, которые создают новые объекты в базе данных удобно и
      безопасно создавать отдельную схему (и включать ее в SEARCH\_PATH) с
      тем, чтобы избежать возможной коллизии с именами объектов и удобством
      обновления приложения.
    \item \emph{Subqueries} --- подзапросы (subselects), полная поддержка
      SQL92.
      Подзапросы делают язык SQL более гибким и зачастую более эффективным.
    \item \emph{Outer Joins} --- внешние связки (LEFT,RIGHT, FULL).
    \item \emph{Rules} --- правила, согласно которым модифицируется исходный
      запрос.
      Главное отличие от триггеров состоит в том, что rule работает на
      уровне запроса и \textbf{перед} исполнением запроса, а триггер --- это
      реакция системы на изменение данных, т.\,е. триггер запускается
      \textbf{в процессе} исполнения запроса для каждой измененной записи
      (PER ROW).
      Правила используются для указания системе, какие действия надо
      произвести при попытке обновления \emph{view}.
    \item \emph{Views} --- представления, виртуальные таблицы.
      Реальных экземпляров этих таблиц не существуют, они материализуются
      только при запросе.
      Одним из основных предназначений \emph{view} является разделение прав
      доступа к родительским таблицам и к \emph{view}, а также обеспечение
      постоянства пользовательского интерфейса при изменении родительских
      таблиц.
      Обновление \emph{view} (материализация) возможно в PostgreSQL с
      помощью \verb'PL/pgSQL'.
    \item \emph{Cursors} --- курсоры, позволяют уменьшить трафик между
      клиентом и сервером, а также память на клиенте, если требуется
      получить не весь результат запроса, а только его часть.
    \item \emph{Table Inheritance} --- наследование таблиц, позволяющее
      создавать объекты, которые наследуют структуру родительского объекта и
      добавлять свои специфические атрибуты.
      При этом наследуются значения атрибутов по умолчанию (DEFAULTS) и
      ограничение целостности (CONSTRAINTS).
      Поиск по родительской таблице автоматически включает поиск по дочерним
      объектам, при этом сохраняется возможность поиска только по ней
      (only).
      Наследование можно использовать для работы с очень большими таблицами
      для эмуляции partitioning.
    \item \emph{Prepared Statements} (преподготовленные запросы) --- это
      объекты, живущие на стороне сервера, которые представляют собой
      оригинальный запрос после команды PREPARE, который уже прошел стадии
      разбора запроса (parser), модификации запроса (rewriting rules) и
      создания плана выполнения запроса (planner), в результате чего, можно
      использовать команду EXECUTE, которая уже не требует прохождения этих
      стадий.
      Для сложных запросов это может быть большим выигрышем.
    \item \emph{Stored Procedures} --- серверные (хранимые) процедуры
      позволяют реализовывать бизнес логику приложения на стороне сервера.
      Кроме того, они позволяют сильно уменьшить трафик между клиентом и
      сервером.
    \item \emph{Savepoints} (nested transactions) --- в отличие от ,,плоских
      транзакций'', которые не имеют промежуточных точек фиксации,
      использование savepoints позволяет отменять работу части транзакции,
      например вследствии ошибочно введенной команды, без влияния на
      оставшуюся часть транзакции.
      Это бывает очень полезно для транзакций, которые работают с большими
      объемами данных.
    \item Права доступа к объектам системы на основе системы привилегий.
      Владелец объекта или суперюзер может как разрешать доступ (GRANT), так
      и отменять (REVOKE).
    \item Система обмена сообщениями между процессами --- LISTEN и NOTIFY
      позволяют организовывать событийную модель взаимодействия между
      клиентом и сервером (клиенту передается название события, назначенное
      командой notify и PID процесса).
    \item \emph{Триггеры} позволяют управлять реакцией системы на изменение
      данных (INSERT,UPDATE,DELETE), как перед самой операцией (BEFORE), так
      и после (AFTER).
      Во время выполнения триггера доступны специальные переменные NEW
      (запись, которая будет вставлена или обновлена) и OLD (запись перед
      обновлением).
    \item \emph{Cluster table} --- упорядочивание записей таблицы на диске
      согласно индексу, что иногда за счет уменьшения доступа к диску
      ускоряет выполнение запроса.
  \end{itemize}\cite{highperf}
  \pagebreak

\section{Ограничения}
  \begin{table}[h]
    \caption{Некоторые ограничения PostgreSQL}\label{tab:limits}
    \begin{tabular}[c]{|p{0.6\textwidth}|p{0.4\textwidth}|}
      \hline
      \textbf{Название} & \textbf{Значение}\\
      \hline
      Максимальный размер БД & $\infty$\\
      Максимальный размер таблицы & \verb'32 TiB'\\
      Максимальная длина записи & \verb'400 GiB'\\
      Максимальная длина атрибута & \verb'1 GiB'\\
      Максимальное количество записей в таблице & $\infty$\\
      Максимальное количество атрибутов в таблице & 250--1600 в зависимости от
      типа атрибута\\
      Максимальное количество индексов на таблицу & $\infty$\\
      \hline
    \end{tabular}
  \end{table}
\section{Безопасность}
  Безопасность данных также является важнейшим аспектом любой СУБД.
  В PostgreSQL она обеспечивается 4-мя уровнями безопасности:
  \begin{enumerate}
    \item PostgreSQL нельзя запустить под привилегированным пользователем ---
      \emph{системный контекст};
    \item SSL\footnote{Secure Sockets Layer}, SSH\footnote{Secure SHell}
      шифрование трафика между клиентом и сервером --- \emph{сетевой
      контекст};
    \item сложная система аутентификации на уровне хоста или IP
      адреса/подсети.
      Система аутентификации поддерживает пароли, шифрованные пароли,
      Kerberos, IDENT и прочие системы, которые могут подключаться используя
      механизм подключаемых аутентификационных модулей.
    \item Детализированная система прав доступа ко всем объектам базы данных,
      которая совместно со схемой, обеспечивающая изоляцию названий объектов
      для каждого пользователя, PostgreSQL предоставляет богатую и гибкую
      инфраструктуру.
  \end{enumerate}
  \pagebreak

\section{Установка соединений}
  Интерфейсы в PostgreSQL реализованы для доступа к базе данных из ряда языков
  (основной список был перечислен на странице \pageref{p:langs}) и методов
  доступа к данным:
  \begin{itemize}
    \item \emph{JDBC}\footnote{англ. Java DataBase Connectivity --- соединение
      с базами данных на Java} --- платформенно"=независимый промышленный
      стандарт взаимодействия Java"=приложений с различными СУБД,
      реализованный в виде пакета \verb'java.sql', входящего в состав Java SE.

      JDBC основан на концепции так называемых драйверов, позволяющих получать
      соединение с базой данных по специально описанному URL.
      Драйверы могут загружаться динамически (во время работы программы).
      Загрузившись, драйвер сам регистрирует себя и вызывается автоматически,
      когда программа требует URL, содержащий протокол, за который драйвер
      отвечает.
    \item \emph{ODBC}\footnote{англ. Open Database Connectivity} --- это
      программный интерфейс (API) доступа к базам данных, разработанный фирмой
      Microsoft, в сотрудничестве с Simba Technologies на основе спецификаций
      Call Level Interface (CLI), который разрабатывался организациями SQL
      Access Group, X/Open и Microsoft.
      Стандарт CLI призван унифицировать программное взаимодействие с СУБД,
      сделать его независимым от поставщика СУБД и программно"=аппаратной
      платформы.
  \end{itemize}
  \pagebreak
\section{Локализация}
  PostgreSQL поддерживает 25 различных наборов символов (charsets), включая
  ASCII, LATIN, WIN, KOI8 и UNICODE, а также locale, что позволяет корректно
  работать с данными на разных языках.

  Также присутствует поддержка \emph{NLS} (Native Language Support) ---
  документация, сообщения об ошибках доступны на различных языках, включая
  японский, немецкий, итальянский, французский, русский, испанский,
  португальский, словенский, словацкий и несколько диалектов китайского
  языков.
  \pagebreak
\section{Клиенты и инструментарий}
  \begin{itemize}
    \item Утилита \emph{psql} (входит в дистрибутив) предоставляет удобный
      интерфейс для работы с базой данных, содержит краткий справочник по SQL,
      облегчает ввод команд (используя стрелки для повтора и табулятор для
      расширения), поддерживает историю и буфер запросов, а также позволяет
      работать как в интерактивном режиме, так и потоковом режиме.
    \item \emph{phpPgAdmin} (распространяется под лицензией
      \textbf{GPL}\footnote{GNU General Public License}) представляет
      возможность с помощью веб"=браузера администрировать PostgreSQL кластер.
    \item \emph{pgAdmin III} (GNU Artistic license) предоставляет удобный
      интерфейс для работы с базами данных PostgreSQL и работает под Linux,
      FreeBSD и Windows 2000/XP.
    \item \emph{PgEdit} --- программная среда для разработки приложений и
      SQL"=редактор, доступна для Windows и Mac.
    \item \ldots
  \end{itemize}\cite{wtfispg}
  \pagebreak

\addcontentsline{toc}{section}{Литература}
\bibliographystyle{ugost2008ls}
\bibliography{src}
\end{document}
