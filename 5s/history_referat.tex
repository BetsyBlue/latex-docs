\documentclass[a4paper,12pt,titlepage,pdftex,headsepline]{scrartcl}

\usepackage{a4wide}
\usepackage{cmap} % чтобы работал поиск по PDF
\usepackage[utf8]{inputenc}
\usepackage[russian]{babel}
\usepackage[T2A]{fontenc}

\usepackage{textcase}
\usepackage[pdftex]{graphicx}

\usepackage{lscape}

\pdfcompresslevel=9 % сжимать PDF
\usepackage{pdflscape} % для возможности альбомного размещения некоторых страниц
\usepackage[pdftex]{hyperref}
% настройка ссылок в оглавлении для pdf формата
\hypersetup{unicode=true,
            pdftitle={РГР по Дифференциальным уравнениям},
            pdfauthor={Погода Михаил},
            pdfcreator={pdflatex},
            pdfsubject={},
            pdfborder    = {0 0 0},
            bookmarksopen,
            bookmarksnumbered,
            bookmarksopenlevel = 2,
            pdfkeywords={},
            colorlinks=true, % установка цвета ссылок в оглавлении
            citecolor=black,
            filecolor=black,
            linkcolor=black,
            urlcolor=blue}

\author{Михаил Погода\\
КМ-92\\
ФПМ}
\title{Реферат по истории науки и техники\\
по теме:\\
,,История параллельных и распределённых вычислений''}
\date{\today}

\begin{document}
\maketitle
Необходимость разделять вычислительные задачи и выполнять их одновременно (параллельно) возникла задолго до появления первых вычислительных машин.

В конце XVIII века во Франции под руководством Гаспара де Прони была начата работа по уточнению логарифмических и тригонометрических таблиц в связи с переходом на метрическую систему.
Для её выполнения был необходим огромный по тем временам объем вычислений.
Исполнители проекта были разбиты на три уровня:
\begin{itemize}
\item квалифицированные специалисты-вычислители, от которых требовалась аккуратность при проведении вычислений;
\item организаторы распределения заданий и обработки полученных результатов;
\item организаторы подготовки математического обеспечения и обобщения полученных результатов (высший уровень, в состав которого входили Адриен Лежандр и Лазар Карно).
\end{itemize}

Работа не была завершена из-за революционных событий 1799 года, однако идеи де Прони подтолкнули Чарльза Бэббиджа к созданию аналитической машины.

Решение для модели атомной бомбы в США было получено коллективом учёных, которые пользовались вычислительными машинами.

В 1973 году Джон Шох и Джон Хапп из калифорнийского научно-исследовательско\-го центра Xerox PARC написали программу, которая по ночам запускалась в локальную сеть PARC и заставляла работающие компьютеры выполнять вычисления.

В 1978 году отечественный математик В.\,М.\,Глушков работал над проблемой макроконвейерных, распределённых вычислений.
Он предложил ряд принципов распределения работы между процессорами.
На базе этих принципов им была разработана \texttt{ЭВМ ЕС-2701}.

В 1988 году Арьен Ленстра и Марк Менес написали программу для факторизации длинных чисел.
Для ускорения процесса программа могла запускаться на нескольких машинах, каждая из которых обрабатывала свой небольшой фрагмент. 
Новые блоки заданий рассылались на компьютеры участников с центрального сервера проекта по электронной почте.
Для успешного разложения на множители числа длиной в сто знаков этому сообществу потребовалось два года и несколько сотен персональных компьютеров.

С появлением и бурным развитием интернета всё большую популярность стала получать идея добровольного использования для распределённых вычислений компьютеров простых пользователей, соединённых через интернет.

В январе 1996 года стартовал проект \texttt{GIMPS} по поиску простых чисел Мерсенна, используя компьютеры простых пользователей как добровольную вычислительная сеть.

28 января 1997 года стартовал конкурс \texttt{RSA Data Security} на решение задачи взлома методом простого перебора 56-битного ключа шифрования информации \texttt{RC5}.
Благодаря хорошей технической и организационной подготовке проект, организованный некоммерческим сообществом \texttt{distributed.net}, быстро получил широкую известность.

17 мая 1999 года на базе платформы \texttt{BOINC} запущен проект \texttt{SETI@home}, занимающийся поиском внеземного разума путём анализа данных с радиотелескопов, используя добровольную вычислительная сеть на базе \texttt{Grid}.

Такие проекты распределённых вычислений в интернете, как \texttt{SETI@Home} и\\ \texttt{Folding@Home} обладает не меньшей вычислительной мощностью, чем самые современные суперкомпьютеры.
Интегральная производительность проектов на платформе \texttt{BOINC} по данным на 16 мая 2010 года составляет 5,2 петафлопс.
Для сравнения, пиковая производительность самого мощного суперкомпьютера (\texttt{,,K''}, Япония) --- 8,16 петафлопс.
До середины 2011 года самым мощным суперкомпьютером являлся \texttt{Тяньхэ-1А} с производительностью ,,всего'' 2,57 петафлопс.
Проект отмечен в Книге рекордов Гиннеса как самое большое вычисление.

На сегодняшний день для упрощения процесса организации и управления распределёнными вычислениями создано множество программных комплексов, как коммерческих, так и бесплатных.

\end{document}
